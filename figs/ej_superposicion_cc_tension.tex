\documentclass{standalone}
\usepackage{mathpazo}
\usepackage{tikz}
\usepackage{circuitikz}
\usetikzlibrary{calc}
\usepackage{siunitx}

\begin{document}

\begin{circuitikz}[american resistors, american voltages, american currents]
\coordinate(A) at (8,6);
  \coordinate(B) at (8,0);
  \coordinate(C) at (8,2.5);
  \draw
 % (0,0) to [isource, l= $2$ mA] (0,3)
  (0,3) to [esource, v^<=$\qty{6}{\volt}$] (0,6)
  (0,3) to [R, l=$\qty{2}{\kilo\ohm}$] (3,3)
  (3,3) to [R, l=$\qty{2}{\kilo\ohm}$] (3,0)
  (3,3) to [R,l_=$\qty{4}{\kilo\ohm}$] (3,6)
  (6,0) to [R, l_= $\qty{6}{\kilo\ohm}$] (6,6)
  (0,0) to [short,-] (6,0)
  (0,6) to [short,-] (6,6)
  (6,6) to [short,-*] (8,6)
  (6,0) to [short,-*] (8,0)
  (A) to  [open, v = $U_{0,1}$] (B);
   \draw[thin, <-] (1.5,4.5)node{$I_a$}  ++(-60:0.5) arc (-60:170:0.5);
   \draw[thin, <-] (4.5,3)node{$I_b$}  ++(-60:0.5) arc (-60:170:0.5);
%   \draw[thin, <-] (7.5,2)node{$I_3$}  ++(-60:0.5) arc (-60:170:0.5);
\end{circuitikz}

\end{document}