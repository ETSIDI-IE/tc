\documentclass{standalone}
\usepackage{mathpazo}
\usepackage{circuitikz}
\usetikzlibrary{calc}
\newcommand{\equal}{=}

\begin{document}

\begin{circuitikz}[american voltages, european resistors]
  \draw
  (0,-0.5) node[below] {O} to [short, *-] (0,0)
  to [R, l = $Z_3$, *-] ++(0, 3)
  to [V, v= $u_{A3}$, invert] ++(0, 2);
  \draw
  (0,0) to [short] ++(-2, 0)
  to [R, l = $Z_2$, *-] ++(0, 3)
  to [V, v^<= $u_{A2}$, invert] ++(0, 2);
  \draw
  (0,0) to [short] ++(-4, 0)
  to [R, l = $Z_1$, *-] ++(0, 3)
  to [V, v^<= $u_{A1}$, invert] ++(0, 2);
  \draw
  (0,0) to [short] ++(2, 0)
  to [R, l = $Z_4$, *-] ++(0, 3)
  to [V, v<= $u_{A4}$, invert] ++(0, 2);
  \draw
  (0,0) to [short] ++(4, 0)
  to [R, l = $Z_5$, *-] ++(0, 3)
  to [V, v<= $u_{A5}$, invert] ++(0, 2);
  \draw
  (0,0) to [short, -*] ++(6, 0)
  to [R, color = red, l = $\color{red} Z_L$] ++(0, 5);
  \draw
  (0, 5.5) node[above] {A}
  to [short, *-*] ++(0,-0.5)
  to [short, -*] ++(2,0)
  to [short, -*] ++(2,0)
  to [short, -*] ++(2,0)
  (0,5) to [short, -*] ++(-2,0)
  to [short, -*] ++(-2,0);
\end{circuitikz}

\end{document}