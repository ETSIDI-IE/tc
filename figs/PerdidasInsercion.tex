\documentclass{standalone}
\usepackage{mathpazo}
\usepackage[americaninductors]{circuitikz}

\begin{document}

\begin{circuitikz}[american voltages, american currents, european resistors]
  % Entrada
  \draw
  (-6, 4) to [sV, v = $\overline{\epsilon}_g$] (-6,0)
  (-6,4) to [R, l = $\overline{Z}_g$] (-3,4)
  to [short, *-] (-2, 4);
  % Salida
  \draw 
  (-2,0) to [R, l_=$\overline{Z}_L$, -] (-2,4)
  (-2,0) to [short, -*] (-3, 0)
  to [short] (-6,0);
  % Flecha potencia
  \draw[->, dashed] (-3.5,-.7) -- node[midway, left] {$P_L$} (-3.5,1.3) -- (-2.5,1.3);
  % Cuadripolo
  \draw[fill=lightgray] (2,-1) rectangle (5,5) node[midway]{Red};
  % Entrada
  \draw
  (0, 4) to [sV, v = $\overline{\epsilon}_g$] (0,0)
  (0,4) to [R, l = $\overline{Z}_g$] (2,4)
  (2,0) to [short] (0,0);
  % Salida
  \draw (7,4) to [short] (6,4)
  to [short, -*] (6, 4) to [short] (5, 4)
  (7,0) to [R, l_=$\overline{Z}_L$, -] (7,4)
  (5,0) to [short, -*] (6, 0) to [short] (7,0);
  % Flecha potencia
  \draw[->, dashed] (5.75,-.7) -- node[midway, left] {$P_{out}$} (5.75,1.3) -- (6.5,1.3);
\end{circuitikz}
\end{document}