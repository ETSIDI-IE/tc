\documentclass{standalone}
\usepackage{mathpazo}

\usepackage[americaninductors, americanvoltages]{circuitikzgit}
\usetikzlibrary{positioning, fit, calc}

\begin{document}

\begin{circuitikz}
  \draw
  (0,0) node[transformer, yscale = 2] (T) {}
  (T.base) node{$a:1$}
  (T.inner dot A1) node[circ]{}
  (T.inner dot B1) node[circ]{};
  \draw
  (T.A1) to[short, -*, i=$I_1$] ++(-1,0) node (A) {}
  (T.A2) to[short, -*] ++(-1,0) node (B) {}
  (A) to[open,v=$U_1$] (B)
  (T.B1) to[short, -*, i<=$I_2$] ++(1,0) node (C) {}
  (T.B2) to[short, -*] ++(1,0) node (D) {}
  (C) to[open,v^=$U_2$] (D);
  \draw
  (C) to[short] ++(1.5,0) node (Cg) {}
  (D) to[short] ++(1.5,0) node (Dg) {}
  (Cg) to[sV, voltage dir = old, v= $\epsilon_g$] (Dg);
  \draw
  ([xshift=2.5cm]Cg.center) node(Z1') {}
  to[short, *-] ++(1,0) node(Z1) {}
  ([xshift=2.5cm]Dg.center) node(Z2') {}
  to[short, *-] ++(1,0) node(Z2) {}
  (Z1) to[sV, voltage dir = old, v= $a \cdot \epsilon_g$, i<= $I_1$] (Z2)
  (Z1') to[open, v=$U_1$] (Z2');
\end{circuitikz}
\end{document}