\documentclass{standalone}
\usepackage{mathpazo}
\usepackage[americaninductors]{circuitikz}

\begin{document}

\begin{circuitikz}[american voltages, american currents, european resistors]
  % Cuadripolos
  \draw[fill=lightgray] (2,-1) rectangle (4,5) node[midway]{Circuito LC};
  \draw[fill=lightgray] (5,-1) rectangle (7,5) node[midway]{Circuito LC};
  \draw[fill=lightgray] (8,-1) rectangle (10,5) node[midway]{Circuito LC};
  \draw[fill=lightgray] (11,-1) rectangle (13,5) node[midway]{Circuito LC};
  \draw
  (4,4) to [short] (5,4)
  (7,4) to [short] (8,4)
  (10,4) to [short] (11,4)
  (4,0) to [short] (5,0)
  (7,0) to [short] (8,0)
  (10,0) to [short] (11,0);
  % Entrada
  \draw
  (0, 4) to [sV, v = $\overline{\epsilon}_g$] (0,0)
  (0,4) to [R, l = $\overline{Z}_g$] (2,4)
  (2,0) to [short] (0,0);
  % Salida
  \draw (14,4) to [short] (13,4)
  (14,0) to [R, l_=$\overline{Z}_L$, -] (14,4)
  (13,0) to [short] (14,0);
  % Flecha potencia
  \draw[->, dashed] (1.25,-.7) -- node[midway, left] {$P_L$} (1.25,1.3) -- (2,1.3);
  \draw[->, dashed] (4.25,-.7) -- node[midway, right] {$P_L$} (4.25,1.3) -- (5,1.3);
  \draw[->, dashed] (7.25,-.7) -- node[midway, right] {$P_L$} (7.25,1.3) -- (8,1.3);
  \draw[->, dashed] (10.25,-.7) -- node[midway, right] {$P_L$} (10.25,1.3) -- (11,1.3);
\end{circuitikz}
\end{document}