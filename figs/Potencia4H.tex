\documentclass{standalone}
\usepackage{mathpazo}
\usepackage{tikz}
\usetikzlibrary{calc}
\usetikzlibrary{arrows}

\begin{document}

\def\wat#1#2#3{
\begin{scope}[shift={#1}]
  \node[draw, circle, radius = 0.5] (#2) at (0,0) {#3};
  \draw (#2.north) -- ++(0, 0.5) node[below left] {*}
  -| ($(#2.west) + (-.5,0)$) node[above right] {*}
  -- (#2.west);
\end{scope}
}

\begin{tikzpicture}
  %% Líneas
  \coordinate (A) at (-2,4.5);
  \coordinate (A') at ($(A) + (6, 0)$);
  \coordinate (B) at ($(A) + (0, -1.5)$);
  \coordinate (B') at ($(B) + (6,0)$);
  \coordinate (C) at ($(B) + (0, -1.5)$);
  \coordinate (C') at ($(C) + (6,0)$);
  \coordinate (N) at ($(C) + (0, -1.5)$);
  \coordinate (N') at ($(N) + (6,0)$);
  %% vatímetros
  \wat{($(A) + (1, 0)$)}{W1}{$W_1$};
  \wat{($(B) + (2.5, 0)$)}{W2}{$W_2$};
  \wat{($(C) + (4, 0)$))}{W3}{$W_3$};
  %% Salida de corriente de vatímetros
  \draw (W1.east) -- (A');
  \draw (W2.east) -- (B');
  \draw (W3.east) -- (C');
  %% Entrada de corriente de vatímetros
  \draw (A) node[left] {A} -- (W1.west);
  \draw (B) node[left] {B} -- (W2.west);
  \draw (C) node[left] {C} -- (W3.west);
  %% Neutro
  \draw (N) node[left] {N} -- (N');
  %% Salida de tensión de vatímetros
  \draw (W1.south) -- (W1.north|-N);
  \draw (W2.south) -- (W2.north|-N);
  \draw (W3.south) -- (W3.north|-N);
  %% Receptor
  \draw [rounded corners, fill= gray!10]
  ($(A') + (0, 1)$) rectangle ($(N') + (3,-1)$)
  node[midway] {Receptor 4H};
\end{tikzpicture}
\end{document}