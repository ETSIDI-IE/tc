\documentclass{standalone}
\usepackage{mathpazo}
\usepackage{tikz}
\usepackage{circuitikz}
\usetikzlibrary{calc}

\begin{document}

\begin{circuitikz}[american resistors, american voltages]
\coordinate (D) at (3,0);
\coordinate (C) at (9,0);
\coordinate (B) at (6,5);
\coordinate (A) at (0,5);
  \draw
   (0,0) to [esource, v= $E_1$] (0,5)
   (0,5) to [short,-] (-3,5)
   (-3,5) to [C, l=$C_1$, i^<=$I_9$] (-3,0)
   (-3,0) to [short,-] (0,0)
   (0,0) to [short, -] (0,0) node[ground] {}
   (0,5) to [short,-, i=$I_1$] (3,5)
   (3,5) to [R, l=$R_1$, i=$I_2$] (3,0)
   (3,0) to [short,-, i=$I_8$] (0,0)
   (3,5) to [R, l=$R_2$, i=$I_3$] (6,5)
   (6,5) to [R, l=$R_3$] (6,2)
   (6,2) to [esource, v=$E_2$, i=$I_4$] (6,0)
   (3,0) to [short,-, i^<=$I_7$] (6,0)
   (6,0) to [R, l=$R_6$, i^<=$I_6$] (9,0)
   (9,0) to [R, l=$R_5$] (9,5)
   (6,5) to [R, l=$R_4$, i=$I_5$] (9,5)
   (D) to [C, l=$C_2$, i=$I_0^1$] (3,-3)
   (3,-3) to [short,-] (6,-3)
   (6,-3) to [short, -] (6,-3) node[ground] {}
   (6,0) to [C, l=$C_3$, i=$I_1^1$] (6,-3)
   (6,-3) to [short,-] (9,-3)
   (9,0) to [C, l=$C_4$, i=$I_2^1$] (9,-3);
%   (0,3) to [R, l= $10\Omega$, i= $I_1$] (3,3)
%   (3,3) to [R, l=$10\Omega$, i=$I_2$] (3,0)
%   (3,0) to [short, -] (0,0)
%   (3,3) to [L, l= $1\mu$H, i^<= $I_8$] (6,3)
%   (6,3) to [esource, v=$60\,\text{V}$] (6,0)
%   (6,0) to [R, l= $10\Omega$, i=$I_3$] (3,0)
%   (6,0) to [C, l= $10\mu$F, i=$I_7$] (9,0)
%   (9,0) to [C, l=$20\mu$F, ] (9,3)
%   (9,3) to [short, -] (12,3)
%   (12,3) to [R, l= $30\Omega$, i=$I_4$] (12,0)
%   (12,3) to [short,-,i^<=$I_6$] (15,3)
%   (15,3) to [esource, v=$30\,\text{V}$] (15,0)
%   
%   (15,0) to [short,-] (12,0)
%   (9,0) to [R, l= $30\Omega$, i=$I_5$] (12,0);
% \node[label=above:E] (E) at ($(E)$) {};
   \node[label=below left:D] (D) at ($(D)$) {};
  \node[label=above:A] (A) at ($(A)$) {};
%   \node[label=below:F] (F) at ($(F)$) {};
  \node[label=right:C] (C) at ($(C)$) {};
  \node[label=above:B] (B) at ($(B)$) {};
   \node[label=above:+] (3.3,-2.2) at ($(3.3,-2.2)$) {};
   \node[label=above:+] (6.3,-2.2) at ($(6.3,-2.2)$) {};
   \node[label=above:+] (9.3,-2.2) at ($(9.3,-2.2)$) {};
   \node[label=above:+] (-3.3,1.8) at ($(-3.3,1.8)$) {};
%   \node[label=above:+] (8.8,1.6) at ($(8.8,1.6)$) {};
  %(6,2) to [short, -] (0,2);
%   \draw[thin, <-] (7.5,2)node{$I_3$}  ++(-60:0.5) arc (-60:170:0.5);
\end{circuitikz}

\end{document}

% \documentclass{standalone}
% \usepackage{mathpazo}
% \usepackage{circuitikz}
% \usetikzlibrary{calc}

% \begin{document}

% \begin{circuitikz}[european resistors, american voltages]
%   \coordinate (Ag) at (0,4);
%   \coordinate (Bg) at (0,0);
%   \node[label=above right:A] (A) at ($(Ag) + (2,0)$) {};
%   \node[label=below right:B] (B) at ($(Bg) + (2,0)$) {};
%   \draw
%   (Ag) to [R, l_= $Z_{th}$, i^>= $I$] (A)
%   (Ag) to [esource, v_= $\epsilon_{th}$] (Bg)
%   (Bg) to [short] (B)
%   (A) to [open, *-*, v^= $U_{AB}$] (B);
%   \draw
%   (A) to [short] ++(1.5,0)
%   to [R, l = $Z_L$] ($(B) + (1.5,0)$)
%   to [short] ++(-1.5,0)
%   to [short,*-] (B);
% \end{circuitikz}

% \end{document}