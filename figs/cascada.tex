\documentclass{standalone}
\usepackage{mathpazo}
\usepackage[americaninductors]{circuitikz}

\begin{document}

\begin{circuitikz}[american voltages, american currents]
  % Cuadripolo A
  \draw[fill=lightgray] (2,-1) rectangle (6,5) node[midway]{$C_A$};
  \draw (0,4)to [short, *-*, i^=$\mathbf{I}_{1A}$] (2,4)
  (2,0) to [short, *-*, i^=$\mathbf{I}_{1A}$] (0,0);
  \draw (2,4) to [open, v=$\mathbf{V}_{1A}$] (2,0);
  \draw (6,4) to [short, *-*, i=$\mathbf{I}'_{2A}$] (8,4)
  (8,0) to [short, *-*, i=$\mathbf{I}'_{2A}$] (6,0);
  \draw (6,4) to [open, v^=$\mathbf{V}_{2A}$] (6,0);
  % Cuadripolo B
  \draw[fill=lightgray] (10,-1) rectangle (14,5) node[midway]{$C_B$};
  \draw (8,4)to [short, *-*, i^=$\mathbf{I}_{1B}$] (10,4)
  (10,0) to [short, *-*, i^=$\mathbf{I}_{1B}$] (8,0);
  \draw (10,4) to [open, v=$\mathbf{V}_{1B}$] (10,0);
  \draw (14,4) to [short, *-*, i=$\mathbf{I}'_{2B}$] (16,4)
  (16,0) to [short, *-*, i=$\mathbf{I}'_{2B}$] (14,0);
  \draw (14,4) to [open, v^=$\mathbf{V}_{2B}$] (14,0);

\end{circuitikz}
\end{document}