\documentclass{standalone}
\usepackage{mathpazo}
\usepackage{tikz}
\usepackage{circuitikz}
\usetikzlibrary{calc}

\begin{document}

\begin{circuitikz}[american resistors, american voltages, american currents]
\coordinate(A) at (-1,3);
  \coordinate(B) at (-1,0);
  \coordinate(C) at (-6,0);
  \coordinate(D) at (-6,3);
  \draw
  (-6,0) to [esource, v= $10$ V] (-6,3)
  (-6,3) to [ccgsw] (-4,3)
  (-4,3) to [L, l=$0.2$ H, i=$i$] (-1,3)
  (-1,3) to [isource, l=$1$ A] (-1,0)
  (2,3) to [R, l=$10\Omega$, i=$i_R$] (2,0)
  (-1,3) to [short,-] (2,3)
  (2,0) to [short,-] (-1,0)
  (-6,0) to [cV, l_=$3\,i_R$] (-1,0);
\node[label=above:A] (A) at ($(A)$) {};
   \node[label=below:B] (B) at ($(B)$) {};
   \node[label=below:C] (C) at ($(C)$) {};
   \node[label=above:D] (D) at ($(D)$) {};
\end{circuitikz}

\end{document}