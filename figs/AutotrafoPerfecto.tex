\documentclass{standalone}
\usepackage{mathpazo}

\usepackage[americaninductors, americanvoltages]{circuitikzgit}
\usetikzlibrary{positioning, fit, calc}

\begin{document}

\begin{circuitikz}
  \draw
  (-1,3) node[] (A') {}
  (0,3) node[] (A) {}
  (-1,0) node[] (B') {}
  (0,0) node[] (B) {}
  (2,1.5) node[] (C') {}
  (0,1.5) node[] (C) {}
  (2,0) node[] (D') {}
  (0,0) node[] (D) {};
  \draw
  (A') to[short, *-, i=$I_1$] (A)
  (A) to[L, l_= $L_1$, name = L, i=$I_1 + I_2$] (B)
  (B') to[short, *-] (B)
  (C') to[short, *-, i =$I_2$] (C)
  (D') to[short, *-] (D);
  \draw
  (A') to[open, v = $V_1$] (B')
  (C') to[open, v^ = $V_2$] (D');
  \draw
  ([xshift = 10]L.north west) edge[bend left, <->, color=blue]
  node [above right] {$M = \sqrt{L' \cdot L_2}$}
  ([xshift = 10]L.north east);
  \draw
  ([xshift = -2]L.south west) node[circ] {}
  ([yshift = -3, xshift = 2]L.north) node[circ] {}
  ([xshift = 5]L.north west) node[] {$L'$}
  ([xshift = 5]L.north east) node[] {$L_2$};
\end{circuitikz}
\end{document}