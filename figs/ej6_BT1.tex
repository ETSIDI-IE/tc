\documentclass{standalone}
\usepackage{mathpazo}
\usepackage{tikz}
\usepackage{circuitikz}
\usetikzlibrary{calc}

\begin{document}

\begin{circuitikz}[american resistors, american voltages]
  \draw
  (0,4) to [esource, v_= $120\,\text{V}$] (0,2)
  (0,4) to [R, l= $0.2\Omega$] (0,6)
  (0,6) to [short, -, i = $20\,\text{A}$] (2,6)
  (2,6) to [R, l= $R_l$] (6,6)
  (6,6) to [R, l= $0.5\Omega$] (6,4)
  (6,4) to [esource, v= $fcem$] (6,2)
%   (6,6) to [R, l= $6\Omega$] (6,0)
%   (6,6) to [R, l= $6\Omega$, ] (9,6)
%   (9,6) to [short, -] (9,0)
  (6,2) to [short, -] (0,2);
%   \draw[thin, <-] (7.5,2)node{$I_3$}  ++(-60:0.5) arc (-60:170:0.5);
\end{circuitikz}

\end{document}

% \documentclass{standalone}
% \usepackage{mathpazo}
% \usepackage{circuitikz}
% \usetikzlibrary{calc}

% \begin{document}

% \begin{circuitikz}[european resistors, american voltages]
%   \coordinate (Ag) at (0,4);
%   \coordinate (Bg) at (0,0);
%   \node[label=above right:A] (A) at ($(Ag) + (2,0)$) {};
%   \node[label=below right:B] (B) at ($(Bg) + (2,0)$) {};
%   \draw
%   (Ag) to [R, l_= $Z_{th}$, i^>= $I$] (A)
%   (Ag) to [esource, v_= $\epsilon_{th}$] (Bg)
%   (Bg) to [short] (B)
%   (A) to [open, *-*, v^= $U_{AB}$] (B);
%   \draw
%   (A) to [short] ++(1.5,0)
%   to [R, l = $Z_L$] ($(B) + (1.5,0)$)
%   to [short] ++(-1.5,0)
%   to [short,*-] (B);
% \end{circuitikz}

% \end{document}