\documentclass{standalone}
\usepackage{mathpazo}
\usepackage[american voltages, european resistors]{circuitikz}
\usetikzlibrary{calc}

\begin{document}
\begin{circuitikz}
  \coordinate (A) at (90:4);
  \coordinate (B) at (210:4);
  \coordinate (C) at (-30:4);
  \coordinate (AB1) at ($(A)!0.3!(B)$);
  \coordinate (AB2) at ($(B)!0.3!(A)$);  
  \coordinate (BC1) at ($(B)!0.3!(C)$);
  \coordinate (BC2) at ($(C)!0.3!(B)$);  
  \coordinate (AC1) at ($(C)!0.3!(A)$);
  \coordinate (AC2) at ($(A)!0.3!(C)$);  
  \draw
  (A) node[above] {A} to [sV, *-, v_=$\overline{\epsilon}_A$] (AB1)
  (B) node[left] {B} to [sV, *-, v=$\overline{\epsilon}_B$] (AB2);
  \draw
  (C) node[right] {C} to [sV, *-, v_=$\overline{\epsilon}_C$] (AC1)
  (A) to [sV, *-, v=$\overline{\epsilon}_A$] (AC2);
  \draw
  (C) to [sV, *-, v=$\overline{\epsilon}_C$] (BC2)
  (B) to [sV, *-, v_=$\overline{\epsilon}_B$] (BC1);  
  \draw
  (AB1) to [R, l = $Z_{AB}$] (AB2)
  (BC1) to [R, l = $Z_{BC}$] (BC2)
  (AC1) to [R, l = $Z_{CA}$] (AC2);
\end{circuitikz}
\end{document}