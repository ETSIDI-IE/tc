\documentclass{standalone}

\usepackage{siunitx}

\usepackage{mathpazo}

\usepackage[american]{circuitikz}

\begin{document}

\begin{circuitikz}
  \coordinate (A) at (0, 0);
  \coordinate (B) at ($(A) + (3, 0)$) ;
  \coordinate (C) at ($(B) + (3, 0)$) ;
  \coordinate (D) at ($(C) + (3, 0)$) ;
  \coordinate (A') at ($(A) + (0, -4)$) ;
  \coordinate (B') at ($(A') + (3, 0)$) ;
  \coordinate (C') at ($(B') + (3, 0)$) ;
  \coordinate (D') at ($(C') + (3, 0)$) ;
  \draw
  (A) to [short, -*] (B) node[above] {A}
  to [R, l = $R_3$, i = $I_3$] (C) node[above] {B}
  to [short, *-] (D)
  (A') to [short, -*] (B')
  to [short, -*] (C')
  to [short, -] (D')
  (A) to [R, l = $R_1$, i = $I_1$] ++(0, -2)
  to [V, v = $\epsilon_1$] (A')
  (B) to [R, l = $R_2$, i = $I_2$] ++(0, -2)
  to [V, v = $\epsilon_2$] (B') node[below] {C}
  (C) to [R, l = $R_4$, i = $I_4$] (C') node[ground] {}
  (D') to [V, v= $\epsilon_3$] ++(0, 2)
  to [R, l = $R_5$, i<= $I_5$] (D);
\end{circuitikz}

\end{document}
