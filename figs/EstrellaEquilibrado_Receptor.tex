\documentclass{standalone}
\usepackage{mathpazo}
\usepackage[american voltages, european resistors]{circuitikz}

\begin{document}
\begin{circuitikz}
  \draw (0,0) coordinate (N) node[above right] {N'};
  \draw (-3,0) coordinate (Ng) node[above right] {N};
  \draw (90:2) coordinate (A) node[right] {A'};
  \draw (A) ++(-3,0) coordinate (Ag) node[left] {};
  \draw (-30:2) coordinate (B) node[right] {B'};
  \draw (B) ++(-4.5,0) coordinate (Bg) node[right] {};
  \draw (-150:2) coordinate (C) node[above left] {C'};
  \draw (C) ++(-1.5,0) coordinate (Cg) node[below right] {};
  \draw (Ag) ++(0,0.5) coordinate (Ag2);
  \draw (Bg) ++(0,-1) coordinate (Bg2);
  \draw (Cg) ++(0,-0.5) coordinate (Cg2);
  \draw
  (N) to [R, *-*,l=$\overline{Z}$] (A)
  (N) to [R, -*,l=$\overline{Z}$] (B)
  (N) to [R, -*,l=$\overline{Z}$] (C);
  \draw
  (Ag2) to [short, -, i=$\overline{I}_A$] (Ag2-|A) -- (A)
  (Bg2) to [short, -, i=$\overline{I}_B$] (Bg2-|B) -- (B)
  (Cg2) to [short, -, i_=$\overline{I}_C$] (Cg2-|C) -- (C);
  \draw
  (Ng) to [short, -, i=$\overline{I}_N$] (N);
\end{circuitikz}
\end{document}