\chapter{Corriente alterna monofásica}

\section{Enunciado}
En un circuito serie RL con $R=\qty{5}{\ohm}$ y $L=\qty{0.06}{\henry}$, la tensión en bornes de la bobina es $u_L(t)=15\sin(200\,t)\,\si{\volt}$. Determinar:
\begin{itemize}
\item La tensión total.
\item Intensidad de corriente.
\item Ángulo de desfase de la intensidad respecto de la tensión.
\item Impedancia del circuito.
\end{itemize}

\subsection*{Solución}

De la expresión temporal de $u_L(t)$ se tiene que $\omega=\qty{200}{\radian\per\second}$, por lo que:
\begin{equation*}
  \overline{X}_L=\mathrm{j}\,\omega L=\mathrm{j}\,200\cdot0.06=\mathrm{j}\,\qty{12}{\ohm}
\end{equation*}
siendo la impedancia del circuito:
\begin{equation*}
  \overline{Z}_{eq}=R+\overline{X}_L= \boxed{5+\mathrm{j}\,12 = 13\phase{67.3801^\circ}\,\si{\ohm}}
\end{equation*}
(es preferible usar cuatro decimales en los ángulos, para reducir los errores numéricos por aproximación).
\vspace{2mm}

\noindent El fasor correspondiente a $u_L(t)$ es:
\begin{equation*}
  \overline{U}_L=\dfrac{15}{\sqrt{2}}\phase{0^\circ}\,\si{\volt}
\end{equation*}
Por la ley de Ohm, la intensidad de corriente en la bobina (igual a la
total, al estar en serie):
\begin{equation*}
  \overline{I}=\dfrac{\overline{U}_L}{\overline{X}_L}=\dfrac{\frac{15}{\sqrt{2}}\phase{0^\circ}}{\mathrm{j}\,12}= \boxed{0.88\phase{-90^\circ}\,\si{\ampere}}
\end{equation*}
y la tensión total, por la 2LK:
\begin{equation*}
  \overline{U}=\overline{U}_R+\overline{U}_L=5\cdot (0.88\phase{-90^\circ})+\dfrac{15}{\sqrt{2}}\phase{0^\circ}= \boxed{11.48\phase{-22.5304^\circ}\,\si{\volt}}
\end{equation*}
siendo el ángulo de desfase de la intensidad respecto a la tensión:
\begin{equation*}
  \theta_I - \theta_U = -90 - (-22.53040) =  \boxed{-67.4696^\circ}
\end{equation*}
(la ligera diferencia en los decimales respecto al ángulo de $\overline{Z}_{eq}$ es debida a las aproximaciones en decimales en operaciones previas).

%%%%%%%%%%%%%%%%%%%%%%%%%%%%%%%%%%%%%%%%%%%%%%%%%%%%%%%%%%%%%%%%%%
\section{Enunciado}

Una resistencia de \qty{5}{\ohm} y un condensador se unen en serie. La tensión en la resistencia es $u_R(t) = 25 \cdot \sin(2000t + \pi/6)\,\si{\volt}$. Si la corriente está adelantada \ang{60} respecto de la tensión aplicada, ¿cuál es el valor de la capacidad C del condensador?

\subsection*{Solución}

El ángulo de la impedancia total es:
\begin{equation*}
    \theta = \theta_V - \theta_I \quad \rightarrow \quad \theta = -\ang{60} = -\pi/3\,\si{\radian}
\end{equation*}
Y la impedancia compleja: 
\begin{equation*}
    \overline{Z} = R - \mathrm{j} \frac{1}{\omega C}
  \end{equation*}
A partir del cual puede calcularse la capacidad del condensador: 
\begin{equation*}
  \tan \theta = - \frac{1}{\omega C R} \quad \rightarrow \quad \sqrt{3} = \frac{1}{10^4 C}
\end{equation*}

\begin{equation*}
  \boxed{C = \SI[parse-numbers = false]{100\sqrt{3}/3}{\micro\farad}}
\end{equation*}

%%%%%%%%%%%%%%%%%%%%%%%%%%%%%%%%%%%%%%%%%%%%%%%%%%%%%%%%%%%%%%%%%%
\section{Enunciado}
Para determinar las constantes R y L de una bobina, se conecta en serie con una resistencia de \qty{25}{\ohm} y al conjunto se le aplica una fuente de tensión de \qty{120}{\volt} a \qty{60}{\hertz}. Se miden las tensiones en bornes de la resistencia y de la bobina, obteniendo los valores $U_R = \qty{70.8}{\volt}$ y $U_B = \qty{86}{\volt}$. ¿Cuáles son las constantes de la bobina en cuestión?

\subsection*{Solución}

Por la 2LK, se debe cumplir que:
\begin{equation*}
  \overline{U} = \overline{U}_B + \overline{U}_R
\end{equation*}


La tensión en la resistencia de \qty{25}{\ohm}, por la ley de Ohm:
\begin{equation*}
  \overline{U}_R = 25 \cdot \overline{I} \quad \rightarrow \quad I = \frac{U_R}{25} = \qty{2.83}{\ampere}
\end{equation*}
Dado que
\begin{equation*}
  \overline{Z}_B = R_B + \mathrm{j}\,\omega L_B
\end{equation*}
y conocido el módulo de la corriente que circula por el circuito, obtenemos:
\begin{equation*}
  \overline{U}_B = \overline{I} \cdot \overline{Z}_B \quad \rightarrow \quad 86 = 2.83 \, Z_B \quad \rightarrow \quad Z_B = \qty{30.37}{\ohm}
\end{equation*}
La impedancia equivalente total del circuito es:
\begin{equation*}
  \overline{Z} = (25 + R_B) + \mathrm{j}\,\omega L_B
\end{equation*}
y por la ley de Ohm:
\begin{equation*}
  \overline{U} = \overline{I} \cdot \overline{Z} \quad \rightarrow \quad 120 = 2.83 \, Z \quad \rightarrow \quad Z = \qty{42.37}{\ohm}
\end{equation*}
Planteamos el sistema de ecuaciones resultantes:
\begin{align*}
  30.37 &= \sqrt{R^2_B + (\omega L_B)^2}\\
  42.37 &= \sqrt{(25 + R_B)^2 + (\omega L_B)^2}
\end{align*}
cuyas soluciones son:
\begin{align*}
  \Aboxed{R &= \qty{5}{\ohm}}\\
  \Aboxed{L &= \qty{79.5}{\milli\henry}}
\end{align*}

%%%%%%%%%%%%%%%%%%%%%%%%%%%%%%%%%%%%%%%%%%%%%%%%%%%%%%%%%%%%%%%%%%

\section{Enunciado}
Un circuito serie RLC con $R = \qty{5}{\ohm}$ , $L = \qty{0.02}{\henry}$ y $C=\qty{80}{\micro\farad}$, tiene aplicada una tensión senoidal de frecuencia variable. Determinar los valores de la pulsación $\omega$ para los cuales la corriente:
\begin{enumerate}
\item Adelanta \qty{45}{\degree} a la tensión.
\item Está en fase con ella.
\item Retrasa \qty{45}{\degree}.
\end{enumerate}

\subsection*{Solución}
La impedancia equivalente del sistema es:

\begin{equation*}
  \overline{Z} = R + \mathrm{j}\, \left(\omega L - \frac{1}{\omega C}\right)
\end{equation*}

La tangente del ángulo es:
\begin{equation*}
  \tan \theta = \frac{\omega L - 1/\omega C}{R} = \frac{0.02\,\omega^2 - 12.5\cdot 10^3}{5\,\omega}
\end{equation*}

En esta ecuación, planteamos las condiciones particulares del enunciado:

\begin{enumerate}
\item Adelanta \qty{45}{\degree} a la tensión:
  \begin{equation*}
    \theta = -\pi/4 \quad \rightarrow \quad \tan \theta = -1
  \end{equation*}

  \begin{equation*}
    \frac{0.02\,\omega^2 - 12.5\cdot 10^3}{5\,\omega} = -1
  \end{equation*}

  \begin{equation*}
    \omega^2 + 250\,\omega - 62.5\cdot 10^4 = 0 \quad \rightarrow \quad \boxed{\omega = \qty{675.4}{\radian\per\second}}
  \end{equation*}
  (se descarta la solución negativa de la ecuación de $2^\circ$ grado por carecer de sentido físico)

\vspace{3mm}
  
\item Está en fase con ella:
  \begin{equation*}
    \theta = 0 \quad \rightarrow \quad \tan \theta = 0
  \end{equation*}

  \begin{equation*}
    0.02\,\omega = \frac{12.5\cdot 10^3}{\omega}
  \end{equation*}

  \begin{equation*}
    \boxed{\omega = \qty{790.6}{\radian\per\second}}
  \end{equation*}
  
\item Retrasa \qty{45}{\degree}:
  \begin{equation*}
    \theta = +\pi/4 \quad \rightarrow \quad \tan \theta = +1
  \end{equation*}

  \begin{equation*}
    \frac{0.02\,\omega^2 - 12.5\cdot 10^3}{5\omega} = 1
  \end{equation*}

  \begin{equation*}
    \omega^2 - 250\,\omega - 62.5\cdot 10^4 = 0 \quad \rightarrow \quad \boxed{\omega = \qty{925.4}{\radian\per\second}}
  \end{equation*}

\end{enumerate}


%%%%%%%%%%%%%%%%%%%%%%%%%%%%%%%%%%%%%%%%%%%%%%%%%%%%%%%%%%%%%%%%%%

\section{Enunciado}

Determinar el triángulo de potencias de un circuito al que se le
aplica una tensión $u(t)=340 \cdot \cos(\omega t - \pi/3)$ V y
por el que circula una intensidad de corriente
$i(t)= 13.3 \cdot \cos(\omega t-0.85)\,\si{\ampere}$.

\subsection*{Solución}

Los fasores de dicha tensión y corriente son:
\begin{align*}
  \overline{U} &= 170 \sqrt{2}\phase{\ang{-60}}\,\si{\volt}\\
  \overline{I} &= 6.65\sqrt{2}\phase{\ang{-48.7}}\,\si{\ampere}
\end{align*}

Por definición, la potencia aparente es:
\begin{equation*}
  \overline{S} = \overline{U}\cdot\overline{I}^* = 2261\phase{\ang{-11.3}}\,\si{\voltampere}
\end{equation*}

que expresada en forma binómica, resulta en los valores de $P$ y $Q$:
\begin{align*}
  P &= S \cos \theta = \qty{2217.17}{\watt}\\
  Q &= S \sin \theta = \qty{-443.03}{\var}
\end{align*}

%%%%%%%%%%%%%%%%%%%%%%%%%%%%%%%%%%%%%%%%%%%%%%%%%%%%%%%%%%%%%%%%%%

\section{Enunciado}

En el esquema de la figura, los elementos tienen los siguientes valores:
\begin{align*}
  R_1 &= R_2 = R_3 = \qty{10}{\ohm}\\
  X_1 &= X_2 = \qty{1}{\ohm}\\
  R_L &= X_L = \qty{1}{\ohm}
\end{align*}

Sabiendo que $U_{CD} = \qty{200}{\volt}$, se debe calcular:
    \begin{itemize}
    \item Intensidades de corriente $I$, $I_1$, $I_2$ e $I_3$ {en forma
        fasorial}, tomando $U_{CD}$ como referencia de fase
    \item Lectura de los vatímetros $W_1$ y $W_2$
    \end{itemize}
    
\begin{center}
  \includegraphics[width=0.85\linewidth]{figuras/BT2_08.pdf}
\end{center}

 

\subsection*{Solución}

Se dice que se tome como referencia de fase el fasor
$\overline{U}_{CD}$:

  \begin{equation*}
    \overline{U}_{CD} = 200\phase{\ang{0}}\,\si{\volt}
  \end{equation*}

  
  Esta tensión está aplicada en tres ramas en paralelo, por lo que podemos calcular las corrientes en esas ramas. En primer lugar, calculamos las impedancias:

\begin{align*}
\overline{Z}_1 &= \qty[parse-numbers=false]{10 + \mathrm{j}}{\ohm}\\
%
\overline{Z}_2 &= \qty[parse-numbers=false]{10 - \mathrm{j}}{\ohm}
\end{align*}

A continuación calculamos las corrientes de rama y la corriente total:
\begin{align*}
\overline{I}_1 &= \frac{\overline{U}_{CD}}{\overline{Z}_1} = \qty[parse-numbers=false]{19.8 - 1.98\mathrm{j}}{\ampere}\\
%
\overline{I}_2 &= \frac{\overline{U}_{CD}}{\overline{Z}_2} = \qty[parse-numbers=false]{19.8 + 1.98\mathrm{j}}{\ampere}\\
%
\overline{I}_3 &= \frac{\overline{U}_{CD}}{R_3} = 20\phase{\ang{0}}\,\si{\ampere}\\
%
\overline{I} &= \overline{I}_1 + \overline{I}_2 + \overline{I}_3 =  59.6\phase{\ang{0}}\,\si{\ampere}\\
\end{align*}

Para obtener la lectura del vatímetro 2, podemos calcular con tensión y corriente:
  \begin{align*}
\overline{S}_2 &= \overline{U}_{CD} \cdot \overline{I}^* = 11920\phase{\ang{0}}\,\si{\voltampere}\\
%
W_2 &= \mathrm{Re}\{\overline{S}_2\} = \qty{11920}{\watt}
\end{align*}

O mediante el teorema de Boucherot:
\begin{align*}
  P_1 = I_1^2 \cdot R_1 &= \qty{3959.6}{\watt}\\
  P_2 = I_2^2 \cdot R_2 &= \qty{3959.6}{\watt}\\
  P_3 = I_3^2 \cdot R_3 &= \qty{4000}{\watt}\\
  W_2 = P = P_1 + P_2 + P_3 &= \qty{11919.2}{\watt}
\end{align*}

Para el vatímetro 1, hay que tener en cuenta la potencia disipada en la línea, y aplicar nuevamente el teorema de Boucherot:
\begin{align*}
P_l &= 2 \cdot I^2 \cdot R_L = \qty{7104.3}{\watt}\\
W_1 &= W_2 + P_l = \qty{19024.3}{\watt}
\end{align*}


%%%%%%%%%%%%%%%%%%%%%%%%%%%%%%%%%%%%%%%%%%%%%%%%%%%%%%%%%%%%%%%%%%

\section{Enunciado}
En el circuito de la figura, los amperímetros $A_1$ y $A_2$ marcan $\qty{4.5}{\ampere}$ y $\qty{6}{\ampere}$, respectivamente, el voltímetro, $\qty{150}{\volt}$, y el
vatímetro, $\qty{900}{\watt}$.

Sabiendo que la frecuencia del generador es de $\qty{250}{\hertz}$ y el f.d.p. de la impedancia $Z$ es de $0.8$ en retraso, calcula:

\begin{itemize}
\item Valores de R, C y Z en forma compleja.
\item La tensión del generador.
\item Triángulo de potencias totales.
\end{itemize}

\begin{center}
  \includegraphics[width=0.6\linewidth]{figuras/BT2_09.pdf}
\end{center}


\subsection*{Solución}
\begin{enumerate}
\item Valores de R, C y Z en forma compleja.
  
\begin{equation*}
    R = \frac{U_{BC}}{A_1} = \frac{150}{4.5} = \qty{33.3}{\ohm}
  \end{equation*}
  \begin{equation*}
    X_c = \frac{U_{BC}}{A_2} = \frac{150}{6} = \qty{25}{\ohm}
  \end{equation*}
  \begin{equation*}
    C = \frac{1}{X_c\omega} = \frac{1}{25 \cdot 2 \pi \cdot 250} = \qty{25.46}{\micro\farad}
  \end{equation*}

  Tomando $\overline{U}_{BC}$ como origen de fases, $\overline{U}_{BC} = \qty[parse-numbers=false]{150\phase{0}}{\volt}$, obtenemos:
  \begin{align*}
    \overline{I}_1 &= \qty[parse-numbers=false]{4.5\phase{0}}{\ampere}\\
    \overline{I}_2 &= \qty[parse-numbers=false]{6\phase{\pi/2}}{\ampere}
  \end{align*}

  Por tanto,
  \[
    \overline{I} = \overline{I}_1 + \overline{I}_2 = \qty[parse-numbers=false]{4.5+6j}{\ampere} = 7.5\phase{\ang{53.13}}\;\si{\ampere}
  \]
  
  El vatímetro está midiendo $P_Z = U_Z \cdot I \cdot \cos \theta_Z$, y por tanto:
  \[
    U_Z = \frac{900}{7.5 \cdot 0.8} = \qty{150}{\volt}
  \]
  \[
    Z = \frac{U_Z}{I} = \qty{20}{\ohm}
  \]

    También puede obtenerse este resultado calculando primero la parte resistiva de la impedancia:
  \[
    R_Z = \frac{P_Z}{I^2} = \qty{16}{\ohm}
  \]
  y a continuación el módulo, teniendo en cuenta que $R = Z \cdot \cos \theta$:
  \[
    Z = \frac{R}{\cos \theta} = \frac{16}{0.8} = \qty{20}{\ohm}
  \]

  Con su factor de potencia obtenemos el ángulo (teniendo en cuenta que es inductiva al ser en retraso), $\theta_Z = \arccos(0.8) = \ang{36.87}$:
  \[
    \overline{Z} = 16 + 12j  = 20\phase{\ang{36.87}}\;\si{\ohm}
  \]

  
\item Tensión del generador.

  \[
    \overline{U}_{AC} = \overline{U}_{AB} + \overline{U}_{BC}
  \]
  \[
    \overline{U}_{AB} = \overline{Z} \cdot \overline{I} = 150\phase{\ang{90}}\;\si{\volt}
  \]
  \[
    \overline{U}_{AC} = 150 + 150j = 150\sqrt{2}\phase{\ang{45}}\;\si{\volt}
  \]

\item Triángulo de potencias totales en forma compleja.

  Podemos calcular a partir de la tensión y la corriente:
  \begin{align*}
    \overline{S}_T &= \overline{U}_{AC} \overline{I}^* =\\
                   &= 150\sqrt{2}\phase{\ang{45}} \cdot 7.5\phase{\ang{-53.13}} =\\
                   &= 1591\phase{\ang{-8.13}}\;\si{\voltampere}=\\
                   &= \qty[parse-numbers=false]{1575 - j225}{\voltampere}
  \end{align*}
  
  o mediante el teorema de Boucherot:
  \begin{align*}
    P_Z &= \qty{900}{\watt}\\
    P_R = 4.5^2 \cdot 33.3 &= \qty{675}{\watt}\\
    P = P_Z + P_R &= \qty{1575}{\watt}
  \end{align*}

    \vspace{-4mm}
  \begin{align*}
    Q_Z = 7.5^2 \cdot 12 &= \qty{675}{\var}\\
    Q_c = - 6^2 \cdot 25 &= \qty{-900}{\var}\\
    Q = Q_Z + Q_c &= \qty{-225}{\var}
  \end{align*}
  
  Por tanto:
  \[
    \overline{S} = P + jQ = \qty[parse-numbers=false]{1575 - j225}{\voltampere}
  \]
  
\end{enumerate}

%%%%%%%%%%%%%%%%%%%%%%%%%%%%%%%%%%%%%%%%%%%%%%%%%%%%%%%%%%%%%%%%%%

\section{Enunciado}

En el circuito de la figura, determinar las lecturas de los aparatos de medida y el balance de potencias activas y reactivas, así como el triángulo global de potencias.\\
Datos: $\; e(t)=100\sqrt{2}\cos(\omega\,t)\,\si{\volt}$;\; $R_1=2\,\Omega$;\;
$R_2=4\,\Omega$;\; $\omega L_1=3\,\Omega$;\; $\omega L_2=4\,\Omega$.

\begin{center}
  \includegraphics{figuras/BT2_11.pdf}
\end{center}

\subsection*{Solución}

El voltímetro $V$ mide la tensión eficaz de la fuente, por lo que:
\begin{equation*}
  V=\qty{100}{\volt}
\end{equation*}

Las impedancias de las dos ramas son:
\begin{align*}
  \overline{Z}_1 &= R_1 + jX_1 = \qty[parse-numbers=false]{2 + j3}{\ohm}\\
  \overline{Z}_2 &= R_2 + jX_2 = \qty[parse-numbers=false]{4 + j4}{\ohm}
\end{align*}

Calculamos el valor eficaz de las corrientes de rama:
\begin{align*}
  I_1 &= \frac{E}{Z_1} = \qty{27.74}{\ampere}\\
  I_2 &= \frac{E}{Z_2} = \qty{17.68}{\ampere}\\
\end{align*}

El vatímetro $W_2$ mide la potencia de $R_2$:
\begin{equation*}
  W_2=R_2 \cdot I_2^2= \qty{1250.33}{\watt}
\end{equation*}

El vatímetro $W_1$ mide la potencia total del circuito:
\begin{equation*}
  W_1= P_{R2} + P_{R1} = R_2 \cdot I_2^2 + R_1 \cdot I_1^2 = \qty{2789.35}{\watt}
\end{equation*}

Por otra parte, las potencias reactivas de las bobinas son:
\begin{align*}
  Q_{L1} &= X_1 \cdot I_1^2 = \qty{2308.52}{\var}\\
  Q_{L2} &= X_2 \cdot I_2^2 = \qty{1250.33}{\var}
\end{align*}

Por tanto, la potencia reactiva total es $Q = \qty{3558.82}{\var}$. Con el valor de la potencia activa podemos obtener la potencia aparente total:
\begin{equation*}
  S = \sqrt{P^2 + Q^2} = \qty{4521.69}{\voltampere}
\end{equation*}

Y, finalmente, la corriente medida por el amperímetro:
\begin{equation*}
  I = \frac{S}{V} = \qty{45.2}{\ampere}
\end{equation*}

%%%%%%%%%%%%%%%%%%%%%%%%%%%%%%%%%%%%%%%%%%%%%%%%%%%%%%%%%%%%%%%%%%

\section{Enunciado}
El circuito de la figura tiene carácter inductivo.  La impedancia de
línea es $Z=\qty[parse-numbers=false]{10\sqrt{2}}{\ohm}$ con
f.d.p. $\sqrt{2}/2$ en retraso. Tómese como referencia de fases la
intensidad total, $I$.

\vspace{3mm}
Se debe calcular:
\begin{enumerate}

\item Potencia activa y reactiva consumida por $Z$.

\item Expresiones complejas de las intensidades medidas por los
  amperímetros, $I$, $I_1$, $I_2$ e $I_3$. 

\item Expresiones complejas de las tensiones $U_{AB}$, $U_{AC}$ y
  $U_{CB}$.

\item Valores de $R_1$, $X_1$, $R_2$, $R_3$ y $X_3$.

\end{enumerate}

Datos: $\;A = \qty[parse-numbers = false]{5\sqrt{5}}{\ampere}$;\; $A_1 = \qty[parse-numbers = false]{5\sqrt{2}}{\ampere}$;\; $A_2 = \qty{5}{\ampere}$;\;  $A_3 = \qty[parse-numbers = false]{\sqrt{10}}{\ampere}$;\;  $U_{AB} = \qty{247}{\volt}$;\;  $W_1 = \qty{2350}{\watt}$;\;
$R_1 = R_3$

\begin{center}
  \includegraphics[width=0.8\linewidth]{figuras/BT2_17.pdf}
\end{center}


\subsection*{Solución}

Dado que disponemos de la potencia y corriente total y la tensión a
la entrada, podemos calcular el factor de potencia del circuito:
\[
\cos \phi = \frac{P_1}{U_{AB} \cdot I} = 0.851
\]

Teniendo en cuenta que la corriente total es la referencia de fases, 
\[
\overline{U}_{AB} = {247 \phase{\ang{31.68}}}\,\si{\volt}
\]

También podemos calcular la potencia reactiva del circuito
(positiva dado que el circuito es inductivo):
\[
Q = P_1 \tan \phi = \qty{1450.2}{\var}
\]


En cuanto a la impedancia $Z$, sabemos que la tensión en sus bornes
es:
\[
\overline{U}_{AC} = \overline{I} \cdot \overline{Z} =
5\sqrt{5}\phase{\ang{0}} \cdot 10\sqrt{2}\phase{\ang{45}} = {50\sqrt{10}\phase{\ang{45}}}\,\si{\volt}
\]

Se cumple que $\overline{U}_{AB} = \overline{U}_{AC} +
\overline{U}_{CB}$, y por tanto:
\[
\overline{U}_{CB} = {100\phase{\ang{10.32}}}\,\si{\volt}
\]

Por otra parte, podemos descomponer esta impedancia en:
\begin{align*}
  R &= Z \cdot \cos \phi_Z = \qty{10}{\ohm}\\
  X &= Z \cdot \sin \phi_Z = \qty{10}{\ohm}
\end{align*}

y por tanto,
\begin{align*}
P_z &= I^2 \cdot R_z = \qty{1250}{\watt}\\
Q_z &= I^2 \cdot X_z = \qty{1250}{\var}
\end{align*}

Aplicando el teorema de Boucherot, podemos calcular la potencia activa
y la potencia reactiva del circuito paralelo:
\begin{align*}
P_{CB} &= P - P_z = \qty{1100}{\watt}\\
Q_{CB} &= Q - Q_z = \qty{200.2}{\var}\\
\overline{S}_{CB} &= P_{CB} + \mathrm{j}\, Q_{CB} = {1118.07\phase{\ang{10.32}}}\,\si{\voltampere}
\end{align*}

Podemos comprobar que estos resultados son coherentes con los
resultados anteriores, usando $\overline{S}_{CB} = \overline{U}_{CB}
\cdot \overline{I}^*$.

\vspace{2mm}
Ahora podemos obtener los módulos de $R_2$, $Z_1$ y $Z_3$:
\begin{align*}
  R_2 &= \frac{U_{CB}}{I_2} = \qty{20}{\ohm}\\
  Z_1 &= \frac{U_{CB}}{I_1} = \qty[parse-numbers=false]{10\sqrt{2}}{\ohm}\\
  Z_3 &= \frac{U_{CB}}{I_3} = \qty[parse-numbers=false]{10\sqrt{10}}{\ohm}
\end{align*}

Por otra parte, la potencia activa del circuito paralelo es:
\begin{align*}
  P_{CB} &= P_1 + P_2 + P_3 =\qty{1100}{\watt}\\
  P_1 &= I_1^2 \cdot R_1 = 50 \cdot R_1\\
  P_2 &= I_2^2 \cdot R_2 = \qty{500}{\watt}\\
  P_3 &= I_3^2 \cdot R_3 = 10 \cdot R_3
\end{align*}

Dado que sabemos que $R_1 = R_3$: 
\[
   P_1 + P_3 = (50+10) \cdot R_1 = \qty{600}{\watt} \quad \rightarrow \quad R_1 = R_3 = \qty{10}{\ohm}              
\]

Con este resultado, y teniendo en cuenta el módulo de $Z_1$ y $Z_3$,
podemos calcular las respectivas reactancias:
\begin{align*}
  R_1 &= \qty{10}{\ohm}\\
  X_1 &= \sqrt{(10\sqrt2)^2 - 10^2} = \qty{10}{\ohm}\\
  \overline{Z}_1 &= 10 + 10j = {10\sqrt{2}\phase{\ang{45}}}\,\si{\ohm}
\end{align*}
\begin{align*}
  R_3 &= \qty{10}{\ohm}\\
  X_3 &= \sqrt{(10\sqrt{10})^2 - 10^2} = \qty{30}{\ohm}\\
  \overline{Z}_3 &= 10 - 30j = {10\sqrt{10}\phase{\ang{-71.56}}}\,\si{\ohm}
\end{align*}

Podemos comprobar que estas soluciones concuerdan con la potencia
reactiva de cada impedancia y con la total del circuito paralelo:
\begin{align*}
  Q_1 &= I_1^2 \cdot X_1 = \qty{500}{\var}\\
  Q_3 &= - I_3^2 \cdot X_3 = \qty{-300}{\var}\\
  Q_{CB} &= Q_1 + Q_3 =\qty{200}{\var}
\end{align*}

Con estos resultados, recordando que  $\overline{U}_{CB} =
100\phase{\ang{10.32}}$ podemos calcular las corrientes de rama en forma
compleja:
\begin{align*}
  \overline{I}_1 &=  {5\sqrt{2}\phase{\ang{-34.68}}}\,\si{\ampere}\\
  \overline{I}_2 &=  {5\phase{\ang{10.32}}}\,\si{\ampere}\\
  \overline{I}_3 &=
  {\sqrt{10}\phase{\ang{81.89}}}\,\si{\ampere}
\end{align*}

Para terminar, podemos comprobar que $\overline{I} = \overline{I}_1
+ \overline{I}_2 + \overline{I}_3$.

%%%%%%%%%%%%%%%%%%%%%%%%%%%%%%%%%%%%%%%%%%%%%%%%%%%%%%%%%%%%%%%%%%

\section{Enunciado}

La potencia reactiva del circuito de la figura es $\qty{80}{\var}$ de tipo capacitivo. La tensión en la impedancia Z está en fase con la intensidad $I_1$ y las lecturas de los aparatos son $A = \qty{4}{\ampere}$, $V = \qty{50}{\volt}$, $W = \qty{200}{\watt}$. Sabiendo que $R_1 = \qty{10}{\ohm}$ y $X_2 = \qty{50}{\ohm}$, calcula:

\begin{enumerate}
\item Las corrientes $I_1$, $I_2$, $I_3$ en forma fasorial.
\item Las reactancias $X_1$, $X_3$, y la impedancia $\overline{Z}$.
\item La fuerza electromotriz $\overline{\epsilon}$.
\end{enumerate}
\begin{center}
  \includegraphics{figuras/BT2_circuitoCapacitivo}
\end{center}

\subsection*{Solución}


El vatímetro está midiendo la potencia activa del circuito paralelo conectado entre A y B. El único elemento que consume potencia activa en ese circuito es la resistencia $R_1$. Por tanto,

\[
  P_{R1} = 200 = I_1^2 R_1 \rightarrow I_1 = \qty[parse-numbers=false]{2\sqrt{5}}{\ampere}
\]

Dado que conocemos la tensión entre A y B, podemos determinar la impedancia de la rama 1:

\[
  Z_1 = \frac{V_{AB}}{I_1} = \qty[parse-numbers=false]{5\sqrt{5}}{\ohm}
\]
y, por tanto, obtenemos $X_1$:

\[
  Z_1 = \sqrt{R_1^2 + X_1^2} \rightarrow X_1 = \qty{5}{\ohm}
\]

\[
  \overline{Z}_1 = 10 + j5 = 5\sqrt{5}\phase{\ang{26.56}} \,\si{\ohm} 
\]

Para obtener las corrientes en forma fasorial necesitamos una referencia de fases, y será la tensión $U_{AB}$:

\[
  \overline{U}_{AB} = 50\phase{\ang{0}}\,\si{\volt}
\]

Además, del circuito AB conocemos la tensión, la corriente y la potencia, luego podemos obtener su factor de potencia:

\[
  \cos\theta_{AB} = \frac{P_{AB}}{I \cdot U_{AB}} = 1
\]

Por tanto,

\[
  \overline{I} = 4\phase{\ang{0}} \,\si{\ampere}
\]


Con $\overline{U}_{AB}$ podemos calcular el ángulo de la corriente $I_1$:

\[
  \overline{I}_1 = \frac{\overline{U}_{AB}}{\overline{Z}_1} = 2\sqrt{5}\phase{\ang{-26.56}} \,\si{\ampere} 
\]

De la misma forma podemos calcular la corriente $I_2$:
\[
  \overline{I}_2 = \frac{\overline{U}_{AB}}{jX_2} = 1\phase{\ang{-90}} \,\si{\ampere} 
\]
  
Mediante la LKC podemos obtener la corriente en la rama 3:

\[
  \overline{I} = \overline{I}_1 + \overline{I}_2 + \overline{I}_3 \rightarrow \overline{I}_3 = \qty[parse-numbers=false]{3\phase{\pi/2}}{\ampere}
\]

Aplicamos el teorema de Boucherot para obtener la reactancia de la rama 3, teniendo que en cuenta que $\cos(\theta_{AB}) = 1 \rightarrow Q_{AB} = 0$:

\begin{align*}
  Q &= Q_1 + Q_2 + Q_3 = 0\\
  Q_1 &= I_1^2 X_1 = \qty{100}{\var}\\
  Q_2 &= I_2^2 X_2 = \qty{50}{\var}\\
  Q_3 &= - I_3^2 X_3
\end{align*}

Por tanto,

\[
   Q_3 = -\qty{150}{\var} \rightarrow X_3 = \qty[parse-numbers=false]{\frac{50}{3}}{\ohm}
\]

Una forma alternativa de calcular $X_3$ es simplemente a partir del cociente entre tensión y corriente: 
\[
    \overline{Z}_3 = \frac{\overline{U}_{AB}}{\overline{I}_3} = j X_3
\]

Para determinar $\overline{Z}$ tenemos en cuenta que la potencia reactiva total es $\qty{80}{\var}$ de tipo capacitivo y que $Q_{AB} = \qty{0}{\var}$:

\[
  Q = Q_Z + Q_{AB} \rightarrow Q_Z = -\qty{80}{\var} 
\]

Por tanto:

\[
  X_Z = \frac{|Q_Z|}{I^2} = \qty{5}{\ohm}
\]

Por otra parte, el enunciado indica que la tensión en esta impedancia está en fase con la intensidad $I_1$. Por tanto, $\theta_{VZ} = \ang{-26.56}$, y $\theta_Z = \theta_{VZ} - \theta_{I} = \ang{-25.56}$. Con este ángulo podemos calcular el valor de la resistencia:

\[
  R_Z = \frac{X_Z}{|\tan\theta_Z|} = \qty{10}{\ohm}
\]

\[
  \overline{Z} =  \qty[parse-numbers=false]{10 - j5}{\ohm}
\]

Finalmente, para calcular la fuerza electromotriz podemos hacerlo de dos formas, mediante potencias o mediante tensiones:

Mediante el teorema de Boucherot calculamos la potencia activa:
\[
P = P_Z + P_{AB} = I^2 R_Z + 200 = \qty{360}{\watt}
\]
Y con la potencia reactiva $Q$ obtenemos la potencia aparente:

\[
  \overline{S} = P + jQ = \qty[parse-numbers=false]{360 -j80}{\voltampere}
\]
y la tensión:

\[
  \overline{\epsilon} = \frac{\overline{S}}{\overline{I}^*} = 90 - j20 = 10\sqrt{85}\phase{\ang{-12.53}} \,\si{\volt}
\]

Podemos llegar a este mismo resultado con un balance de tensiones:

\[
  \overline{\epsilon} = \overline{U}_Z + \overline{U}_{AB} = \overline{Z} \cdot \overline{I} + \overline{V}_{AB}
\]
%%%%%%%%%%%%%%%%%%%%%%%%%%%%%%%%%%%%%%%%%%%%%%%%%%%%%%%%%%%%%%%%%%
\section{Enunciado}
Un motor monofásico de $S = \qty{10}{\kilo\voltampere}$ y $fdp = 0.8$ está alimentado por una fuente de $\qty{230}{\volt}$ a $f = \qty{50}{\hertz}$. 
Calcula:
\begin{enumerate}
\item El valor eficaz de la corriente absorbida por el motor.
\item La potencia aparente del generador.
\item La capacidad del condensador necesario para compensar el factor de potencia a la unidad.
\item El valor eficaz de la corriente absorbida por el conjunto condensador-motor. 
\item La potencia aparente del generador necesario una vez conectado el condensador del tercer apartado.
\item Compara de forma razonada los resultados de los apartados 4 y 5 con los valores calculados en los apartados 1 y 2.
\end{enumerate}
\begin{enumerate}

\subsection*{Solución}

\item El valor eficaz de la corriente absorbida por el motor.


  \begin{align*}
  \overline{S}_m &= \overline{U} \cdot \overline{I}^*\\
%
  I &= \frac{\num{10000}}{230} = \qty{43.5}{\ampere}
\end{align*}

\item La potencia aparente del generador.
Suponemos línea ideal (sin pérdidas):
\[
  S_g = S_m = \qty{10}{\kilo\voltampere} 
\]

\item La capacidad del condensador necesario para compensar el factor de potencia a la unidad.
\begin{align*}
Q_m &= S \cdot \sin(\theta_m) = \qty{6}{\kilo\var}\\
Q_c &= Q_m\\
C &= \frac{Q_m}{\omega \cdot V^2} = \qty{361}{\micro\farad}
\end{align*}

\item El valor eficaz de la corriente absorbida por el conjunto condensador-motor. 
\begin{align*}
Q' &= \qty{0}{\var}\\
S' &= P_m = \qty{8}{\kilo\voltampere}\\
I' &= \frac{S'}{V} = \qty{34.8}{\ampere}
\end{align*}

\item La potencia aparente del generador necesario una vez conectado el condensador del tercer apartado.
\begin{align*}
S'_g &= S' = \qty{8}{\kilo\voltampere}\\
\end{align*}

\item Compara de forma razonada los resultados de los apartados 4 y 5 con los valores calculados en los apartados 1 y 2.

  La compensación de reactiva mediante la inserción del condensador ha reducido la corriente que circula por la línea y la potencia del generador en un 20\%.

\end{enumerate}

%%%%%%%%%%%%%%%%%%%%%%%%%%%%%%%%%%%%%%%%%%%%%%%%%%%%%%%%%%%%%%%%%% 

\section{Enunciado}
                                              
Un generador de corriente alterna monofásica ($f=\qty{50}{\hertz}$) alimenta a dos cargas a través de una línea de cobre. Esta línea, de resistividad $\rho=\qty{21}{\milli\ohm\milli\meter\squared\per\meter}$, tiene una longitud de $\qty{100}{\meter}$ y una sección de $\qty{16}{\milli\meter\squared}$. Las dos cargas, cuya tensión de alimentación es de $\qty{230}{\volt}$, son dos motores, uno con potencia de $\qty{7}{\kilo\watt}$ y f.d.p. de $0.65$, y otro con una potencia de $\qty{5}{\kilo\watt}$ y f.d.p. de $0.85$. Con esta información, se pide calcular:
\begin{itemize}
    \item Triángulo de potencias de cada carga y del conjunto de ambas.
    \item Valor eficaz de las corrientes en cada carga y de la corriente   total.
    \item Triángulo de potencias del generador.
    \item Valor eficaz de la tensión en bornes del generador.
    \item Capacidad del condensador a instalar en bornes de las cargas para mejorar el factor de potencia a $0.95$.
    \item Valor eficaz de la corriente entregada por el generador una vez instalado el condensador.
    \item Triángulo de potencias del generador una vez instalado el condensador.
\end{itemize}

\subsection*{Solución}

Las potencias del motor 1 son:
\begin{align*}
  P_1&=\qty{7000}{\watt}\\ 
  Q_1&=P_1\,\tan(\phi_1)=7000\cdot
                    \tan(\arccos(0.65))=\qty{8183.91}{\var}\\
  S_1&=\dfrac{P_1}{\cos(\phi_1)}=\dfrac{7000}{0.65}=\qty{10769.23}{\voltampere}
\end{align*}

y las del motor 2:
\begin{align*}
  P_2&=\qty{5000}{\watt}\\ 
  Q_2&=P_2\,\tan(\phi_2)=5000\cdot
                    \tan(\arccos(0.85))=\qty{3098.72}{\var}\\
  S_2&=\dfrac{P_2}{\cos(\phi_2)}=\dfrac{5000}{0.85}=\qty{5882.53}{\voltampere}
\end{align*}

Por el teorema de Boucherot, la potencia total de las cargas es:
\begin{align*}
  P_T&=P_1+P_2 = \qty{12000}{\watt}\\
  Q_T&=Q_1+Q_2 = \qty{11282.63}{\var}\\
  S_T&=\sqrt{P_T^2+Q_T^2}=\qty{16471.12}{\voltampere}
\end{align*}

por lo que la instalación conjunta tiene un f.d.p. de:
\begin{equation*}
  \mbox{f.d.p.}_{total}=\dfrac{P_T}{S_T}=\dfrac{12000}{16471.12}=0.7285
\end{equation*}

Usando la definición de potencia activa, se obtienen los valores eficaces de las corrientes:
\begin{align*}
  I_1&=\dfrac{P_1}{U\,\cos(\phi_1)}=\dfrac{7000}{230\cdot
       0.65}=\qty{46.82}{\ampere}\\
  I_2&=\dfrac{P_2}{U\,\cos(\phi_2)}=\dfrac{5000}{230\cdot
       0.85}=\qty{25.58}{\ampere}\\
  I_T&=\dfrac{P_T}{U\,\cos(\phi_T)}=\dfrac{12000}{230\cdot
       0.7285}=\qty{71.62}{\ampere}
\end{align*}

La resistencia de cada conductor de la línea es:
\begin{equation*}
  R_l=\rho\,\dfrac{l}{S}=21\cdot 10^{-3}\cdot
  \dfrac{100}{16}=\qty{0.13}{\ohm}
\end{equation*}

Así, las pérdidas en la línea son:
\begin{equation*}
  P_l= 2 \cdot R_l \cdot I^2=\qty{1346.2}{\watt}
\end{equation*}

y el triángulo de potencias del generador, por el teorema de Boucherot:
\begin{align*}
  P_g&=P_l+P_T= \qty{13346.23}{\watt}\\ Q_g&=Q_T=\qty{11282.63}{\var}\\
  S_g&=\sqrt{P_g^2+Q_g^2}=\qty{17476.26}{\voltampere}
\end{align*} 
por lo que la tensión a la salida del generador es:
\begin{equation*}
  U_g=\dfrac{S_g}{I}=\qty{244.4}{\volt}
\end{equation*}

Para mejorar el factor de potencia, se sabe que la potencia reactiva inicial es $\qty{11282.63}{\var}$. Puesto que se quiere un f.d.p.' de $0.95$, la potencia reactiva final será:
\begin{equation*}
  Q_T'=P_T\,\tan(\phi')=12000\cdot \tan(\arccos(0.95))=\qty{3944.21}{\var}
\end{equation*} 
siendo la potencia reactiva restante la generada por la batería de condensadores ($Q_C=Q_T'-Q_T=3944.21-11282.63=\qty{-7338.42}{\var}$). Por tanto, la capacidad del condensador equivalente a instalar es:
\begin{equation*}
  Q_c=X_c\,I^2=\dfrac{U^2}{X_c} \quad \Rightarrow \quad
  C=\dfrac{Q}{\omega\,U^2}=\dfrac{7338.42}{2\cdot\pi\cdot 50\cdot
    230^2}=\qty{441.57}{\micro\farad}
\end{equation*} 
A este mismo resultado se llegaría a partir de la expresión:
\begin{equation*}
  C=\frac{P_T \left[\tan (\phi) - \tan (\phi')\right]}{\omega
    U^2}=\dfrac{12000\left[\tan (\arccos(0.7285)) - \tan
      (\arccos(0.95))\right]}{2\cdot\pi\cdot 50\cdot 230^2}=\qty{441.66}{\micro\farad}
\end{equation*}

Una vez instalado el condensador, la potencia aparente es:
\begin{equation*}
  S_T'=\sqrt{P_T^2+Q_T'^2}=\sqrt{12000^2+3944.21^2}=\qty{12631.58}{\voltampere}
\end{equation*} 
siendo la corriente total en las cargas (entregada por el generador):
\begin{equation*}
  I'=\dfrac{S'}{U}=\dfrac{12631.58}{230}=\qty{54.92}{\ampere}
\end{equation*} 
Con esta corriente, las pérdidas en la línea se reducen a:
\begin{equation*}
  P_l'=2 \cdot R_l \cdot I'^2= \qty{791.76}{\watt}
\end{equation*} 
y el triángulo de potencias del generador, por el teorema de Boucherot:
\begin{align*}
  P_g'&=P_l'+P_T=\qty{12791.75}{\watt}\\ Q_g'&=Q_T'=\qty{3944.21}{\var}\\
  S_g'&=\sqrt{P_g'^2+Q_g'^2}=\qty{13386.02}{\voltampere}
\end{align*}

%%%%%%%%%%%%%%%%%%%%%%%%%%%%%%%%%%%%%%%%%%%%%%%%%%%%%%%%%%%%%%%%%%
\section{Enunciado}
Un generador de corriente alterna monofásica ($f = \qty{50}{\hertz}$) alimenta
a dos cargas a través de una línea de cobre. Esta línea, de
resistividad $\rho = \qty{0.017}{\ohm\per\milli\meter\squared\per\meter}$, tiene una longitud de
\qty{40}{\meter} y una sección de \qty{6}{\milli\meter\squared}. Las dos cargas, cuya tensión de
alimentación es de \qty{200}{\volt}, son:
\begin{enumerate}
\item Un motor de \qty{7}{\kilo\watt} con f.d.p. {0,7}.
\item Un grupo de lámparas fluorescentes con potencia total \qty{200}{\watt} y
  f.d.p. {0,5}.
\end{enumerate}
Se pide:
\begin{itemize}
\item Esquema del circuito señalando adecuadamente los elementos,
  corrientes y tensiones
\item Potencias activa, reactiva y aparente de cada carga
\item Valor eficaz de las corrientes en cada carga, y de la corriente
  total
\item Potencia activa y reactiva entregada por el generador
\item Valor eficaz de la tensión en bornes del generador
\item Capacidad necesaria a instalar en bornes de las cargas para
  mejorar el factor de potencia de las mismas a la unidad
\item Valor eficaz de la tensión en bornes del generador, y potencia
  aparente entregada por el mismo una vez instalada la capacidad
  determinada en el apartado anterior
\end{itemize}

\subsection*{Solución}

\begin{enumerate}
  
\item El esquema del circuito es el mostrado en la figura.

\begin{center}
  \includegraphics{figuras/circuito_cargas.pdf}
\end{center}
  
\item Potencias activa, reactiva y aparente de cada carga.

\begin{align*}
  P_M &= \qty{7000}{\watt}\\
  Q_M &= \qty{7141.4}{\var}\\
  S_M &= \qty{10000}{VA}\\
  P_F &= \qty{200}{\watt}\\
  Q_F &= \qty{346.4}{\var}\\
  S_F &= \qty{400}{VA}
\end{align*}

\item Valor eficaz de las corrientes en cada carga, y de la corriente
  total.

  \begin{align*}
    I_M &= S_M / V = \qty{50}{\ampere}\\
    I_F &= S_F / V = \qty{2}{\ampere}
  \end{align*}

  Por teorema de Boucherot la potencia total en cargas es:

\begin{align*}
  P_T &= \qty{7200}{\watt}\\
  Q_T &= \qty{7487.8}{\var}\\
  S_T &= \qty{10387.9}{VA}
\end{align*}

Y, por tanto, la corriente total es:
\[
  I = S_T / U = \qty{51.9}{\ampere}
\]

\item Potencia activa y reactiva entregada por el generador.
La resistencia de la línea (una resistencia por cada conductor) es:

\[
R_L = \rho L/S = \qty{0.113}{\ohm}
\]

La potencia activa disipada en la línea es:

\[
P_L = 2 \cdot I^2 R_L = \qty{611.48}{\watt}
\]

Por tanto, la potencia entregada por el generador es:

\begin{align*}
P_g &= P_L + P_T = \qty{7811.5}{\watt}\\
Q_g &= Q_T = \qty{7487.8}{\var}\\
S_g &= \qty{10820.7}{VA}
\end{align*}

\item Valor eficaz de la tensión en bornes del generador.
\[
U_g = S_g / I = \qty{208.3}{\volt}
\]

\item Capacidad necesaria a instalar en bornes de las cargas para
  mejorar el factor de potencia de las mismas a la unidad.

  \[
C = \frac{Q_t}{\omega V^2} = \qty{595.9}{\micro\farad}
\]


\item Valor eficaz de la tensión en bornes del generador, y potencia
  aparente entregada por el mismo una vez instalada la capacidad
  determinada en el apartado anterior.

  Una vez instalado este condensador, la corriente total en las cargas es:

\[
I' = P_T / V = \qty{36}{\ampere}
\]

La potencia disipada en la línea es ahora:

\[
P'_L = 2 \cdot I'^2 R_L = \qty{293.8}{\watt} 
\]

Y la potencia entregada por el generador es:
\begin{align*}
P'_g &= \qty{7493.8}{\watt}\\
Q'_g &= \qty{0}{\var}\\
S'_g &= \qty{7493.8}{VA}
\end{align*}

Por tanto, la tensión en bornes del generador es:

\[
U'_g = S'_g / I' = \qty{208.2}{\volt}
\]

\end{enumerate}
%%%%%%%%%%%%%%%%%%%%%%%%%%%%%%%%%%%%%%%%%%%%%%%%%%%%%%%%%%%%%%%%%% 

\section{Enunciado}


Un generador de corriente alterna ($f = \SI{50}{\hertz}$) alimenta una instalación eléctrica a través de una línea de cobre ($\rho = \SI{0.017}{\ohm\milli\meter\squared\per\meter}$) de $\SI{25}{\milli\meter\squared}$ de sección. La instalación eléctrica está compuesta por un motor de $S_m = \SI{10}{\kilo\voltampere}$ y $\mathrm{fdp} = 0.8$, una instalación de alumbrado fluorescente de $P_f = \SI{800}{\watt}$ y $\mathrm{fdp} = 0.9$, y diversas cargas electrónicas con una potencia conjunta $P_e = \SI{540}{\watt}$ y $\mathrm{fdp} = 0.5$ en retraso.

Suponiendo que las cargas trabajan a su tensión nominal de $\SI{230}{\volt}$ y que están situadas a $\SI{100}{\meter}$ del generador, calcule:

\begin{enumerate}
\item Triángulo de potencias total de las cargas ($P_T$, $Q_T$, $S_T$) y factor de potencia.
\item Valor eficaz de la corriente que circula por la línea.
\item Potencia disipada en la línea.
\item Triángulo de potencias del generador ($P_g$, $Q_g$, $S_g$) y factor de potencia.
\item Valor eficaz de la tensión de salida del generador.
\item Capacidad del banco de condensadores a instalar en bornes de la carga necesario para reducir la corriente que circula por la línea a un valor de $\SI{45}{\ampere}$.
\end{enumerate}

Independientemente del resultado obtenido, suponga que la capacidad instalada es $C = \SI{172}{\micro\farad}$. En estas condiciones, calcule:
\begin{enumerate}[resume]
\item Potencia aparente de las cargas (incluyendo al banco de condensadores)
\item Valor eficaz de la corriente que circula por la línea y potencia disipada en la misma.
\item Triángulo de potencias del generador y factor de potencia.
\item Tensión de trabajo del generador.
\end{enumerate}

\subsection*{Solución}

\begin{enumerate}
\item Triángulo de potencias total de las cargas ($P_T$, $Q_T$, $S_T$) y factor de potencia.

    Motor:
  \begin{align*}
    P_m &= \SI{8000}{\watt}\\
    Q_m &= \SI{6000}{\var}\\
  \end{align*}

  Alumbrado
  \begin{align*}
    P_f &= \SI{800}{\watt}\\
    Q_f &= \SI{387.5}{\var}\\
  \end{align*}

  Cargas Electrónicas
  \begin{align*}
    P_e &= \SI{540}{\watt}\\
    Q_e &= \SI{935.3}{\var}\\
  \end{align*}

  Total (Teorema de Boucherot)
  \begin{align*}
    P_T &= P_m + P_f + P_e = \SI{9340}{\watt}\\
    Q_T &= Q_m + Q_f + Q_e = \SI{7322.8}{\var}\\
  \end{align*}

  Por tanto, $S_T = \SI{11868.4}{\voltampere}$ y
  $\mathrm{fdp}_T = 0.787$.

\item Valor eficaz de la corriente que circula por la línea.
\[
  I = \frac{S_T}{U} = \frac{11868.4}{230} = \SI{51.6}{\ampere}
\]

\item Potencia disipada en la línea.

  \begin{align*}
  R &= \SI{0.068}{\ohm}\\
  P_L &= 2 \cdot I^2 \cdot R = \SI{362.1}{\watt}
  \end{align*}

\item Triángulo de potencias del generador ($P_g$, $Q_g$, $S_g$) y factor de potencia.


  \begin{align*}
    P_g &= P_T + P_L = \SI{9702.1}{\watt}\\
    Q_g &= Q_T = \SI{7322.8}{\var}\\
    S_g &= \SI{12155.4}{\voltampere}\\    
    \mathrm{fdp} &= 0.798
  \end{align*}

\item Valor eficaz de la tensión de salida del generador.

  \[
    U_g = \frac{S_g}{I} = \SI{235.6}{\volt}
  \]

\item Capacidad del banco de condensadores a instalar en bornes de la carga necesario para reducir la corriente que circula por la línea a un valor de $\SI{45}{\ampere}$.

  Si la corriente en línea se reduce a $\SI{45}{\ampere}$ la potencia aparente resultante en cargas (incluyendo al condensador) es $S'_T = 230 \cdot 45 = \SI{10350}{\voltampere}$. Por tanto, $Q'_T = \SI{4459.5}{\var}$. Así, es necesario instalar un banco de condensadores que aporte $Q_c = Q_T - Q'_T = \SI{2863.3}{\var}$.

\[
C = \frac{Q_c}{\omega U^2} = \SI{172.3}{\micro\farad}
\]

  
\end{enumerate}

Independientemente del resultado obtenido, suponga que la capacidad instalada es $C = \SI{172}{\micro\farad}$. En estas condiciones, calcule:
\begin{enumerate}[resume]
\item Potencia aparente de las cargas (incluyendo al banco de condensadores)
\[
S'_T = \sqrt{P^2_T + Q'^2_T} = \SI{10350.1}{\voltampere}
\]

\item Valor eficaz de la corriente que circula por la línea y potencia disipada en la misma.
\[
I' = \frac{S'_T}{U} = \SI{45}{\ampere}
\]

\[
  P'_L = 2 \cdot I'^2 \cdot R = \SI{275.4}{\watt}
\]

\item Triángulo de potencias del generador y factor de potencia.
  \begin{align*}
    P'_g &= P_T + P'_L = \SI{9615.4}{\watt}\\
    Q'_g &= Q'_T = \SI{4459.5}{\var}\\
    S'_g &= \SI{10599.2}{\voltampere}\\    
  \end{align*}

\item Tensión de trabajo del generador.
\[
U'_g = \frac{S'_g}{I'} = \SI{235.5}{\volt}
\]
\end{enumerate}

%%%%%%%%%%%%%%%%%%%%%%%%%%%%%%%%%%%%%%%%%%%%%%%%%%%%%%%%%%%%%%%%%% 
\section{Enunciado}

Calcular la corriente $i(t)$ del circuito de la figura.

\begin{center}
  \includegraphics{figuras/BT2_13.pdf}
\end{center}
Datos: $i_g(t) = 10\sqrt{2}\sin(100t)\unit{\ampere}$; $R_1 = R_2 = \qty{1}{\ohm}$; $L_1 = L_2 = \qty{0.01}{\henry}$; $C_1 = \qty{0.01}{\farad}$; $u_g(t) = 10\sqrt{2}\cos(100t)\unit{\volt}$
\subsection*{Solución}

En primer lugar, se deben indicar las dos expresiones de tensión y
corriente en una misma función senoidal. En este caso, se opta por
pasar la corriente a función coseno:
\begin{align*}
  u(t)=\sqrt{2}\,10\,\cos(100\,t) \;\si{\volt}  &\quad\Rightarrow\quad \overline{U}=10\phase{\ang{0}} V\\
  i_g(t)=\sqrt{2}\,10\,\cos(100\,t-\frac{\pi}{2}) \;\si{\ampere} &\quad\Rightarrow\quad \overline{I}_g=10\phase{\ang{-90}} A
\end{align*}
Se transforma la fuente de corriente en una fuente de tensión en serie
con la resistencia $R_1$:
\begin{equation*}
  \overline{U}_{I_g}=\overline{I}_g \cdot R_{1}=(10\phase{\ang{-90}})\cdot 1=10\phase{\ang{-90}} V
\end{equation*}
y estableciendo corrientes de malla como se muestra en la siguiente
figura, se puede plantear el sistema de ecuaciones en forma matricial
tras determinar el valor de las impedancias:
\begin{align*}
  \overline{X}_L&=\mathrm{j}\,\omega\,L=\mathrm{j}\, 100\cdot 0.01= \mathrm{j}\,\Omega\\
  \overline{X}_C&=-\mathrm{j}\,\dfrac{1}{\omega\,C}=-\mathrm{j}\, \dfrac{1}{100\cdot 0.01}= -\mathrm{j}\,\Omega
\end{align*}

\begin{center}
  \includegraphics{figuras/BT2_13_mod.pdf}
\end{center}

\begin{equation*}
  \begin{bmatrix}
    1+\mathrm{j}-\mathrm{j} & -(-\mathrm{j})\\
    -(-\mathrm{j}) & 1+\mathrm{j}-\mathrm{j}
  \end{bmatrix}
  \cdot
  \begin{bmatrix}
    \overline{I_a}\\
    \overline{I_b}
  \end{bmatrix}
  =
  \begin{bmatrix}
    10\phase{-90^\circ}\\
    -10\phase{0^\circ}
  \end{bmatrix}
\end{equation*}
cuya solución es:
\begin{align*}
  \overline{I}_a&=0\;\si{\ampere}\\
  \overline{I}_b&=-10\;\si{\ampere}
\end{align*}
Dado que la corriente $i(t)$ se relaciona con las corrientes de malla
por:
\begin{equation*}
  \overline{I}=\overline{I}_a-\overline{I}_b=0-(-10)=10\;\si{\ampere}
\end{equation*}
siendo su expresión temporal:
\begin{equation*}
  i(t)=\sqrt{2}\,10\,\cos(100\,t)\;\si{\ampere}
\end{equation*}

%%%%%%%%%%%%%%%%%%%%%%%%%%%%%%%%%%%%%%%%%%%%%%%%%%%%%%%%%%%%%%%%%%


%%% Local Variables:
%%% mode: latex
%%% TeX-master: "Problemas_TC"
%%% ispell-local-dictionary: "castellano"
%%% End:

