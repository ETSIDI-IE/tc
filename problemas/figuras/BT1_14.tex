
\documentclass{standalone}
\usepackage{mathpazo}
\usepackage{circuitikz}
\usetikzlibrary{calc}

\begin{document}

\begin{circuitikz}[american resistors, american voltages, american currents]
\coordinate(A) at (2,3);
\coordinate(B) at (5,3);
  \draw
  (0,0) to [R, l=$R_1$] (0,3)
  (2,0) to [isource, l=$I_1$,*-*] (2,3)
  (0,3) to [short,-] (3,3) 
  (5,3) to [cI, l=$2.5\,I_x$] (2,3)
  (2,4.5) to [R, l=$R_2$] (5,4.5)
  (2,3) to [short,-] (2,4.5)
  (5,3) to [short,-] (5,4.5)
  (5,3) to [R, l=$R_3$, i = $I_x$,*-*] (5,0) 
  %(6,3) to [short,-] (6,5)
  (5,3) to [short,-] (7,3)
  (7,0) to [isource,l=$I_2$] (7,3)
  (7,0) to [short,-] (0,0)
  (3.5,0) to [short, -*] (3.5,0) node[ground] {};
  %\draw[thin, <-] (1.5,1.5)node{$I_a$}  ++(-60:0.5) arc (-60:170:0.5);
  %\draw[thin, <-] (5,1.5)node{$I_b$}  ++(-60:0.5) arc (-60:170:0.5);
  \node[label=above left:A] (A) at ($(A)$) {};
  \node[label=above right:B] (B) at ($(B)$) {};
\end{circuitikz}

\end{document}
