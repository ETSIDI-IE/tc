\documentclass{standalone}
\usepackage{mathpazo}
\usepackage{tikz}
\usepackage{circuitikz}
\usetikzlibrary{calc}

\begin{document}

\begin{circuitikz}[american resistors, american voltages, american currents]
\coordinate(A) at (8,6);
  \coordinate(B) at (8,0);
  \coordinate(C) at (8,2.5);
  \draw
  (0,0) to [isource, l= $2$ mA] (0,3)
  (0,3) to [short,-] (0,6)
  (0,3) to [R, l=$2k\Omega$] (3,3)
  (3,3) to [R, l=$2k\Omega$] (3,0)
  (3,3) to [R,l=$4k\Omega$] (3,6)
  (6,0) to [R, l= $6k\Omega$] (6,6)
  (0,0) to [short,-] (6,0)
  (0,6) to [short,-] (6,6)
  (6,6) to [short,-*] (8,6)
  (6,0) to [short,-*] (8,0);
  \node[label=above:$+$] (A) at ($(A)$) {};
   \node[label=below:$-$] (B) at ($(B)$) {};
   \node[label=$U_{0,2}$] (C) at ($(C)$) {};
   \draw[thin, <-] (1.5,4.5)node{$I_a''$}  ++(-60:0.5) arc (-60:170:0.5);
   \draw[thin, <-] (4.5,3)node{$I_b''$}  ++(-60:0.5) arc (-60:170:0.5);
  \draw[thin, <-] (1.5,1.5)node{$I_c''$}  ++(-60:0.5) arc (-60:170:0.5);
\end{circuitikz}

\end{document}