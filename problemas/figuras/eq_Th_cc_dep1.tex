
\documentclass{standalone}
\usepackage{mathpazo}
\usepackage{circuitikz}
\usetikzlibrary{calc}

\begin{document}

\begin{circuitikz}[american resistors, american voltages, american currents]
  \coordinate(A) at (10,4);
  \coordinate(B) at (10,1);
  \coordinate(C) at (6,4);
  \coordinate(D) at (14,1);
  \coordinate(E) at (14,2);
  \draw
  (3,4) to [cV, v_=$2\,I_x$, i^>=$I_1$] (3,1)
  (3,4) to [R, l=$1\Omega$] (6,4) 
  (6,4) to [R, l=$2\Omega$, i=$I_2$] (6,1)
  (6,4) to [esource, v=$12$ V] (9,4) 
  (9,4) to [R, l=$2\Omega$, i=$I_x$] (9,1)
 % (9,4) to [R, l=$1k\Omega$] (12,4)
 % (12,4) to [R, l=$1k\Omega$] (12,1)
  %(6,4) to [short,-*] (7,4)
  %(6,1) to [short,-*] (7,1)
  (3,1) to [short,-*] (10,1)
  (9,4) to [short,-*] (10,4);
  \node[label=above:A] (A) at ($(A)$) {};
   \node[label=below:B] (B) at ($(B)$) {};
   \node[label=above:C] (C) at ($(C)$) {};
  %\draw[thin, <-] (4.5,2.5)node{$I_a$}  ++(-60:0.5) arc (-60:170:0.5);
 % \draw[thin, <-] (7.5,2.5)node{$I_b$}  ++(-60:0.5) arc (-60:170:0.5);
\end{circuitikz}

\end{document}
