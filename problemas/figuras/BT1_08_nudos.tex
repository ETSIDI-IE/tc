\documentclass{standalone}
\usepackage{mathpazo}
\usepackage{tikz}
\usepackage{circuitikz}
\usetikzlibrary{calc}

\begin{document}

\begin{circuitikz}[american resistors, american voltages,american currents]
\coordinate (A) at (3,3);
\coordinate (B) at (6,3);
  \draw
  (-3,0) to [isource, l= $I_{g1}$] (-3,3)
  (0,3) to [R, l= $R_1$,*-*] (0,0)
  (3,3) to [short, -, i = $I_1$] (0,3)
  (0,3) to [short, -] (-3,3)
  (3,3) to [R, l= $R_2$, i=$I_2$,*-*] (3,0)
   (3,3) to [R, l= $R_3$, i=$I_3$] (6,3)
   (6,3) to [R, l= $R_4$, i=$I_4$,*-*] (6,0)
   (6,3) to [short, -, i = $I_5$] (9,3)
   (9,3) to [short, -] (12,3)
   (9,3) to [R, l= $R_5$,*-* ] (9,0)
   (12,0) to [isource, l_= $I_{g2}$] (12,3)
   (12,0) to [short, -] (-3,0); 
   \node[label=above:A] (A) at ($(A)$) {};
   \node[label=above:B] (B) at ($(B)$) {};
   %\draw[thin, <-] (1.5,1.5)node{$I_a$}  ++(-60:0.5) arc (-60:170:0.5);
   %\draw[thin, <-] (4.5,1.5)node{$I_b$}  ++(-60:0.5) arc (-60:170:0.5);
   %\draw[thin, <-] (7.5,1.5)node{$I_c$}  ++(-60:0.5) arc (-60:170:0.5);
%   (0,2) to [short, -] (0,0)
%   (0,0) to [R, l_= $2\Omega$, ] (6,0)
%   (6,0) to [short, -] (6,2);
\end{circuitikz}

\end{document}
