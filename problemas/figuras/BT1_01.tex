\documentclass{standalone}
\usepackage{mathpazo}
\usepackage{circuitikz}
\usetikzlibrary{calc}

\begin{document}

\begin{circuitikz}[american resistors, american voltages]
  \coordinate(A) at (3,3);
  \coordinate(E) at (0,0);
  \draw
  (0,3) to [R, l_= $R_1$, i= $I_1$,-*] (A)
  (A) to [R, l=$R_2$, i = $I_2$,-*] (3,0) 
  (A) to [short, -*] ++(2,0) coordinate (C)
  (C) to [R, l=$R_3$, i = $I_3$,-*] (5,0) 
  (C) to [short, -] ++(2,0) coordinate (D)
  (D) to [R, l=$R_4$, i = $I_4$] (7,0) 
  (7,0) to [short, -] (E)
  (0,3) to [esource, v_= ${V}$] (E);
\end{circuitikz}

\end{document}

% \documentclass{standalone}
% \usepackage{mathpazo}
% \usepackage{circuitikz}
% \usetikzlibrary{calc}

% \begin{document}

% \begin{circuitikz}[european resistors, american voltages]
%   \coordinate (Ag) at (0,4);
%   \coordinate (Bg) at (0,0);
%   \node[label=above right:A] (A) at ($(Ag) + (2,0)$) {};
%   \node[label=below right:B] (B) at ($(Bg) + (2,0)$) {};
%   \draw
%   (Ag) to [R, l_= $Z_{th}$, i^>= $I$] (A)
%   (Ag) to [esource, v_= $\epsilon_{th}$] (Bg)
%   (Bg) to [short] (B)
%   (A) to [open, *-*, v^= $U_{AB}$] (B);
%   \draw
%   (A) to [short] ++(1.5,0)
%   to [R, l = $Z_L$] ($(B) + (1.5,0)$)
%   to [short] ++(-1.5,0)
%   to [short,*-] (B);
% \end{circuitikz}

% \end{document}