\documentclass{standalone}
\usepackage{mathpazo}
\usepackage{tikz}
\usepackage{circuitikz}
\usetikzlibrary{calc}
\usepackage{siunitx}

\begin{document}

\begin{circuitikz}[american resistors, american voltages]
\coordinate (D) at (3,0);
\coordinate (C) at (9,0);
\coordinate (B) at (6,5);
\coordinate (A) at (3,5);
  \draw
   (3,0) to [esource, v= $\qty{8}{\volt}$] (3,5)
   (3,5) to [R, l=$\qty{4}{\ohm}$, i=$I_3$] (6,5)
   (6,5) to [R, l=$\qty{2}{\ohm}$] (6,2)
   (6,2) to [esource, v=$\qty{8}{\volt}$, i=$I_4$] (6,0)
   (3,0) to [short,-, i^<=$I_7$] (6,0)
   (6,0) to [R, l=$\qty{1}{\ohm}$, i^<=$I_6$] (9,0)
   (9,0) to [R, l=$\qty{2}{\ohm}$] (9,5)
   (6,5) to [R, l=$\qty{1}{\ohm}$, i=$I_5$] (9,5);

  \node[label=below:D] (D) at ($(D)$) {};
  \node[label=above:A] (A) at ($(A)$) {};
  \node[label=below:C] (C) at ($(C)$) {};
  \node[label=above:B] (B) at ($(B)$) {};
  \draw[thin, <-] (4.5,2.5)node{$I_b$}  ++(-60:0.5) arc (-60:170:0.5);
  \draw[thin, <-] (7.5,2.5)node{$I_c$}  ++(-60:0.5) arc (-60:170:0.5);

\end{circuitikz}

\end{document}
