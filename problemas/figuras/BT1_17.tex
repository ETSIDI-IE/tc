
\documentclass{standalone}
\usepackage{mathpazo}
\usepackage{circuitikz}
\usetikzlibrary{calc}

\begin{document}

\begin{circuitikz}[american resistors, american voltages, american currents]
\coordinate (A) at (5,4);
\coordinate (B) at (5,1);
  \draw
  (5,7) to [esource, v_=$E$] (2,7)
  (2,7) to [R, l=$R_1$] (2,4)
  (2,4) to [R, l=$R_2$, *-*] (5,4)
  (5,4) to [short,-] (5,7)
  (5,4) to [R, l=$R_L$] (5,1)
  (2,4) to [short] (2,1)
  (2,1) to [isource, l=$I_g$, *-*] (B)
  (2,1) to [short] (2, -1)
  (2,-1) to [R, l=$R_3$] (5,-1)
  (5,1) to [short] (5,-1);
  %(2,1) to [short,-] (2,4);
  \node[label=right:A] (A) at ($(A)$) {};
   \node[label=right:B] (B) at ($(B)$) {};
%\draw[thin, <-] (3.5,5.5)node{$I_1$}  ++(-60:0.5) arc (-60:170:0.5);
%  \draw[thin, <-] (3.5,2.5)node{$I_2$}  ++(-60:0.5) arc (-60:170:0.5);

\end{circuitikz}

\end{document}
