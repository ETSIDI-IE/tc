\documentclass[12pt]{article}
\usepackage[a4paper]{geometry}
\usepackage{fullpage}
\usepackage[T1]{fontenc}
\usepackage[utf8]{inputenc}
\usepackage{graphicx}
\usepackage{mathpazo}
\pagenumbering{gobble}
\usepackage{siunitx}
\DeclareSIUnit\voltampere{VA}
\usepackage{amsmath}
\usepackage[spanish]{babel}
\usepackage{steinmetz}
\usepackage{enumitem}

\begin{document}

\title{}

\section{Transitorio de primer orden}

\subsection{FM 4.2}

Calcular la corriente $i(t)$ para $t > 0$. 

\begin{minipage}{0.5\textwidth}
\includegraphics{figs/FM_4_2}
\end{minipage}
\hfill
\begin{minipage}{0.5\textwidth}
Datos:
\begin{align*}
  \epsilon &= \SI{24}{\volt}\\
  R_1 &= \SI{8}{\ohm}\\
  R_2 &= \SI{4}{\ohm}\\
  R_3 &= \SI{4}{\ohm}\\
  L &= \SI{15}{\henry}
\end{align*}
\end{minipage}

\subsubsection*{Solución}

Calculamos las condiciones iniciales ($t = 0^-$)

Dibujamos el circuito para $t < 0$ y obtenemos:

\begin{minipage}{0.3\textwidth}
\includegraphics{figs/FM_4_2_t0-}
\end{minipage}
\begin{minipage}{0.7\textwidth}
  \begin{equation*}
    i(t) = \frac{\epsilon}{R_2 + R_3}
  \end{equation*}
\end{minipage}

Por tanto, $i(0^-) = \SI{3}{\ampere}$. Al tratarse de una bobina, $i(0^+) = i(0^-) = \SI{3}{\ampere}$.

A continuación dibujamos el circuito para $t > 0$ para obtener la respuesta natural y la respuesta forzada.

Para obtener la respuesta natural apagamos las fuentes. En este circuito obtenemos:

\begin{minipage}{0.3\textwidth}
\includegraphics{figs/FM_4_2_natural}
\end{minipage}
\begin{minipage}{0.7\textwidth}
  \begin{align*}
    R_{th} &= R_3 + R_1||R_2 = 20/3\si{\ohm}\\
    \tau &= L/R_{th} = 9/4\si{\second}\\
    i_n(t) &= A \cdot e^{-\frac{t}{\tau}} = A \cdot e^{-4t/9}
  \end{align*}
\end{minipage}
Queda por determinar la constante de integración.

Para obtener la respuesta forzada volvemos a activar las fuentes. En este circuito obtenemos:

\begin{minipage}{0.3\textwidth}
\includegraphics{figs/FM_4_2_forzada}
\end{minipage}
\begin{minipage}{0.7\textwidth}
  \begin{align*}
    I_g &= \frac{\epsilon}{R_2 + R_1||R_3}\\
    i_\infty(t) &= I_g \cdot \frac{G_3}{G_3 + G_1} = \SI{2.4}{\ampere}
  \end{align*}
\end{minipage}

Con estos dos resultados podemos obtener la respuesta completa:

\begin{align*}
  i(t) &= i_n(t) + i_\infty(t)\\
  i(t) &= A \cdot e^{-4t/9} + 2.4
\end{align*}

Para determinar la constante de integración recurrimos a las condiciones iniciales:

\begin{align*}
  i(0^+) &= A + 2.4\\
  i(0^+) &= 3\\
  A &= 0.6
\end{align*}

Por tanto,

\begin{equation*}
  i(t) = 0.6 \cdot e^{-4t/9} + 2.4
\end{equation*}

\clearpage

\subsection{FM 4.3}

Calcular la tensión en bornes del condensador para $t > 0$.

\begin{minipage}{0.5\textwidth}
  \includegraphics[scale=0.85]{figs/FM_4_3}
\end{minipage}
\hfill
\begin{minipage}{0.5\textwidth}
Datos:
\begin{align*}
  \epsilon &= \SI{20}{\volt}\\
  I_g &= \SI{4}{\ampere}\\
  R_1 &= \SI{6}{\ohm}\\
  R_2 &= \SI{4}{\ohm}\\
  R_3 &= \SI{12}{\ohm}\\
  C &= \SI[parse-numbers=false]{1/16}{\farad}      
\end{align*}

\end{minipage}

\subsubsection*{Solución}

Calculamos las condiciones iniciales ($t = 0^-$)

Dibujamos el circuito para $t < 0$ y obtenemos:

\begin{minipage}{0.3\textwidth}
\includegraphics{figs/FM_4_3_t0-}
\end{minipage}
\begin{minipage}{0.7\textwidth}
  \begin{equation*}
    u_C(t) = I_g \cdot R_1 
  \end{equation*}
\end{minipage}

Por tanto, $u_c(0^-) = \SI{24}{\volt}$. Al tratarse de un condensador, $u_C(0^+) = u_C(0^-) = \SI{24}{\volt}$.

A continuación dibujamos el circuito para $t > 0$ para obtener la respuesta natural y la respuesta forzada.

Para obtener la respuesta natural apagamos las fuentes. En este circuito obtenemos:

\begin{minipage}{0.3\textwidth}
\includegraphics{figs/FM_4_3_natural}
\end{minipage}
\begin{minipage}{0.7\textwidth}
  \begin{align*}
    R_{th} &= R_2 + R_3 = \SI{16}{\ohm}\\
    \tau &= C/G_{th} = \SI{1}{\second}\\
    u_{Cn}(t) &= A \cdot e^{-\frac{t}{\tau}} = A \cdot e^{-t}
  \end{align*}
\end{minipage}
Queda por determinar la constante de integración.

Para obtener la respuesta forzada volvemos a activar las fuentes. En este circuito obtenemos:

\begin{minipage}{0.3\textwidth}
\includegraphics{figs/FM_4_3_forzada}
\end{minipage}
\begin{minipage}{0.7\textwidth}
  \begin{equation*}
    u_{c\infty}(t) = \epsilon = \SI{20}{\volt}
  \end{equation*}
\end{minipage}

Con estos dos resultados podemos obtener la respuesta completa:

\begin{align*}
  u_C(t) &= u_{Cn}(t) + u_{c\infty}(t)\\
  u_C(t) &= A \cdot e^{-t} + 20
\end{align*}

Para determinar la constante de integración recurrimos a las condiciones iniciales:

\begin{align*}
  u_C(0^+) &= A + 20\\
  u_C(0^+) &= 24\\
  A &= \SI{4}{\volt}
\end{align*}

Por tanto,

\begin{equation*}
  u_C(t) = 4 \cdot e^{-t} + 20
\end{equation*}

\clearpage

\subsection{HKD 8.4}

Determina las corrientes $i_L(t)$ e $i_1(t)$ para $t > 0$.
\begin{center}
\includegraphics{figs/HKD_8_4}
\end{center}

% \subsection{HKD 8.10}

% Determina la tensión $u_c(t)$ y la corriente $i(t)$ para $t > 0$.
% \begin{center}
% \includegraphics{figs/HKD_8_10}
% \end{center}

\section{Transitorio de segundo orden}

\subsection{FM 4.8}

El circuito de la figura ha alcanzado el régimen permanente con el interruptor cerrado. El interruptor se abre en $t = 0$. Calcula las expresiones de la tensión en bornes del condensador y de la corriente por la bobina para $t > 0$.

\vspace*{1cm}

\begin{minipage}{0.7\textwidth}
  \includegraphics[scale=0.95]{figs/FM_4_8}
\end{minipage}
\hfill
\begin{minipage}{0.3\textwidth}
Datos:
\begin{align*}
  \epsilon_g &= \SI{10}{\volt}\\
  R_1 &= \SI{10}{\ohm}\\
  R_2 &= \SI{5}{\ohm}\\
  L &= \SI{2.5}{\henry}\\
  C &= \SI{0.2}{\farad}      
\end{align*}
\end{minipage}

\subsection{FM 4.9}

En el circuito de la figura, calcula la tensión $u_c(t)$ para $t > 0$.

\vspace*{1cm}

\begin{minipage}{0.7\textwidth}
  \includegraphics[scale=0.8]{figs/FM_4_9}
\end{minipage}
\hfill
\begin{minipage}{0.3\textwidth}
Datos:
\begin{align*}
  \epsilon_g &= \SI{4}{\volt}\\
  R_1 &= \SI{2}{\ohm}\\
  R_2 &= \SI{2}{\ohm}\\
  L &= \SI{2}{\henry}\\
  C &= \SI{0.25}{\farad}      
\end{align*}
\end{minipage}


\end{document}