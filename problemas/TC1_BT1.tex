\documentclass[12pt]{article}
\usepackage[a4paper]{geometry}
\usepackage{fullpage}
\usepackage[T1]{fontenc}
\usepackage[utf8]{inputenc}
\usepackage{graphicx}
\usepackage{mathpazo}
\pagenumbering{gobble}
\usepackage{siunitx}
\DeclareSIUnit\voltampere{VA}
\DeclareSIUnit\kWh{kWh}
\usepackage{amsmath}
\usepackage[spanish]{babel}
\usepackage{steinmetz}

\renewcommand{\thesection}{Problema \arabic{section}}

\begin{document}

\title{}

\date{Curso 2020-21}

\section{}

Un generador cuya fuerza electromotriz es de \SI{120}{V} y resistencia interna \SI{0.2}{\ohm}, entrega una corriente de \SI{20}{\ampere} a un motor situado a \SI{300}{\meter} de distancia y de resistencia interna \SI{0.5}{\ohm}. La línea es de cobre de resistividad $\SI{17.24}{\ohm\milli\meter\squared\per\meter}$. Sabiendo que el motor absorbe \SI{10.2}{\kWh} en 5 horas, hallar: 
\begin{enumerate}
\item Fuerza contraelectromotriz del motor.
\item Sección de los conductores.
\item Rendimiento del motor, del generador, de la línea y rendimiento total.
\item Balance general de potencias.
\end{enumerate}

\section{}
Un generador de corriente continua alimenta a dos cargas. La primera está situada a \SI{2100}{\meter}, tiene una resistencia de \SI{215}{\ohm} y rendimiento unidad. La segunda está situada a \SI{270}{\meter} después de la primera, tiene una potencia de \SI{4662}{\watt}, un rendimiento del 75\%, y una tensión aplicada de \SI{420}{\volt}.

Sabiendo que la línea es de cobre, de \SI{6}{\milli\meter\squared} de sección, y que la resistividad es de $\SI{17.24}{\ohm\milli\meter\squared\per\meter}$, determinar:

\begin{enumerate}
\item Tensión en bornes del generador.
\item Intensidad entregada por el generador.
\item Rendimiento de la instalación.
\end{enumerate}

\section{}

Convierte en fuente de tensión o intensidad, según corresponda.

\includegraphics{figs/Conversion_Fuentes.pdf}
\includegraphics{figs/Conversion_Fuentes_2.pdf}
\includegraphics{figs/Conversion_Fuentes_3.pdf}
\includegraphics{figs/Conversion_Fuentes_4.pdf}

\section{}

Calcula la resistencia equivalente entre A y B.

\includegraphics{figs/CircuitoResistivo_FM.pdf}

\end{document}





