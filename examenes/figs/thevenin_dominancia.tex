\documentclass{standalone}

\usepackage{mathpazo}

\usepackage[american voltages, american currents, european resistors]{circuitikz}

\begin{document}

\begin{circuitikz}
  \draw
  (0,0) node[] (K) {}
  (0,3) node[] (H) {}
  (0,6) node[] (E) {}
  (0,9) node[] (D) {}
  (3,0) node[] (L) {}
  (3,3) node[] (I) {}
  (3,6) node[] (F) {}
  (3,9) node[] (C) {}
  (6,0) node[] (B) {}
  (6,3) node[] (J) {}
  (6,6) node[] (G) {}
  (6,9) node[] (A) {};
  \draw
  (K) to[sV, v= $\epsilon_1$] (L)
  (K) to [short] (H)
  (I) to[sV, v= $\epsilon_2$] (H)
  to[R, l= $Z_1$] (E)
  to[R, l= $Z_3$] (F)
  (E) to[short] (D)
  (C) to[sV, v= $\beta I_{Z2}$] (D)
  (C) to[R, l_= $Z_4$] (F)
    (I) to[R, l = $Z_2$, i = $I_{Z2}$] (L)
  (L) to[sV, v=$\epsilon_3$] (B)
  (C) to[short] (A);
  \draw
  (H) to[short] ++(.85,.85)
  to[I, l_=$I_{g2} + \alpha U_{Z6}$] ($(E) + (.85,-.85)$)
  to[short] (E)
  (E) to[short] ++(.85,-.85)
  to[I] ($(F) + (-.85,-.85)$)
  to[short] (F);
  \draw
  (A) to[short, *-*] ++(2,0) node[label =$A$]{}
  (B) to[short, *-*] ++(2,0) node[label =$B$]{};
\end{circuitikz}

\end{document}
