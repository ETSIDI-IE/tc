% Created 2018-09-03 lun 09:51
% Intended LaTeX compiler: pdflatex
\documentclass[xcolor={usenames,svgnames,dvipsnames}]{beamer}
\usepackage[utf8]{inputenc}
\usepackage[T1]{fontenc}
\usepackage{graphicx}
\usepackage{grffile}
\usepackage{longtable}
\usepackage{wrapfig}
\usepackage{rotating}
\usepackage[normalem]{ulem}
\usepackage{amsmath}
\usepackage{textcomp}
\usepackage{amssymb}
\usepackage{capt-of}
\usepackage{hyperref}
\usepackage{color}
\usepackage{listings}
\usepackage[spanish]{babel}
\usecolortheme{rose}
\setbeamercolor{alerted text}{fg=Blue}
\setbeamerfont{alerted text}{series=\bfseries}
\setbeamerfont{block title}{series=\bfseries}
\setbeamercolor{block title}{bg=structure.fg!20!bg!50!bg}
\setbeamercolor{block body}{use=block title,bg=block title.bg}
\setbeamertemplate{navigation symbols}{}
\AtBeginSection[]{\begin{frame}[plain]\tableofcontents[currentsection,sectionstyle=show/shaded]\end{frame}}
\AtBeginSubsection[]{\begin{frame}[plain]\tableofcontents[currentsubsection,sectionstyle=show/shaded,subsectionstyle=show/shaded]\end{frame}}
\lstset{keywordstyle=\color{blue}, commentstyle=\color{gray!90}, basicstyle=\ttfamily\small, columns=fullflexible, breaklines=true,linewidth=\textwidth, backgroundcolor=\color{gray!23}, basewidth={0.5em,0.4em}, literate={¡}{{\textexclamdown}}1 {á}{{\'a}}1 {ñ}{{\~n}}1 {é}{{\'e}}1 {ó}{{\'o}}1 {í}{{\'i}}1 {ú}{{\'u}}1 {º}{{\textordmasculine}}1, showstringspaces=false}
\usepackage{mathpazo}
\hypersetup{colorlinks=true, linkcolor=Blue, urlcolor=Blue}
\usepackage{fancyvrb}
\DefineVerbatimEnvironment{verbatim}{Verbatim}{fontsize=\tiny, formatcom = {\color{black!70}}}
\beamertemplatenavigationsymbolsempty
\setbeamertemplate{footline}[frame number]
\usetheme{Goettingen}
\usefonttheme{serif}
\author{Oscar Perpiñán Lamigueiro}
\date{Septiembre 2018}
\title{Bibliografía}
\subtitle{Teoría de Circuitos III}
\hypersetup{
 pdfauthor={Oscar Perpiñán Lamigueiro},
 pdftitle={Bibliografía},
 pdfkeywords={},
 pdfsubject={},
 pdfcreator={Emacs 25.2.2 (Org mode 9.1.13)}, 
 pdflang={Spanish}}
\begin{document}

\maketitle


\begin{frame}[label={sec:orgbf658ec}]{Bibliografía}
\begin{itemize}
\item Circuitos eléctricos (\alert{FM})
\item Análisis de circuitos en ingeniería (\alert{HKD})
\item Fundamentos de circuitos eléctricos (\alert{AS})
\item Circuitos eléctricos, Volumen II (\alert{PO})
\end{itemize}
\end{frame}


\begin{frame}[label={sec:org669a934}]{Circuitos Eléctricos (\alert{FM})}
\begin{block}{Datos}
\begin{itemize}
\item Autor: J. Fraile Mora
\item Editorial: Pearson Educación, S.A., Madrid, 2012
\item \href{https://ingenio.upm.es/primo-explore/fulldisplay?docid=34UPM\_ALMA2150534070004212\&context=L\&vid=34UPM\_VU1\&search\_scope=TAB1\_SCOPE1\&tab=tab1\&lang=es\_ES}{Disponibilidad en Biblioteca UPM}
\item Problemas resueltos y soluciones numéricas al final de cada capítulo.
\end{itemize}
\end{block}

\begin{block}{Capítulos}
\begin{itemize}
\item Capítulo 4. Régimen transitorio de los circuitos eléctricos
\item Apéndice 2. Transformada de Laplace
\item \emph{Sección 2.15 Resonancia en CA, y Sección 2.16  filtros eléctricos}
\item \emph{Sección 1.17. Cuadripolos}
\end{itemize}
\end{block}
\end{frame}

\begin{frame}[label={sec:org157b968}]{Análisis de Circuitos en Ingeniería (\alert{HKD})}
\begin{block}{Datos}
\begin{itemize}
\item Autores: W.H. Hayt; J.E. Kemmerly; S.M. Durbin
\item Editorial: 8ª ed. McGraw-Hill, 2012
\item Disponibilidad en Biblioteca UPM: \href{https://ingenio.upm.es/primo-explore/fulldisplay?docid=34UPM\_ALMA2167505120004212\&context=L\&vid=34UPM\_VU1\&search\_scope=TAB1\_SCOPE1\&isFrbr=true\&tab=tab1\&lang=es\_ES}{8ºEd.} y \href{https://ingenio.upm.es/primo-explore/fulldisplay?docid=34UPM\_ALMA2154460960004212\&context=L\&vid=34UPM\_VU1\&search\_scope=TAB1\_SCOPE1\&isFrbr=true\&tab=tab1\&lang=es\_ES}{7ªEd.}
\item Soluciones numéricas a problemas \href{http://highered.mheducation.com/sites/0073529575/student\_view0/answers\_to\_selected\_problems.html}{online}.
\end{itemize}
\end{block}

\begin{block}{Capítulos}
\begin{itemize}
\item Capítulo 8: Circuitos de primer orden
\item Capítulo 9: Circuitos de segundo orden
\item Capítulos 14 y 15: Transformada de Laplace
\item Capítulo 19 y Apéndice 1: Variables de Estado
\item Capítulo 16: Análisis en Frecuencia
\item Capítulo 17: Cuadripolos
\end{itemize}
\end{block}
\end{frame}

\begin{frame}[label={sec:org178b2f7}]{Fundamentos de Circuitos Eléctricos (\alert{AS})}
\begin{block}{Datos}
\begin{itemize}
\item Autores: C. K. Alexander; M. N. O. Sadiku
\item Editorial: 3ª ed. McGraw-Hill, 2006
\item \href{https://ingenio.upm.es/primo-explore/fulldisplay?docid=34UPM\_ALMA2164599810004212\&context=L\&vid=34UPM\_VU1\&search\_scope=TAB1\_SCOPE1\&isFrbr=true\&tab=tab1\&lang=es\_ES}{Disponibilidad en Biblioteca UPM}
\item Soluciones numéricas a problemas en apéndice.
\end{itemize}
\end{block}
\begin{block}{Capítulos}
\begin{itemize}
\item Capítulo 7: Circuitos de primer orden
\item Capítulo 8: Circuitos de segundo orden
\item Capítulos 15 y 16: Transformada de Laplace
\item Capítulo 14: Análisis en Frecuencia
\item Capítulo 19: Cuadripolos
\end{itemize}
\end{block}
\end{frame}

\begin{frame}[label={sec:org307bb66}]{Circuitos Eléctricos (\alert{PO})}
\begin{block}{Datos}
\begin{itemize}
\item Autores: A. Pastor y J. Ortega
\item Editorial: UNED
\item \href{https://ingenio.upm.es/primo-explore/fulldisplay?docid=34UPM\_ALMA2148217180004212\&context=L\&vid=34UPM\_VU1\&search\_scope=TAB1\_SCOPE1\&tab=tab1\&lang=es\_ES}{Disponibilidad en Biblioteca UPM}
\item Problemas resueltos al final de cada capítulo.
\end{itemize}
\end{block}

\begin{block}{Capítulos}
\begin{itemize}
\item Capítulo 14: Circuitos de primer orden
\item Capítulo 15: Circuitos de segundo orden
\item Capítulo 16: Transformada de Laplace
\item Capítulo 17: Variables de Estado
\item Capítulo 23: Análisis en Frecuencia
\item Capítulo 19 y 20: Cuadripolos
\end{itemize}
\end{block}
\end{frame}
\end{document}