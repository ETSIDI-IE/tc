\documentclass[aspectratio=169, usenames,svgnames,dvipsnames]{beamer}
\usepackage[utf8]{inputenc}
\usepackage[T1]{fontenc}
\usepackage{graphicx}
\usepackage{grffile}
\usepackage{longtable}
\usepackage{wrapfig}
\usepackage{rotating}
\usepackage[normalem]{ulem}
\usepackage{amsmath}
\usepackage{textcomp}
\usepackage{amssymb}
\usepackage{capt-of}
\usepackage{hyperref}
\usepackage{color}
\usepackage{listings}
\usepackage{mathpazo}
\usepackage{gensymb}
\usepackage{amsmath}
\usepackage{diffcoeff}
\usepackage{steinmetz}
\usepackage{mathtools}
\bibliographystyle{plain}
\usepackage{siunitx}
\sisetup{output-decimal-marker={,}}
\DeclareSIUnit{\watthour}{Wh}
\hypersetup{colorlinks=true, linkcolor=Blue, urlcolor=Blue}
\renewcommand{\thefootnote}{\fnsymbol{footnote}}
\newcommand{\laplace}[1]{\mathbf{#1}(\mathbf{s})}
\newcommand{\slp}{\mathbf{s}}
\newcommand{\fasor}[1]{\mathbf{#1}(\omega)}
\newcommand{\atan}{\mathrm{atan}}
\parskip=5pt
\usetheme{Boadilla}
\usecolortheme{rose}
\usefonttheme{serif}
\date{}
\title{\LARGE Bibliografía \vspace{5mm}}
\subtitle{Teoría de Circuitos}
\setbeamercolor{alerted text}{fg=blue!50!black} \setbeamerfont{alerted text}{series=\bfseries}
\AtBeginSubsection[]{\begin{frame}[plain]\tableofcontents[currentsubsection,sectionstyle=show/shaded,subsectionstyle=show/shaded/hide]\end{frame}}
\AtBeginSection[]{\begin{frame}[plain]\tableofcontents[currentsection,hideallsubsections]\end{frame}}
\beamertemplatenavigationsymbolsempty
\setbeamertemplate{footline}[frame number]
\setbeamertemplate{itemize items}[triangle]
\setbeamertemplate{enumerate items}[circle]
\setbeamertemplate{section in toc}[circle]
\setbeamertemplate{subsection in toc}[circle]
\hypersetup{
 pdfauthor={},
 pdftitle={Bibliografía},
 pdfkeywords={},
 pdfsubject={},
 pdfcreator={}, 
 pdflang={Spanish}}
 
\begin{document}

\maketitle


\begin{frame}{Bibliografía}
    \begin{itemize}
    \item ``Circuitos eléctricos'' (\alert{Fraile Mora})
    \vspace{3mm}
    \item ``Análisis de circuitos en ingeniería'' (\alert{Hayt})
    \vspace{3mm}
    \item ``Fundamentos de circuitos eléctricos'' (\alert{Sadiku})
    \vspace{3mm}
    \item ``Circuitos eléctricos. Volumen I'' (\alert{Pastor})
    \end{itemize}
\end{frame}

%%%%%%%%%%%%%%%%%%%

\begin{frame}{``Circuitos eléctricos'' (\alert{Fraile Mora})}
    \begin{block}{Datos}
        \begin{itemize}
        \item Autor: J. Fraile Mora
        \item Editorial: Pearson Educación, S.A., Madrid, 2012
        \item Disponibilidad en biblioteca UPM: \href{https://ingenio.upm.es/primo-explore/fulldisplay?docid=34UPM\_ALMA2150534070004212\&context=L\&vid=34UPM\_VU1\&search\_scope=TAB1\_SCOPE1\&tab=tab1\&lang=es\_ES}{Teoría} y \href{https://ingenio.upm.es/permalink/f/1vo0cl5/34UPM\_ALMA2164586310004212}{Problemas}
        \item Problemas resueltos y soluciones numéricas al final de cada capítulo
        \end{itemize}
    \end{block}

    \begin{block}{Capítulos relevantes}
        \begin{itemize}
        \item Bloque Temático (BT) 1: $\;\;$ Capítulo 1
        \item BT 2: $\;\;$ Capítulo 2
        \item BT 3: $\;\;$ Capítulo 3
        \item BT 4: $\;\;$ Capítulo 5
        \end{itemize}
    \end{block}
\end{frame}

%%%%%%%%%%%%%%%%%%%

\begin{frame}{``Análisis de circuitos en ingeniería'' (\alert{Hayt})}
    \begin{block}{Datos}
        \begin{itemize}
        \item Autores: W. H. Hayt; J. E. Kemmerly; S. M. Durbin
        \item Editorial: 8ª ed. McGraw-Hill, 2012
        \item Disponibilidad en Biblioteca UPM: \href{https://ingenio.upm.es/primo-explore/fulldisplay?docid=34UPM\_ALMA2167505120004212\&context=L\&vid=34UPM\_VU1\&search\_scope=TAB1\_SCOPE1\&isFrbr=true\&tab=tab1\&lang=es\_ES}{8º ed.} y \href{https://ingenio.upm.es/primo-explore/fulldisplay?docid=34UPM\_ALMA2154460960004212\&context=L\&vid=34UPM\_VU1\&search\_scope=TAB1\_SCOPE1\&isFrbr=true\&tab=tab1\&lang=es\_ES}{7ª ed.}
        \item Soluciones numéricas a problemas \href{http://highered.mheducation.com/sites/0073529575/student\_view0/answers\_to\_selected\_problems.html}{online} (en inglés)
        \end{itemize}
    \end{block}
    
    \begin{block}{Capítulos relevantes}
        \begin{itemize}
        \item BT 1: $\;\;$ Capítulos 1 al 5
        \item BT 2: $\;\;$ Capítulos 10 y 11
        \item BT 3: $\;\;$ Capítulo 12
        \item BT 4: $\;\;$ Capítulos 8 y 9
        \end{itemize}
    \end{block}
\end{frame}

%%%%%%%%%%%%%%%%%%%

\begin{frame}{``Fundamentos de circuitos eléctricos'' (\alert{Sadiku})}
    \begin{block}{Datos}
        \begin{itemize}
        \item Autores: C. K. Alexander; M. N. O. Sadiku
        \item Editorial: 3ª ed. McGraw-Hill, 2006
        \item \href{https://ingenio.upm.es/primo-explore/fulldisplay?docid=34UPM\_ALMA2164599810004212\&context=L\&vid=34UPM\_VU1\&search\_scope=TAB1\_SCOPE1\&isFrbr=true\&tab=tab1\&lang=es\_ES}{Disponibilidad en Biblioteca UPM}
        \item Soluciones numéricas a problemas en apéndice
        \end{itemize}
    \end{block}
    
    \begin{block}{Capítulos relevantes}
        \begin{itemize}
        \item BT 1: $\;\;$ Capítulos 1 al 4
        \item BT 2: $\;\;$ Capítulos 9 al 11
        \item BT 3: $\;\;$ Capítulo 12
        \item BT 4: $\;\;$ Capítulos 7 y 8
        \end{itemize}
        \end{block}
\end{frame}

%%%%%%%%%%%%%%%%%

\begin{frame}{``Circuitos eléctricos. Volumen 1'' (\alert{Pastor})}
    \begin{block}{Datos}
        \begin{itemize}
        \item Autores: A. Pastor y J. Ortega
        \item Editorial: UNED 2003
        \item \href{https://ingenio.upm.es/primo-explore/fulldisplay?docid=34UPM\_ALMA2148217180004212\&context=L\&vid=34UPM\_VU1\&search\_scope=TAB1\_SCOPE1\&tab=tab1\&lang=es\_ES}{Disponibilidad en Biblioteca UPM}
        \item Problemas resueltos al final de cada capítulo
        \end{itemize}
    \end{block}
    
    \begin{block}{Capítulos relevantes}
        \begin{itemize}
        \item BT 1: $\;\;$ Capítulos 1 al 5, y 8
        \item BT 2: $\;\;$ Capítulos 9 y 10, y 8
        \item BT 3: $\;\;$ Capítulos 12 y 13
        \item BT 4: $\;\;$ Capítulo 14
        \end{itemize}
    \end{block}
\end{frame}

\end{document}
