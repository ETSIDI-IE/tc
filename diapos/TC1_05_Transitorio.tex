% Created 2022-05-12 jue 22:49
% Intended LaTeX compiler: pdflatex
\documentclass[aspectratio=169, usenames,svgnames,dvipsnames]{beamer}
\usepackage[utf8]{inputenc}
\usepackage[T1]{fontenc}
\usepackage{graphicx}
\usepackage{grffile}
\usepackage{longtable}
\usepackage{wrapfig}
\usepackage{rotating}
\usepackage[normalem]{ulem}
\usepackage{amsmath}
\usepackage{textcomp}
\usepackage{amssymb}
\usepackage{capt-of}
\usepackage{hyperref}
\usepackage{color}
\usepackage{listings}
\usepackage{mathpazo}
\usepackage{gensymb}
\usepackage{amsmath}
\usepackage{diffcoeff}
\usepackage{steinmetz}
\usepackage{mathtools}
\bibliographystyle{plain}
\usepackage{siunitx}
\sisetup{output-decimal-marker=comma}
\DeclareSIUnit{\watthour}{Wh}
\hypersetup{colorlinks=true, linkcolor=Blue, urlcolor=Blue}
\renewcommand{\thefootnote}{\fnsymbol{footnote}}
\newcommand{\laplace}[1]{\mathbf{#1}(\mathbf{s})}
\newcommand{\slp}{\mathbf{s}}
\newcommand{\fasor}[1]{\mathbf{#1}(\omega)}
\newcommand{\atan}{\mathrm{atan}}
\parskip=5pt
\usetheme{Boadilla}
\usecolortheme{rose}
\usefonttheme{serif}
\author{Oscar Perpiñán Lamigueiro}
\date{}
\title{Introducción al Régimen Transitorio}
\subtitle{Teoría de Circuitos}
\setbeamercolor{alerted text}{fg=blue!50!black} \setbeamerfont{alerted text}{series=\bfseries}
\AtBeginSubsection[]{\begin{frame}[plain]\tableofcontents[currentsubsection,sectionstyle=show/shaded,subsectionstyle=show/shaded/hide]\end{frame}}
\AtBeginSection[]{\begin{frame}[plain]\tableofcontents[currentsection,hideallsubsections]\end{frame}}
\beamertemplatenavigationsymbolsempty
\setbeamertemplate{footline}[frame number]
\setbeamertemplate{itemize items}[triangle]
\setbeamertemplate{enumerate items}[circle]
\setbeamertemplate{section in toc}[circle]
\setbeamertemplate{subsection in toc}[circle]
\hypersetup{
 pdfauthor={Oscar Perpiñán Lamigueiro},
 pdftitle={Introducción al Régimen Transitorio},
 pdfkeywords={},
 pdfsubject={},
 pdfcreator={Emacs 27.1 (Org mode 9.4.6)}, 
 pdflang={Spanish}}
\begin{document}

\maketitle

\section{Conceptos Fundamentales}
\label{sec:orgeba49c3}

\subsection{¿Qué es el régimen transitorio?}
\label{sec:org411be45}
\begin{frame}[label={sec:org8aa5412}]{Permanente y Transitorio}
\begin{block}{Régimen permanente o estacionario}
Las tensiones y corrientes de un circuito son constantes (continua) o periódicas (alterna) (circuito estabilizado)
\end{block}
\begin{block}{Régimen transitorio}
\begin{itemize}
\item Para alcanzar el régimen permanente (o para alternar entre dos regímenes permanentes) el circuito atraviesa el régimen transitorio.
\item Posibles cambios: activación o apagado de fuentes, cambio en las cargas, cambio en el circuito (línea).
\item En general, el estado transitorio es indeseado en sistemas eléctricos, pero provocado en sistemas electrónicos.
\end{itemize}
\end{block}
\end{frame}

\begin{frame}[label={sec:org6935d2b}]{Acumulación de Energía}
\begin{block}{Régimen Permanente}
\alert{Energía acumulada} en \alert{bobinas} y \alert{condensadores}
\end{block}
\begin{block}{Régimen Transitorio}
\begin{itemize}
\item \alert{Redistribución} y \alert{disipación} de energía acumulada.
\item La redistribución de energía \alert{no} se puede realizar de forma \alert{inmediata}
\item \alert{Duración corta} (\(\si{\micro\second}\)) pero superior a 0, dependiendo de \alert{relación entre acumulación y disipación} (resistencia).
\end{itemize}
\end{block}
\end{frame}

\subsection{Respuesta de una red lineal}
\label{sec:org18fca1c}

\begin{frame}[label={sec:org5b6aa91}]{Ecuaciones integro-diferenciales}
Al aplicar Kirchhoff a un circuito lineal obtenemos ecuaciones integro-diferenciales. 

\[
  u_L(t) = L \cdot \diff{i_L(t)}{t}
  \leftrightarrow
  i_L(t) = \frac{1}{L} \int^t_{-\infty}u_L(t') \mathrm{d}t'
\]
\[
  i_C(t) = C \cdot \diff{u_C(t)}{t}
  \leftrightarrow
  u_C(t) = \frac{1}{C} \int^t_{-\infty}i_C(t') \mathrm{d}t'
\]

Por ejemplo, la ecuación de un circuito RLC será de la forma:

\[
  a \cdot \diff[2]{f(t)}{t} + b \cdot \diff{f(t)}{t} + c \cdot f(t) = g(t)
\]
\end{frame}

\begin{frame}[label={sec:org77d4c45}]{Respuesta completa de una red lineal}
La solución de esta ecuación para \(t > 0\) (respuesta completa del circuito lineal al transitorio) tiene dos componentes:

\[
 \boxed{f(t) = f_n(t) + f_\infty(t) }
 \]

\begin{itemize}
\item Respuesta \alert{natural} o propia, \(f_n(t)\):
\begin{itemize}
\item Respuesta sin fuentes.
\item Determinada por la energía almacenada previamente y por la configuración del circuito.
\item Contiene constantes de integración. Se necesita información del estado del circuito en el instante que da origen al transitorio.
\end{itemize}
\item Respuesta \alert{forzada} o particular, \(f_\infty(t)\):
\begin{itemize}
\item Determinada por las fuentes existentes en \(t > 0\).
\item Es la respuesta del circuito tras un tiempo suficiente, \(t \to \infty\) (régimen permanente).
\end{itemize}
\end{itemize}
\end{frame}

\begin{frame}[label={sec:orgfa38687}]{Condiciones iniciales}
\begin{itemize}
\item El instante del cambio se representa habitualmente con \(t = 0\):
\begin{itemize}
\item \(t = 0^-\): tiempo inmediatamente anterior al cambio.
\item \(t = 0^+\): tiempo inmediatamente posterior al cambio.
\end{itemize}

\item Las \alert{condiciones iniciales} son el estado del circuito en el instante temporal en el que se produce el cambio.

\item Determinan las \alert{constantes de integración} de la respuesta natural.

\item \alert{Se calculan} con las energías almacenadas en bobinas y condensadores en \(t = 0^-\).

\item \alert{Se aplican} a la topología del circuito en \(t = 0^+\).
\end{itemize}
\end{frame}

\begin{frame}[label={sec:org2933e8a}]{Resistencia}
No acumula energía: sigue los cambios de forma instantánea.

\[
u(t) = R i(t)
\]
\end{frame}



\begin{frame}[label={sec:org88790ce}]{Inductancia}
La corriente en una bobina no puede variar de forma abrupta (implica tensión infinita).

\[
u_L(t) = L \cdot \diff{i_L(t)}{t}
\leftrightarrow
i_L(t) = \frac{1}{L} \int^t_{-\infty}u_L(t') \mathrm{d}t'
\]

\[
\boxed{i_L(0^-) = i_L(0^+)}
\]
\end{frame}

\begin{frame}[label={sec:org4e6e103}]{Capacidad}
La tensión en un condensador no puede variar de forma abrupta (implica corriente infinita).

\[
i_C(t) = C \cdot \diff{u_C(t)}{t}
\leftrightarrow
u_C(t) = \frac{1}{C} \int^t_{-\infty}i_C(t') \mathrm{d}t'
\]

\[
\boxed{u_C(0^-) = u_C(0^+)}
\]
\end{frame}


\section{Circuitos de Primer Orden}
\label{sec:org3a8303c}
\begin{frame}[label={sec:org0fcb33c}]{Definición}
\begin{itemize}
\item Circuitos que tienen un \alert{único elemento de acumulación} (o \emph{varios elementos que pueden ser simplificados a un elemento equivalente}) y parte resistiva.
\item \alert{Ecuación diferencial de primer orden}: la respuesta natural es siempre una \alert{exponencial decreciente}.
\item Circuitos típicos:
\begin{itemize}
\item RL serie
\item RC paralelo
\end{itemize}
\end{itemize}
\end{frame}
\begin{frame}[label={sec:org58dc34f}]{Respuesta natural y forzada}
\begin{itemize}
\item El método de resolución analiza el circuito en tres etapas:
\begin{enumerate}
\item Cálculo de las \alert{condiciones iniciales}, analizando el circuito en \(t < 0\).
\item \alert{Respuesta natural}: análisis del circuito \emph{sin fuentes} en \(t > 0\) (la energía acumulada en \(t < 0\) se disipa en la resistencia).
\item \alert{Respuesta forzada}: análisis del circuito \emph{con fuentes} en \(t > 0\) (la respuesta está determinada por la forma de onda de las fuentes).
\end{enumerate}
\end{itemize}
\end{frame}

\subsection{Circuito RL serie}
\label{sec:org6bba37c}

\begin{frame}[label={sec:orgb2f41e2}]{Circuito básico}
\begin{itemize}
\item En \(t < 0\) la fuente alimenta el circuito RL (la bobina almacena energía).
\item En \(t = 0\) la fuente se desconecta.
\item En \(t > 0\) la bobina se descarga en la resistencia.
\end{itemize}
\begin{center}
\includegraphics[width=.9\linewidth]{../figs/transitorio_circuitoRL.pdf}
\end{center}
\end{frame}

\begin{frame}[label={sec:orge2cd24f}]{Respuesta natural}
\begin{center}
\includegraphics[height=0.25\textheight]{../figs/transitorio_circuitoRL_t0+.pdf}
\end{center}
Ecuaciones
\begin{align*}
  u_R(t) + u_L(t) &= 0\\
  R i + L\diff{i}{t} &= 0
\end{align*}

Solución Genérica
\[
  i(t) = A e^{st}
\]
Ecuación Característica
\[
  s + \frac{R}{L} = 0 \Rightarrow s = -\frac{R}{L}
\]
\end{frame}

\begin{frame}[label={sec:org7567c72}]{Condiciones Iniciales}
Analizando circuito para \(t < 0\) \ldots{} 
\begin{center}
\includegraphics[width=.9\linewidth]{../figs/transitorio_circuitoRL_t0-.pdf}
\end{center}
\ldots{}obtenemos  \(i(0^-) = I_0 = \frac{U_0}{R}\) 
\end{frame}

\begin{frame}[label={sec:org01bc563}]{Respuesta Natural}
Por otra parte, para \(t > 0\):
\begin{align*}
  i(t) &= A e^{-R/L t}\\
  i(0^+) &= A e^0 = A\\
\end{align*}

Y dada la condición de continuidad, \(i(0^+) = i(0^-)\):
\[
  A = I_0
\]
Por tanto, la respuesta natural es:
\[
  \boxed{i(t) = I_0 e^{-R/L t}}
\]
\end{frame}



\begin{frame}[label={sec:orgc46478b}]{Constante de tiempo}
\[
  \boxed{i(t) = I_0 e^{-t/\tau}}
\]

\begin{itemize}
\item \(\tau = \frac{L}{R}\) es la constante de tiempo (unidades [s]).
\item Ratio entre almacenamiento (\(L\)) y disipación (\(R\)).
\item Valores altos de \(\tau\) implican decrecimiento lento.
\item La respuesta natural \guillemotleft{}desaparece\guillemotright{} tras \(\simeq 5\tau\).
\end{itemize}
\begin{center}
\includegraphics[height=0.5\textheight]{../figs/constante_tiempo.pdf}
\end{center}
\end{frame}

\begin{frame}[label={sec:org0c07241}]{Balance Energético}
La energía acumulada en la bobina en \(t < 0\) se disipa en la resistencia en \(t > 0\)

\begin{align*}
  W_R &= \int_0^\infty R i^2(t)  \mathrm{d}t =\\
  &= \int_0^\infty R (I_0 e^{-t/\tau})^2  \mathrm{d}t = \\
  &= \frac{1}{2} L I_0^2 = W_L  
\end{align*}
\end{frame}

\begin{frame}[label={sec:org8688b1d}]{Respuesta forzada}
Cambiemos el funcionamiento del interruptor: en \(t > 0\) la fuente alimenta el circuito RL.
\begin{center}
\includegraphics[width=.9\linewidth]{../figs/transitorio_circuitoRL2.pdf}
\end{center}
\end{frame}

\begin{frame}[label={sec:org59b44b3}]{Respuesta forzada}
\begin{center}
\includegraphics[height=0.25\textheight]{../figs/transitorio_circuitoRL2_t0+.pdf}
\end{center}
Las ecuaciones son ahora:
\[
  u_R(t) + u_L(t) = u(t) \rightarrow R i + L\diff{i}{t} = U_0
\]

Para la solución particular, \(i_\infty\), se propone una función análoga a la excitación (analizando circuito para \(t > 0\))
\begin{align*}
  i(t) &= i_n(t) + i_\infty(t)\\
  i_n(t) &= A e^{st}\\
  i_\infty(t) &= U_0/R\\
\end{align*}
\end{frame}

\begin{frame}[label={sec:orgb2c96a7}]{Constante de integración}
Particularizamos las ecuaciones en \(t = 0^+\):
\begin{align*}
  i(0^+) &= i_n(0^+) + i_\infty(0^+)\\
  i(0^+) &= A + i_\infty(0^+)\\
  A &= i(0^+) - i_\infty(0^+)
\end{align*}
\end{frame}

\begin{frame}[label={sec:org1f2d008}]{Respuesta completa (ejemplo)}
\begin{align*}
  i(t) &= i_n(t) + i_\infty(t)\\
  i_n(t) &= A e^{st}\\
  i_\infty(t) &= U_0/R\\
  A &= i(0^+) - i_\infty(0^+)
\end{align*}

Suponiendo que la bobina está inicialmente descargada, \(i(0^-) = 0\), y teniendo en cuenta la condición de continuidad, \(i(0^+) = i(0^-) = 0\), obtenemos \(A= 0 - U_0/R\). 

La solución completa es:
\[
  \boxed{i(t) = \frac{U_0}{R}(1 - e^{-\frac{t}{\tau}})  }
\]
\end{frame}

\begin{frame}[label={sec:orgafcc767}]{Respuesta completa}
\[
  \boxed{i(t) = \frac{U_0}{R}(1 - e^{-\frac{t}{\tau}})  }
\]

\begin{center}
\includegraphics[height=0.45\textheight]{../figs/RespuestaCompleta_RL.pdf}
\end{center}
\end{frame}


\begin{frame}[label={sec:org5353378}]{Expresión general de la respuesta completa}
\[
\boxed{i(t) = \left[i(0^+) - i_\infty(0^+)\right] e^{-t/\tau} + i_\infty(t)}
\]

\begin{itemize}
\item \(i(0^+)\): corriente en la bobina, condiciones iniciales, \(i(0^-) = i(0^+)\).
\item \(i_\infty(t)\): corriente en la bobina en régimen permanente para \(t > 0\).
\item \(i_\infty(0^+)\): corriente en la bobina en régimen permanente particularizada en \(t = 0\).
\end{itemize}
\end{frame}

\begin{frame}[label={sec:org6f40344}]{Equivalente de Thévenin}
\begin{center}
\includegraphics[height=0.85\textheight]{../figs/Thevenin_PrimerOrden.pdf}
\end{center}
\end{frame}
\subsection{Circuito RC paralelo}
\label{sec:orga8c227c}

\begin{frame}[label={sec:orga8bef42}]{Circuito básico}
\begin{itemize}
\item En \(t <0\) la fuente alimenta el circuito RC (el condensador se carga).
\item En \(t = 0\) se desconecta la fuente
\item En \(t > 0\) el condensador comienza a descargarse en la resistencia.
\end{itemize}

\begin{center}
\includegraphics[height=0.45\textheight]{../figs/transitorio_circuitoRC.pdf}
\end{center}
\end{frame}

\begin{frame}[label={sec:org2c8fa07}]{Respuesta natural}
\begin{center}
\includegraphics[height=0.3\textheight]{../figs/transitorio_circuitoRC_t0+.pdf}
\end{center}

Ecuaciones
\begin{align*}
  i_R(t) + i_C(t) &= 0\\
  G u + C\diff{u}{t} &= 0
\end{align*}

Solución Genérica
\[
  u(t) = A e^{st}
\]

Respuesta natural
\[
  \boxed{u(t) = U_0 e^{-G/C t}}
\]
\end{frame}



\begin{frame}[label={sec:orga190a60}]{Constante de tiempo}
\begin{itemize}
\item \(\tau = \frac{C}{G}\) es la constante de tiempo (unidades [s]).
\item Ratio entre almacenamiento (\(C\)) y disipación (\(G\)).
\end{itemize}

\[
  \boxed{u(t) = U_0 e^{-t/\tau}  }
\]

\begin{center}
\includegraphics[height=0.55\textheight]{../figs/constante_tiempo.pdf}
\end{center}
\end{frame}

\begin{frame}[label={sec:org9a14ea5}]{Balance Energético}
La energía acumulada en el condensador en \(t < 0\) se disipa en la resistencia (conductancia) en \(t > 0\)

\[
  W_G = \int_0^\infty G u^2(t)  \mathrm{d}t = \frac{1}{2} C U_0^2 = W_C
\]
\end{frame}

\begin{frame}[label={sec:org6aaf7fc}]{Expresión general de la respuesta completa}
\[
\boxed{u(t) = \left[u(0^+) - u_\infty(0^+)\right] e^{-t/\tau} + u_\infty(t)}
\]

\begin{itemize}
\item \(u(0^+)\): tensión en el condensador, condiciones iniciales, \(u(0^-) = u(0^+)\).
\item \(u_\infty(t)\): tensión en el condensador en régimen permanente para \(t > 0\).
\item \(u_\infty(0^+)\): tensión en el condensador en régimen permanente particularizada en \(t = 0\).
\end{itemize}
\end{frame}

\begin{frame}[label={sec:orgb1a0f6d}]{Ejemplo con respuesta forzada}
\begin{center}
\includegraphics[height=0.3\textheight]{../figs/transitorio_circuitoRC2.pdf}
\end{center}

\[
\boxed{u(t) = \left[u(0^+) - u_\infty(0^+)\right] e^{-t/\tau} + u_\infty(t)}
\]

Suponiendo que el condensador está inicialmente descargado:
\begin{align*}
  u(0^+) &= u(0^-) = 0\\
  u_\infty(0^+) &= I_0/G\\
  u(t) &= \frac{I_0}{G}(1 - e^{-\frac{t}{\tau}})  
\end{align*}
\end{frame}

\begin{frame}[label={sec:org680f7e2}]{Equivalente de Norton}
\begin{center}
\includegraphics[height=0.85\textheight]{../figs/Thevenin_PrimerOrden.pdf}
\end{center}
\end{frame}

\subsection{Análisis Sistemático}
\label{sec:orgf499ac0}

\begin{frame}[label={sec:orgdd1a6fe}]{Equivalente de Thévenin/Norton}
\begin{center}
\includegraphics[height=0.85\textheight]{../figs/Thevenin_PrimerOrden.pdf}
\end{center}
\end{frame}

\begin{frame}[label={sec:orgc576ad3}]{Procedimiento General}
\begin{itemize}
\item Dibujar el circuito para \(t < 0\).
\begin{itemize}
\item Determinar variables en régimen permanente, \(u_c(t)\), \(i_L(t)\).
\item Particularizar para \(t = 0\), obteniendo \(u_c(0^-)\) o \(i_L(0^-)\).
\item Continuidad: \(u_c(0^+) = u_c(0^-)\), \(i_L(0^+) = i_L(0^-)\).
\end{itemize}
\item Dibujar el circuito para \(t > 0\).
\begin{itemize}
\item Calcular el equivalente de Thevenin (Norton) visto por el elemento de acumulación.
\item La constante de tiempo de la respuesta natural es \(\tau = \frac{L}{R_{th}}\) o \(\tau = \frac{C}{G_{th}}\).
\item Calcular las variables \(i_L(t)\) o \(u_c(t)\) en régimen permanente, obteniendo \(i_\infty(t)\) o \(u_\infty(t)\).
\end{itemize}
\item Obtener respuesta completa:
\end{itemize}
\begin{align*}
i_L(t) &= \left(i_L(0^+) - i_\infty(0^+)\right) e^{-t/\tau} + i_\infty(t)\\
u_C(t) &= \left(u_C(0^+) - u_\infty(0^+)\right) e^{-t/\tau} + u_\infty(t)\\
\end{align*}
\end{frame}


\section{Circuitos de Segundo Orden}
\label{sec:orga1f9c3e}
\begin{frame}[label={sec:orgeb7a8ac}]{Introducción}
\begin{itemize}
\item Circuitos que tienen \alert{dos elementos de acumulación} que intercambian energía, y parte resistiva que disipa energía.
\item \alert{Ecuación diferencial de segundo orden}: la respuesta natural incluye exponenciales decrecientes y quizás señal sinusoidal.
\item Circuitos típicos:
\begin{itemize}
\item RLC serie
\item RLC paralelo
\end{itemize}
\end{itemize}
\end{frame}
\begin{frame}[label={sec:orgb177e8f}]{Respuesta natural y forzada}
\begin{itemize}
\item El método de resolución analiza el circuito en dos etapas:
\begin{itemize}
\item Sin fuentes: \alert{respuesta natural} (la energía acumulada en \(t < 0\) se redistribuye).
\item Con fuentes: \alert{respuesta forzada} (determinada por la forma de onda de las fuentes).
\end{itemize}
\end{itemize}
\end{frame}

\subsection{Circuito RLC serie}
\label{sec:orga82d1ff}
\begin{frame}[label={sec:org678e447}]{Circuito básico}
\begin{center}
\includegraphics[width=.9\linewidth]{../figs/transitorio_circuitoRLC_serie.pdf}
\end{center}
\end{frame}


\begin{frame}[label={sec:org38d0452}]{Respuesta natural (t > 0)}
\begin{center}
\includegraphics[height=0.45\textheight]{../figs/transitorio_circuitoRLC_serie_t0+.pdf}
\end{center}

\[
  Ri(t) + L\diff{i(t)}{t} + \frac{1}{C}\int_{-\infty}^t i(t') \mathrm{d}t' = 0
\]

\[
  L\diff[2]{i}{t} + R\diff{i}{t} + \frac{1}{C} i = 0 \Rightarrow
  \boxed{\diff[2]{i}{t} + \frac{R}{L} \diff{i}{t} + \frac{1}{LC} i = 0}
\]
\end{frame}

\begin{frame}[label={sec:org6edf116}]{Solución}
\begin{block}{Ecuación diferencial}
\[
\diff[2]{i}{t} + \frac{R}{L} \diff{i}{t} + \frac{1}{LC} i = 0
\]
\[
    i_n(t) = A_1 e^{s_1 t} + A_2 e^{s_2 t}
\]
\end{block}

\begin{block}{Ecuación característica}
\[
s^2 + \frac{R}{L} s + \frac{1}{LC} = 0  
\]
\[
  s_{1,2} = -\frac{R}{2L} \pm \sqrt{\left(\frac{R}{2L}\right)^2 - \frac{1}{LC}}
\]
\end{block}
\end{frame}


\begin{frame}[label={sec:org069f8f6}]{Parámetros}
\begin{columns}
\begin{column}{.5\columnwidth}
\begin{align*}
  s^2 + \frac{R}{L} s + \frac{1}{LC} &= 0\\
  s^2 + 2\alpha s + \omega_0^2 &= 0  
\end{align*}

\[
  s_{1,2} = -\alpha \pm \sqrt{\alpha^2 - \omega_0^2}
\]
\[
  i_n(t) = A_1 e^{s_1 t} + A_2 e^{s_2 t}
\]
\end{column}

\begin{column}{.5\columnwidth}
\begin{align*}
  \alpha &= \frac{R}{2L}\\
  \omega_0 &= \frac{1}{\sqrt{LC}}\\
  \omega_d &= \sqrt{\omega_0^2 - \alpha^2}\\
  \xi &= \frac{\alpha}{\omega_0}
\end{align*}
\end{column}
\end{columns}
\vspace{1cm}
\begin{itemize}
\item \(\alpha\): coeficiente de amortiguamiento exponencial
\item \(\omega_0\): pulsación natural no amortiguada
\item \(\omega_d\): pulsación natural amortiguada
\item \(\xi\): factor de amortiguamiento
\end{itemize}
\end{frame}
\begin{frame}[label={sec:org8502e1e}]{Posibles soluciones}
\[
  \boxed{s_{1,2} = -\alpha \pm \sqrt{\alpha^2 - \omega_0^2}}
\]

\begin{block}{\(\alpha > \omega_0\), \(\xi > 1\)}
\begin{itemize}
\item \(s_{1,2}\): dos soluciones reales (negativas) distintas
\item Circuito \alert{sobreamortiguado}.
\end{itemize}
\end{block}

\begin{block}{\(\alpha = \omega_0\), \(\xi = 1\)}
\begin{itemize}
\item \(s_{1,2}\): solución real doble.
\item Circuito con \alert{amortiguamiento crítico}.
\end{itemize}
\end{block}

\begin{block}{\(\alpha < \omega_0\), \(\xi < 1\)}
\begin{itemize}
\item \(s_{1,2}\): dos soluciones complejas conjugadas
\item Circuito \alert{subamortiguado}.
\end{itemize}
\end{block}
\end{frame}

\begin{frame}[label={sec:org4d7698e}]{Tipos de Respuesta}
\begin{itemize}
\item Tipo de respuesta determinado por relación entre \(R\) y \(L\), \(C\) (disipación y almacenamiento).
\item Resistencia crítica (\(\alpha = \omega_0\), \(\xi = 1\)):
\end{itemize}

\[
  R_{cr} = 2\sqrt{\frac{L}{C}}
\]

\begin{block}{Tipos}
\begin{itemize}
\item \(R > R_{cr}\), \(\alpha > \omega_0\), \(\xi > 1\): \alert{sobreamortiguado}
\item \(R = R_{cr}\),  \(\alpha = \omega_0\), \(\xi = 1\): \alert{amortiguamiento crítico}
\item \(R < R_{cr}\),  \(\alpha < \omega_0\), \(\xi < 1\): \alert{subamortiguado}
\end{itemize}
\end{block}
\end{frame}

\begin{frame}[label={sec:orgf723a47}]{Circuito Sobreamortiguado (\(\alpha > \omega_0\))}
\[
  \boxed{i_n(t) = A_1 e^{s_1 t} + A_2 e^{s_2 t}}
\]
\begin{center}
\includegraphics[height=0.65\textheight]{../figs/Sobreamortiguado_HKD.pdf}
\end{center}
\end{frame}

\begin{frame}[label={sec:org9e205a5}]{Amortiguamiento Crítico (\(\alpha = \omega_0\))}
\[
  \boxed{i_n(t) = (A_1 + A_2 t) e^{s t} }
\]
\begin{center}
\includegraphics[height=0.65\textheight]{../figs/AmortiguamientoCritico_HKD.pdf}
\end{center}
\end{frame}


\begin{frame}[label={sec:org8af113b}]{Circuito Subamortiguado (\(\alpha < \omega\))}
\[
  \boxed{i_n(t) = (B_1\cos(\omega_d t) + B_2\sin(\omega_d t)) e^{-\alpha t}}
\]
\begin{center}
\includegraphics[height=0.65\textheight]{../figs/Subamortiguado_AS.pdf}
\end{center}
\end{frame}

\begin{frame}[label={sec:orga4b6666}]{Valores Importantes}
\begin{center}
\includegraphics[height=0.6\textheight]{../figs/RespuestaEscalon_SegundoOrden.png}
\end{center}

\begin{itemize}
\item \alert{Tiempo de Subida}: tiempo para subir de 10\% al 90\% del valor en régimen permanente.

\item \alert{Tiempo de Establecimiento}: tiempo para que la diferencia entre la respuesta y el régimen permanente permanezca dentro de una banda del 1\%.
\end{itemize}
\end{frame}

\begin{frame}[label={sec:org82c9ca3}]{Valores Importantes}
\begin{center}
\includegraphics[height=0.6\textheight]{../figs/RespuestaEscalon_SegundoOrden.png}
\end{center}

\begin{itemize}
\item \alert{Valor máximo} y \alert{Tiempo del Valor Máximo}.

\item \alert{Sobretensión}: porcentaje del valor máximo respecto del régimen permanente.
\end{itemize}
\end{frame}


\begin{frame}[label={sec:orgc2b798b}]{Condiciones Iniciales}
\begin{block}{Dos constantes a determinar}
Son necesarias dos tipos de condiciones iniciales:

\begin{align*}
  i_L(0^+) &= i_L(0^-)\\
  u_L(t) = L \cdot \diff{i_L(t)}{t} \longrightarrow   \diff{i_L(t)}{t}[t = 0^+] &= \frac{1}{L} u_L(0^+)
\end{align*}
\end{block}

\begin{block}{Derivadas en el origen}
Para obtener valores de las derivadas en el origen hay que resolver el circuito en \(t = 0^+\) empleando las condiciones de continuidad.
\end{block}
\end{frame}

\begin{frame}[label={sec:orgcff63f8}]{Derivadas en \(t = 0^+\)}
\begin{center}
\includegraphics[height=0.35\textheight]{../figs/transitorio_circuitoRLC_serie_t0+.pdf}
\end{center}

\[
  \diff{i_L(t)}{t}[t = 0^+] = \frac{1}{L} u_L(0^+)
\]
\begin{align*}
  u_L(0^+) &= -u_R(0^+) - u_c(0^+)\\
  u_R(0^+) &= R i_L(0^+)
\end{align*}
\[
\boxed{\diff{i_L(t)}{t}[t = 0^+] = - \frac{1}{L}\left(R i_L(0^+) + u_c(0^+)\right)}
\]
\end{frame}

\begin{frame}[label={sec:org9163546}]{Respuesta Completa}
Las condiciones iniciales deben evaluarse teniendo en cuenta la respuesta forzada (si existe).
\begin{align*}
  i_L(0^+) &= i_n(0^+) + i_{\infty}(0^+)\\
  \diff{i_L}{t}[t = 0^+] &= \diff{i_n}{t}[t = 0^+] + \diff{i_{\infty}}{t}[t = 0^+]  
\end{align*}
\end{frame}

\begin{frame}[label={sec:org629af6e}]{Ejemplo de Respuesta Completa}
Circuito RLC serie sobreamortiguado con generador de tensión DC funcionando en \(t > 0\). 

\begin{block}{Respuesta Completa}
\[
  i_L(t) = I_{\infty} + A_1 e^{s_1 t} + A_2 e^{s_2 t}
\]
\end{block}

\begin{block}{Condiciones Iniciales}
\begin{align*}
i_L(0^+) &= I_\infty + A_1 + A_2\\
\diff{i_L(t)}{t}[t = 0^+] &= 0 + A_1 s_1 + A_2 s_2
\end{align*}
\end{block}
\end{frame}


\subsection{Circuito RLC paralelo}
\label{sec:org1e69f60}
\begin{frame}[label={sec:org350f3a3}]{Circuito básico}
\begin{center}
\includegraphics[width=.9\linewidth]{../figs/transitorio_circuitoRLC_paralelo.pdf}
\end{center}
\end{frame}

\begin{frame}[label={sec:orgcc00b8f}]{Respuesta natural (t > 0)}
\begin{center}
\includegraphics[height=0.45\textheight]{../figs/transitorio_circuitoRLC_paralelo_t0+.pdf}
\end{center}

\[
  Gu(t) + C\diff{u(t)}{t} + \frac{1}{L}\int_{-\infty}^t u(t') \mathrm{d}t' = 0
\]

\[
  \diff[2]{u}{t} + \frac{G}{C} \diff{u}{t} + \frac{1}{LC} u = 0
\]
\end{frame}

\begin{frame}[label={sec:org1df34bd}]{Solución}
\begin{block}{Ecuación diferencial}
\[
  \diff[2]{u}{t} + \frac{G}{C} \diff{u}{t} + \frac{1}{LC} u = 0
\]
\[
  u_n(t) = A_1 e^{s_1 t} + A_2 e^{s_2 t}
\]
\end{block}

\begin{block}{Ecuación característica}
\[
s^2 + \frac{G}{C} s + \frac{1}{LC} = 0  
\]
\[
  s_{1,2} = -\frac{G}{2C} \pm \sqrt{\left(\frac{G}{2C}\right)^2 - \frac{1}{LC}}
\]
\end{block}
\end{frame}


\begin{frame}[label={sec:orgc87f77f}]{Parámetros}
\begin{block}{Ecuación característica}
\begin{columns}
\begin{column}{.5\columnwidth}
\begin{align*}
s^2 + \frac{G}{C} s + \frac{1}{LC} &= 0\\
s^2 + 2\alpha s + \omega_0^2 &= 0  
\end{align*}

\[
  s_{1,2} = -\alpha \pm \sqrt{\alpha^2 - \omega_0^2}
\]

\[
  u_n(t) = A_1 e^{s_1 t} + A_2 e^{s_2 t}
\]
\end{column}
\begin{column}{.5\columnwidth}
\begin{align*}
  \alpha &= \frac{G}{2C}\\
  \omega_0 &= \frac{1}{\sqrt{LC}}\\
  \omega_d &= \sqrt{\omega_0^2 - \alpha^2}\\
  \xi &= \frac{\alpha}{\omega_0}
\end{align*}
\end{column}
\end{columns}
\end{block}
\end{frame}

\begin{frame}[label={sec:org2b55428}]{Tipos de Respuesta}
\begin{itemize}
\item Tipo de respuesta determinado por relación entre \(G\) y \(L\), \(C\) (disipación y almacenamiento).
\item Conductancia crítica (\(\alpha = \omega_0\), \(\xi = 1\)):
\end{itemize}

\[
  G_{cr} = 2\sqrt{\frac{C}{L}}
\]

\begin{block}{Tipos}
\begin{itemize}
\item \(G > G_{cr}\), \(\alpha > \omega_0\), \(\xi > 1\): \alert{sobreamortiguado}
\item \(G = G_{cr}\),  \(\alpha = \omega_0\), \(\xi = 1\): \alert{amortiguamiento crítico}
\item \(G < G_{cr}\),  \(\alpha < \omega_0\), \(\xi < 1\): \alert{subamortiguado}
\end{itemize}
\end{block}
\end{frame}

\begin{frame}[label={sec:org2d32702}]{Tipos de Respuesta}
\begin{itemize}
\item Circuito Sobreamortiguado (\(\alpha > \omega_0\))
\end{itemize}
\[
  \boxed{u_n(t) = A_1 e^{s_1 t} + A_2 e^{s_2 t}}
\]
\begin{itemize}
\item Amortiguamiento Crítico (\(\alpha = \omega_0\))
\end{itemize}
\[
  \boxed{u_n(t) = (A_1 + A_2 t) e^{s t} }
\]

\begin{itemize}
\item Circuito Subamortiguado (\(\alpha < \omega\))
\end{itemize}
\[
  \boxed{u_n(t) = (B_1\cos(\omega_d t) + B_2\sin(\omega_d t)) e^{-\alpha t}}
\]
\end{frame}


\begin{frame}[label={sec:org50bcb70}]{Condiciones Iniciales}
\begin{block}{Dos constantes a determinar}
Son necesarias dos tipos de condiciones iniciales:


\begin{align*}
  u_C(0^+) &= u_C(0^-)\\
  i_c(t) = C \cdot \diff{u_c(t)}{t} \longrightarrow \diff{u_c(t)}{t}[t = 0^+] &= \frac{1}{C}i_C(0^+)
\end{align*}
\end{block}

\begin{block}{Derivadas en el origen}
Para obtener valores de las derivadas en el origen hay que resolver el circuito en \(t = 0^+\) empleando las condiciones de continuidad.
\end{block}
\end{frame}

\begin{frame}[label={sec:orgaf93375}]{Derivadas en \(t = 0^+\): ejemplo RLC paralelo}
\begin{center}
\includegraphics[height=0.25\textheight]{../figs/transitorio_circuitoRLC_paralelo_t0+.pdf}
\end{center}

\[
  \diff{u_c(t)}{t}[t = 0^+] = \frac{1}{C}i_C(0^+)
\]

\begin{align*}
  i_C(0^+) &= -i_R(0^+) - i_L(0^+)\\
  i_R(0^+) &= \frac{1}{R} u_C(0^+)
\end{align*}
\[
  \boxed{\diff{u_c(t)}{t}[t = 0^+] = - \frac{1}{C} \left( \frac{1}{R} u_C(0^+) +  i_L(0^+)\right)}
\]
\end{frame}


\begin{frame}[label={sec:org9770cba}]{Respuesta Completa}
Las condiciones iniciales deben evaluarse teniendo en cuenta la respuesta forzada (si existe).
\begin{align*}
  u_C(0^+) &= u_n(0^+) + u_{\infty}(0^+)\\
  \diff{u_C(t)}{t}[t = 0^+] &= \diff{u_n(t)}{t}[t = 0^+] + \diff{u_{\infty}(t)}{t}[t = 0^+]  
\end{align*}
\end{frame}

\begin{frame}[label={sec:orge2d0095}]{Ejemplo de Respuesta Completa}
Circuito RLC paralelo sobreamortiguado con generador de corriente DC funcionando en \(t > 0\). 

\begin{block}{Respuesta Completa}
\[
  u_c(t) = U_{\infty} + A_1 e^{s_1 t} + A_2 e^{s_2 t}
\]
\end{block}

\begin{block}{Condiciones Iniciales}
\begin{align*}
u_c(0^+) &= U_\infty + A_1 + A_2\\
\diff{u_C(t)}{t}[t = 0^+] &= 0 + A_1 s_1 + A_2 s_2
\end{align*}
\end{block}
\end{frame}
\end{document}