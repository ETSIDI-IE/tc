\documentclass[aspectratio=169, usenames,svgnames,dvipsnames]{beamer}
\usepackage[utf8]{inputenc}
\usepackage[T1]{fontenc}
\usepackage{graphicx}
\usepackage{grffile}
\usepackage{longtable}
\usepackage{wrapfig}
\usepackage{rotating}
\usepackage[normalem]{ulem}
\usepackage{amsmath}
\usepackage{textcomp}
\usepackage{amssymb}
\usepackage{capt-of}
\usepackage{hyperref}
\usepackage{color}
\usepackage{listings}
\usepackage{mathpazo}
\usepackage{gensymb}
\usepackage{amsmath}
\usepackage{diffcoeff}
\usepackage{steinmetz}
\usepackage{mathtools}
\bibliographystyle{plain}
\usepackage{siunitx}
\sisetup{output-decimal-marker={,}}
\DeclareSIUnit{\watthour}{Wh}
\hypersetup{colorlinks=true, linkcolor=Blue, urlcolor=Blue}
\renewcommand{\thefootnote}{\fnsymbol{footnote}}
\newcommand{\laplace}[1]{\mathbf{#1}(\mathbf{s})}
\newcommand{\slp}{\mathbf{s}}
\newcommand{\fasor}[1]{\mathbf{#1}(\omega)}
\newcommand{\atan}{\mathrm{atan}}
\parskip=5pt
\usetheme{Boadilla}
\usecolortheme{rose}
\usefonttheme{serif}
%\author{Autor: \hspace{2mm} Luis Badesa Bernardo}
\date{}
\title{\LARGE Información práctica sobre la asignatura \vspace{5mm}}
\subtitle{Teoría de Circuitos}
\setbeamercolor{alerted text}{fg=blue!50!black} \setbeamerfont{alerted text}{series=\bfseries}
\AtBeginSubsection[]{\begin{frame}[plain]\tableofcontents[currentsubsection,sectionstyle=show/shaded,subsectionstyle=show/shaded/hide]\end{frame}}
\AtBeginSection[]{\begin{frame}[plain]\tableofcontents[currentsection,hideallsubsections]\end{frame}}
\beamertemplatenavigationsymbolsempty
\setbeamertemplate{footline}[frame number]
\setbeamertemplate{itemize items}[triangle]
\setbeamertemplate{enumerate items}[circle]
\setbeamertemplate{section in toc}[circle]
\setbeamertemplate{subsection in toc}[circle]
\hypersetup{
 pdfauthor={},
 pdftitle={Información práctica sobre la asignatura},
 pdfkeywords={},
 pdfsubject={},
 pdfcreator={}, 
 pdflang={Spanish}}
\begin{document}

\maketitle

\section*{Datos del profesor}

\begin{frame}{Datos del profesor}
    
    Luis Badesa Bernardo
    \vspace{2mm}
        \begin{itemize}
        \item \href{mailto:luis.badesa@upm.es}{luis.badesa@upm.es}
        \vspace{1mm}
        \item Web: \url{https://badber.github.io/}
        \vspace{1mm}
        \item \alert{Investigación}: mercados eléctricos y redes dominadas por renovables
        \vspace{1mm}
        \item \alert{Despacho} A-139-5 (al lado del laboratorio)
        \vspace{1mm}
        \item Ver horario de tutorías 
        \href{http://programas.etsidi.upm.es/SOA/tutorias/}{aquí}            
    \end{itemize}
    \vspace{6mm}
    Horario de clases, \href{https://www.etsidi.upm.es/Estudiantes/AgendaAcademica/AAHorarioClases}{aquí}
    
    %\vspace{5mm}
    %
    %\center{\large \alert{Encuesta en Moodle para \underline{elegir grupo de laboratorio}} }

\end{frame}

%%%%%%%%%%%%%%%%%%%%%%%%%%%%

\section*{Temario}

\begin{frame}{Temario}
    \begin{enumerate}
        \item Conceptos básicos. Circuitos de corriente continua
        \vspace{1mm}
        \item Corriente alterna monofásica
        \vspace{1mm}
        \item Sistemas trifásicos
        \vspace{1mm}
        \item Introducción al régimen transitorio de los circuitos
    \end{enumerate}
    \hspace{7mm} $+$
    
    \hspace{7mm} Laboratorio (6 prácticas)

    \vspace{1mm}

    \noindent\rule{\textwidth}{0.5pt}
    
    \vspace{3mm}
    
    \alert{Material de la asignatura}
    
    (diapositivas, libro, problemas propuestos, exámenes anteriores, bibliografía): 
    
    %\vspace{1mm}
    \begin{itemize}
        \addtolength{\itemsep}{10mm}
        \item Disponible en Moodle
    \end{itemize}

\end{frame}

%%%%%%%%%%%%%%%%%%%%

\section*{Convocatoria ordinaria}

\begin{frame}{Convocatoria ordinaria}

    \begin{minipage}[t]{0.34\linewidth}
    \alert{Evaluación}
    \vspace{3mm}
    \begin{itemize}
    \item \alert{Progresiva}, exámenes:
        \begin{itemize}
        \item Tema 1
        \item Tema 2
        \item Tema 3
        \item Tema 4 (con Global)
        \item Laboratorio
        \end{itemize}
    \end{itemize}
    (fechas de examen en \href{https://moodle.upm.es/titulaciones/oficiales/course/view.php?id=1218\#section-1}{Moodle})
    \begin{itemize}
    \item \alert{Global}, exámenes:
        \begin{itemize}
        \item Global
        \item Laboratorio
        \end{itemize}
    \end{itemize}
    \end{minipage}
    %
    \hfill%
    %
    \begin{minipage}[t]{0.65\linewidth}
    %\pause
    \vspace{13.2mm}
        \begin{itemize}
        %\item Cada examen parcial (T1–T4) se califica de 0 a 10
            %\begin{itemize}
            %\item {\normalsize Si `nota $\geq$ 5', ese \alert{tema queda liberado} del global}
            %\end{itemize}
        \vspace{33mm}
        
        \item En el examen global, existe la \alert{opción de subir nota}: para cada tema, cuenta la nota más alta entre parcial y global
        \end{itemize}    
    \end{minipage}

\end{frame}

%%%%%%%%%%%%%%%%%%%%

\subsection*{Laboratorio}

\begin{frame}{Laboratorio}

    %\alert{Evaluación del laboratorio}
    %\vspace{1mm}

    \vspace{3mm}
    
    \begin{itemize}
        \item \alert{Prácticas obligatorias} (asistencia $+$ entrega de informes)
        \vspace{1mm}
        \begin{itemize}
            \item 
            {\normalsize Laboratorio A-139-L1}
            \vspace{2mm}
            \item {\normalsize Las prácticas comienzan un par de semanas después del inicio del cuatrimestre (se avisará por Moodle)}
            \vspace{2mm}
            \item {\normalsize Se ofertarán grupos por Moodle para elegir horario}
            \end{itemize}
        
        \vspace{2.5mm}
        \item \alert{Examen de laboratorio} (mismo día que el global, ver fecha \href{https://www.etsidi.upm.es/Estudiantes/AgendaAcademica/AAFechaExamenes}{aquí}):
        \vspace{1mm}
            \begin{itemize}           
            \item {\normalsize Si \alert{asistencia a todas las prácticas} $\rightarrow$ \underline{examen escrito (tipo test)}}
            \vspace{2mm}
            \item {\normalsize Si \alert{se han cursado TODAS las prácticas} en años anteriores, con calificación 
            
            ``No apto/a'', y no se quiere volver a cursarlas $\rightarrow$ \underline{examen escrito $+$ práctico}}
            \vspace{2mm}
            \item {\normalsize Calificación: ``\alert{Apto/a}'' o ``\alert{No apto/a}'' (no hay calificación numérica)}
            \end{itemize}
            \vspace{-4mm}
        \end{itemize}    
    
    \noindent\rule{\textwidth}{0.5pt}
    
    \vspace{-1mm}
    
    \centering{\alert{Si ya se superó} el Laboratorio en una convocatoria anterior, \\    
    \alert{NO hay que repetir} prácticas ni examen}

\end{frame}

%%%%%%%%%%%%%%%%%%%%

\section*{Convocatoria extraordinaria}

\begin{frame}{Convocatoria extraordinaria (junio/julio)}

    \vspace{-23mm}
    \begin{itemize}
        \item Fecha del examen en la \href{https://www.etsidi.upm.es/Estudiantes/AgendaAcademica/AAFechaExamenes}{web} de la ETSIDI
        \item \alert{Mismas condiciones} que para el examen global de la convocatoria ordinaria

        \vspace{6mm}

        \item La nota obtenida en los parciales de evaluación progresiva \alert{NO se conserva} para la convocatoria extraordinaria
    \end{itemize}

\end{frame}
        
%%%%%%%%%%%%%%%%%%%%%%

\section*{Condiciones para aprobar}

\begin{frame}{Condiciones para aprobar}
    \begin{itemize}
    \item \alert{Condiciones para aprobar Teoría y Problemas}  (TyP):

    \vspace{1mm}
    Ejercicios agrupados por temas: cada tema se califica de 0 a 10
        \vspace{1mm}
        \begin{itemize}
        \item {\normalsize Obtener una calificación $\geq$ 5 en, al menos, 2 de los 4 temas}
        \vspace{1mm}
        \item {\normalsize Obtener una calificación promedio $\geq$ 5} 
            \vspace{1mm}
            \begin{itemize}
            \item {\normalsize Si se cumple esta, pero no la condición anterior $\rightarrow$ nota de TyP $=$ 4,5}
            \end{itemize}
        \end{itemize}
    \vspace{3mm}
    \item \alert{Condiciones para aprobar la asignatura} (i.e.,~TyP $+$ Laboratorio):
        \vspace{1mm}
        \begin{itemize}
        \item {\normalsize \textbf{if} $\;$ nota\_TyP $\geq$ 5 $\;$ \textbf{and} $\;$ laboratorio $=$ Apto $\; \rightarrow \;$ nota\_Acta = nota\_TyP} (\textcolor{green}{aprobada})
        
        \vspace{1mm}
        \item {\normalsize \textbf{else if} $\;$ nota\_TyP $\geq$ 5 $\;$ \textbf{and} $\;$ laboratorio $=$ No\_Apto $\; \rightarrow \;$ nota\_Acta $=$ 4,5}
        \vspace{1mm}
        \item {\normalsize \textbf{else} $\; \rightarrow \;$ nota\_Acta $=$ nota\_TyP}
        \end{itemize}
    \vspace{6mm}
    Si se aprueba TyP o Laboratorio, \alert{se guarda la califación} para convocatorias futuras
    \end{itemize}

\end{frame}


\end{document}