\chapter{Variables}
\label{cha:variables}
El análisis de un circuito eléctrico lineal consiste en determinar tres variables principales: tensión, corriente, y potencia.

\section{Tensión Eléctrica}
\label{sec:tension}

La tensión o diferencia de potencial entre dos puntos A y B es el trabajo realizado por el campo eléctrico al desplazar una carga unitaria entre esos puntos. 


\begin{equation}
u_{AB} = \frac{dW_{e}}{dq}\label{eq:uab}
\end{equation}

Dado que el campo eléctrico es conservativo, la diferencia de potencial entre A y B no depende de la trayectoria seguida para realizar el desplazamiento, sino únicamente del potencial existente en cada uno de los puntos:

\begin{equation}
  \label{eq:potencial}
  u_{AB} = v_A - v_B
\end{equation}

Por tanto, aunque la trayectoria no sea relevante, siempre hay que tener en cuenta el sentido del desplazamiento. Así, si el movimiento se produce desde B hasta A obtenemos el signo contrario al anterior resultado:

\begin{equation}
  \label{eq:signo_tension}
  u_{BA} = v_B - v_A = - u_{AB} 
\end{equation}


%%% Local Variables:
%%% mode: latex
%%% TeX-master: "TC"
%%% End:
