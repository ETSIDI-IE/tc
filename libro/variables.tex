\chapter{Variables}
\label{cha:variables}
El análisis de un circuito eléctrico lineal permite determinar tres variables principales: tensión, corriente, y potencia.

\section{Tensión Eléctrica}
\label{sec:tension}

El potencial eléctrico $v(t)$ en un punto es la energía potencial que tiene una carga unitaria en ese punto debida al campo eléctrico. La tensión o diferencia de potencial entre dos puntos A y B $u_{AB}(t)$ es, por tanto, el trabajo realizado por el campo eléctrico al desplazar una carga unitaria entre esos puntos. 


\begin{equation}
  \label{eq:uab}
  u_{AB}(t) = v_A(t) - v_B(t) = \frac{dW_{e}}{dq}
\end{equation}

Dado que el campo eléctrico es conservativo, la diferencia de potencial entre A y B no depende de la trayectoria seguida para realizar el desplazamiento, sino únicamente del potencial existente en cada uno de los puntos. Por tanto, aunque la trayectoria no sea relevante para el cálculo de la tensión, siempre hay que tener en cuenta el sentido del desplazamiento. Así, si el movimiento se produce desde B hasta A obtenemos el signo contrario al anterior resultado:

%%TODO: diagrama de tensión entre puntos con flecha

\begin{equation}
  \label{eq:signo_tension}
  u_{BA} = v_B - v_A = - u_{AB} 
\end{equation}

La unidad de la tensión eléctrica es el voltio (V).
\section{Corriente Eléctrica}
\label{sec:corriente}

Se define la intensidad de la corriente eléctrica como la variación de la carga $q(t)$ que atraviesa la sección transversal de un conductor por unidad de tiempo:
%%TODO: dibujo de sección de conductor con flujo de cargas
\begin{equation}
  \label{eq:corriente}
  i(t)=\frac{dq(t)}{dt}
\end{equation}

La corriente eléctrica se produce por el movimiento de los electrones libres que fluyen por el conductor. Sin embargo, por razones históricas, el convenio que se se emplea considera como sentido de la corriente el debido al movimiento de las cargas positivas.
%% TODO: diagrama de sentido de la corriente

La unidad de la corriente es el amperio (A).
\section{Potencia Eléctrica}
\label{sec:potencia}

La potencia eléctrica es la variación del trabajo del campo eléctrico por unidad de tiempo:

\begin{equation}
  \label{eq:potencia_general}
  p(t)=\frac{dW_{e}}{dt}
\end{equation}

Esta definición genérica puede relacionarse con las anteriores variables gracias a las ecuaciones \ref{eq:uab} y \ref{eq:corriente}:

\begin{align}
  \label{eq:potencia_vi}
  p(t) &= \frac{dW_e}{dq} \cdot \frac{dq(t)}{dt}\\
       &= v(t)\cdot i(t)
\end{align}

La unidad de la potencia eléctrica es el vatio (W).

Para determinar el signo de la potencia eléctrica hay que tener en consideración los signos de las variables de las que depende, la tensión y la corriente. Cuando las flechas de ambas variables tienen el mismo sentido la potencia eléctrica es positiva, y cuando las flechas tienen sentidos opuestos la potencia eléctrica es negativa. Es habitual interpretar este resultado en términos de potencia absorbida o potencia entregada. Así, un circuito receptor absorbe potencia y la corriente \emph{entra} por el terminal de mayor potencial, mientras que un circuito generador entrega potencia y la corriente \emph{sale} por el terminal de mayor potencial.

Supongamos el circuito de la figura, en el que la corriente eléctrica entra por el terminal A y sale por B, siendo $v_A > v_B$. En estas condiciones, $u_{AB} > 0$, $i > 0$, y consecuentemente $p > 0$.  El circuito conectado entre A y B está absorbiendo potencia eléctrica, dado que las cargas que entran en el circuito pierden potencial.
%%TODO: diagrama de dipolo receptor y dipolo generador
%%% Local Variables:
%%% mode: latex
%%% TeX-master: "TC"
%%% End:
