

\chapter{Corriente alterna monofásica}\label{chap:ca_mono}
 	
\section{Formas de onda periódicas}
 	
En los circuitos eléctricos, las funciones de excitación y respuesta
son tensiones e intensidades que varían con el tiempo:
\begin{align*}
  u&=u(t)\\
  i&=i(t)
\end{align*}
Estas funciones pueden representarse de forma gráfica o analítica. En
ambos casos, esa relación funcional se conoce mediante el nombre de
\textbf{forma de onda}. Las formas de onda pueden clasificarse según
(de manera análoga a lo indicado en la Sección~\ref{sec:cc-ca}):
\begin{itemize}
\item \textbf{Signo de la magnitud}:
  \begin{itemize}
  \item \textbf{Unidireccionales}: la magnitud que la representa
    siempre tiene una única polaridad (signo constante, aunque el
    valor puede ser constante o variable)
  \item \textbf{Bidireccionales}: la magnitud toma valores positivos y
    negativos (signo variable con el tiempo)
  \end{itemize}
\item \textbf{Repetición del valor de la magnitud}:
  \begin{itemize}
  \item \textbf{Periódicas}: el valor de la magnitud se repite de
    forma regular
  \item \textbf{No periódicas}: el valor de la magnitud varía de forma
    arbitraria con el tiempo
  \end{itemize}
\end{itemize}

Cuando se trabaje con \textbf{corriente alterna}, siempre se usarán
\textbf{funciones de onda periódicas}, generalmente sinusoidales. Las
formas de onda periódicas son aquellas que se repiten a intervalos
iguales de tiempo y en el mismo orden, siguiendo la expresión:
\begin{equation*}
  y(t)=y(t+T)=y(t+n\cdot T)
\end{equation*}
	
Existen una serie de definiciones y valores de interés para las ondas
periódicas:
\begin{itemize}
\item \textbf{Período ($T$)}: intervalo de tiempo mínimo a partir del
  cual se repite la forma de onda [s]
\item \textbf{Frecuencia ($f$)}: número de veces que se repite la onda
  por unidad de tiempo [Hz]:
  \begin{equation*}
    f = \dfrac{1}{T}
  \end{equation*}
  \begin{remark}
    La unidad [Hz] se escribe en mayúsculas en honor a Heinrich Rudolf
    Hertz, físico alemán del siglo XIX que descubrió el efecto
    fotoeléctrico, la propagación de las ondas electromagnéticas y las
    formas para producirlas y detectarlas.
  \end{remark}
\item \textbf{Valor instantáneo}: valor $y(t)$ que toma la forma de
  onda en un instante de tiempo dado
\item \textbf{Valores de pico ($Y_{max}$, $Y_{min}$)}: valores máximo
  y mínimo que toma la forma de onda en un periodo:
  \begin{equation*}
    Y_{max} = \max(f(t)); \qquad Y_{min} = \min(f(t))
  \end{equation*}
\item \textbf{Valor pico a pico ($Y_{PP}$)}: se corresponde con la
  diferencia (en valor absoluto) entre los valores de pico
  considerados con signo:
  \begin{equation*}
    Y_{PP}=|Y_{max} - Y_{min}|
  \end{equation*}
\item \textbf{Valor medio ($Y_m$)}: en un intervalo ($t_1,t_2$),
  corresponde con la media aritmética de los valores instantáneos que
  toma la función en dicho intervalo:
  \begin{equation*}
    Y_m=\dfrac{1}{t_2-t_1}\cdot\int_{t_1}^{t_2}y(t)\, dt
  \end{equation*}
  En una onda periódica, se calcula para un intervalo de tiempo igual
  a un periodo:
  \begin{equation}\label{eq:valor_medio}
    \boxed{Y_m=\frac{1}{T}\int_{a}^{a+T}y(t)\, dt}
  \end{equation}
  \begin{remark}
    En caso de que el valor medio sea nulo en un periodo, el cálculo
    se realiza en un semi-periodo ($T/2$) o en un cuarto de periodo
    ($T/4$)
  \end{remark}
\item \textbf{Valor eficaz ($Y_{ef}$)}: es la raíz cuadrada de la
  media de los cuadrados de los valores que toma la función en un
  intervalo:
  \begin{equation*}
    Y_{ef}=\sqrt{\dfrac{1}{t_2-t_1}\cdot\int_{t_1}^{t_2}y^2(t)\, dt}
  \end{equation*}
  Si es periódica:
  \begin{equation}\label{eq:valor_eficaz}
    \boxed{Y_{ef} = \sqrt{\frac{1}{T}\cdot\int_{a}^{a+T}y^{2}(t)\, dt}}
  \end{equation}
\item \textbf{Factor de amplitud ($FA$)}: es el cociente entre el
  valor máximo y el valor eficaz de una onda:
  \begin{equation*}
    FA = \dfrac{Y_{max}}{Y_{ef}}
  \end{equation*}
\item \textbf{Factor de forma ($FF$)}: es el cociente entre el valor
  eficaz y el valor medio:
  \begin{equation*}
    FF = \dfrac{Y_{ef}}{Y_{m}}
  \end{equation*}
  \begin{remark}
    Si el valor medio fuese nulo en un período, se toma el de un
    semiperíodo
  \end{remark}
\end{itemize}
	
\begin{example}\label{ex.forma_onda}
  \textbf{Hallar el valor medio y eficaz de la onda periódica de la
    Figura~\ref{fig:forma_onda}.}
  \begin{figure}[H]
    \centering \includegraphics{../figs/ejemplo_forma_onda.pdf}
    \caption{Ejemplo~\ref{ex.forma_onda}}
    \label{fig:forma_onda}
  \end{figure}
	    
  La onda es una función periódica, de periodo $T=4$ s, para la cual,
  en el intervalo $[0,4]$ s, la función se expresa como:
  \begin{equation*}
    u(t)=\dfrac{100}{4}\,t=25\,t\;\si{\volt} \quad (0\leq t\leq 4\,\text{s})
  \end{equation*}
	    
  Se utiliza la expresión~\eqref{eq:valor_medio} para determinar el
  valor medio:
  \begin{equation*}
    U_m=\dfrac{1}{T}\int_0^T u(t)\,dt=\dfrac{1}{4}\int_0^4 (25\,t)\,dt=\dfrac{1}{4}\,\left[25\,\dfrac{t^2}{2} \right]_0^4={50}\;\si{\volt}
  \end{equation*}
	    
  Se utiliza la ecuación~\eqref{eq:valor_eficaz} para el valor eficaz:
  \begin{align*}
    U_{ef}&=\sqrt{\frac{1}{T}\cdot\int_{0}^{T}u(t)^{2}\, dt}=\sqrt{\frac{1}{4}\cdot\int_{0}^{4}(25\,t)^{2}\, dt}=\sqrt{\frac{1}{4}\cdot\int_{0}^{4}(625\,t^2)\, dt}=\\
          &=\sqrt{\frac{1}{4}\, 625\,\left[\dfrac{t^3}{3}\right]_0^4}={57.74}\;\si{\volt}
  \end{align*}
\end{example}
	

\subsection{Función sinusoidal}\label{sec:sinusoidal}
Dentro de las ondas periódicas, las \textbf{ondas sinusoidales} son de
gran importancia en el campo de la electricidad. Estas formas de onda
vienen determinadas por:
\begin{equation}\label{eq:y_senoidal}
  \boxed{y(t)=Y_{max}\cdot\sin(\omega t+\theta)} 
\end{equation}
siendo $Y_{max}$ el valor máximo de la onda, $\omega$ la pulsación o
frecuencia angular [rad/s] ($\omega=2\cdot\pi\cdot f$, siendo $f$ la
frecuencia de la onda [Hz]) y $\theta$ la fase [rad]. Un ejemplo de
este tipo de forma de onda se muestra en la Figura~\ref{fig:sin}.
\begin{figure}[H]
  \centering \includegraphics[width=.9\linewidth]{../figs/sin.pdf}
  \caption{Ejemplo de forma de onda sinusoidal}
  \label{fig:sin}
\end{figure}
	
La fase representa el argumento de la onda para $t=0$. Tomando una
onda como referencia, si la fase de otra onda es $0^\circ$, se dice
que la onda está \textbf{en fase} con la onda de referencia; si la
fase es positiva ($+$) respecto a la de referencia, se dice que la
onda está \textbf{en adelanto}; y si la fase es negativa ($-$)
respecto a la de referencia, se dice que la onda está \textbf{en
  retraso}. Así, en la Figura~\ref{fig:desfase}, considerando como
referencia la onda de color negro (que tiene una fase $\theta=0$), la
{\color{blue} onda azul} está en retraso, mientras que la {\color{red}
  onda roja} está en adelanto. En caso de que el desfase entre dos
ondas sea de 90$^\circ$, se dice que están \textbf{en cuadratura}: el
paso por 0 de una onda, coincide con el paso por el máximo/mínimo de
la otra.
\begin{figure}[H]
  \centering \includegraphics[width=.9\linewidth]{../figs/desfase.pdf}
  \caption{Fases entre ondas sinusoidales}
  \label{fig:desfase}
\end{figure}
	
	
Las propiedades de las formas de onda senoidales que hacen que sea la
preferida para la generación de energía eléctrica a gran escala son
las siguientes:
\begin{enumerate}
\item Su forma básica se mantiene siempre, puesto que sus derivadas e
  integrales sucesivas son funciones sinusoidales $\rightarrow$ si la
  excitación es sinusoidal, las respuestas también lo son (pasado un
  corto periodo de tiempo transitorio)
\item La suma o resta de funciones senoidales de la misma frecuencia
  es otra función senoidal de la misma frecuencia
\end{enumerate}
Respecto al estudio de otras formas de onda, su interés reside en el
teorema de Fourier, dado que cualquier onda periódica no senoidal
puede suponerse formada por infinitas ondas senoidales de distinta
frecuencia.
	
Al ser un caso particular de una onda periódica, las definiciones y
valores de interés indicados previamente también son válidos. Por
simplicidad, se considera la función sinusoidal con fase inicial nula,
$y(t)=Y_{max}\cdot\sin(\omega t)$:
\begin{itemize}
\item \textbf{Valor pico a pico ($Y_{PP}$)}: es el doble de la
  amplitud:
  \begin{equation*}
    Y_{PP}=|Y_{max} - Y_{min}|=2\cdot Y_{max}
  \end{equation*}
\item \textbf{Valor medio ($Y_m$)}: en un periodo, el valor medio es
  0, puesto que el área positiva es igual al área
  negativa. Considerando entonces un semiperíodo:
  \begin{equation*}
    Y_m(T/2)=\frac{1}{T/2}\int_{0}^{T/2} Y_{max}\cdot \sin(\omega t)\, dt=\dfrac{2\cdot Y_{max}}{T\cdot\omega}\left[-\cos(\omega\cdot t)\right]_0^{T/2} =\dfrac{2\cdot Y_{max}}{\pi}\approx 0.637\cdot Y_{max}
  \end{equation*}
\item \textbf{Valor eficaz ($Y_{ef}$)}: para simplificar el cálculo,
  se hace en primer lugar el valor eficaz al cuadrado:
  \begin{equation*}
    Y_{ef}^2=\dfrac{1}{T}\cdot\int_{0}^{T}\left(Y_{max}\cdot\sin(\omega  t)\right)^{2}\,dt=\dfrac{Y_{max}^2}{T}\cdot\int_{0}^{T}\left(\sin(\omega t)\right)^{2}\,dt=\dfrac{Y_{max}^2}{2}
  \end{equation*}
  luego:
  \begin{equation*}
    Y_{ef}=\sqrt{Y_{ef}^2}=\dfrac{Y_{max}}{\sqrt{2}}\approx0.707\cdot Y_{max}
  \end{equation*}
\item \textbf{Factor de amplitud ($FA$)}:
  \begin{equation*}
    FA = \dfrac{Y_{max}}{Y_{ef}}=\dfrac{\cancel{Y_{max}}}{\frac{\cancel{Y_{max}}}{\sqrt{2}}}=\sqrt{2}\approx 1.414
  \end{equation*}
\item \textbf{Factor de forma ($FF$)}:
  \begin{equation*}
    FF = \dfrac{Y_{ef}}{Y_{m}} = \dfrac{\frac{\cancel{Y_{max}}}{\sqrt{2}}}{\frac{2\cdot \cancel{Y_{max}}}{\pi}}=\dfrac{\pi}{2\cdot\sqrt{2}}\approx 1.111
  \end{equation*}
\end{itemize}
	
\subsection{Cálculo fasorial}
	
Cuando se trabaje con corriente alterna, siempre se usarán funciones
de onda sinusoidales, todas ellas de la misma pulsación $\omega$. Por
tanto, las diferencias que habrá en dichas ondas serán, únicamente,
las amplitudes $Y_{max}$ y las fases $\theta$. Esto permite que, en
lugar de trabajar con formas de onda, se pueda trabajar con
\textbf{fasores}, números complejos que representan una tensión o
corriente sinusoidales.  La longitud del fasor (su módulo) es el
\textbf{valor eficaz} de la función sinusoidal, que se define como el
valor de la corriente alterna que consigue generar el mismo resultado
de tensión/corriente que si fuera en corriente continua, y se calcula
como el valor máximo entre $\sqrt{2}$ (como ya se justificó).

La posición del fasor, conocida como argumento, se determina en
cualquier instante de tiempo haciendo el producto $\omega t+\theta$,
pero la de más interés es el instante inicial, es decir, para $t=0$ en
la expresión~\eqref{eq:y_senoidal}:
\begin{equation*}
  y(t)=Y_{max}\cdot\sin(\cancelto{0}{\omega \cdot 0}+\theta)=Y_{max}\cdot\sin(\theta)
\end{equation*}
Por simplicidad, la fase $\theta$, al trabajar con fasores, puede
expresarse en grados [$^\circ$]. Por tanto, al fasor le corresponde
por módulo y argumento:
\begin{equation}
  \boxed{\overline{Y}=Y_{ef}\,\phase{\theta}}
\end{equation}
que es conocida como la \textbf{forma polar} del fasor.
	
\begin{example}
  \textbf{Expresar en modo de fasor las siguientes funciones
    sinusoidales:}
  \begin{align*}
    u(t)&=150\,\sqrt{2}\cdot \sin(500\cdot t+\frac{\pi}{4})\, V\\
    i(t) &= 3\,\sqrt{2}\cdot \sin(2000\cdot t+\frac{\pi}{6})\,A
  \end{align*}
		
  Para expresar los fasores, hay que utilizar los valores eficaces
  ($U_{ef}=150$ V; $I_{ef}=3$ A) y los desfases
  ($\frac{\pi}{4}=45^\circ$ para la $u(t)$ y $\frac{\pi}{6}=60^\circ$
  para la $i(t)$). Así:
  \begin{align*}
    \overline{U}&=150\phase{45^\circ} \,\si{\volt}\\
    \overline{I}&=3\phase{30^\circ}\,\si{\ampere}
  \end{align*}
\end{example}
	
Considerando la Figura~\ref{fig:fasor}, donde el \texttt{eje X}
representa la parte real del fasor y, el \texttt{eje Y}, la parte
imaginaria, la \textbf{forma binómica} (o rectangular) de un fasor, se
obtiene como:
\begin{equation}
  \boxed{\overline{Y} = Y_{ef}\cdot(\cos(\theta)+\mathrm{j}\cdot\sin(\theta))}
\end{equation}
\begin{figure}[H]
  \centering \includegraphics{../figs/fasor.pdf}
  \caption{Concepto de fasor}
  \label{fig:fasor}
\end{figure}
	
El empleo de estas notaciones permite operar con las funciones
sinusoidales del mismo modo que con vectores en el plano y números
complejos. En general, es habitual emplear la \textbf{forma binómica}
para sumar y restar, y la \textbf{forma polar} para multiplicar y
dividir. Debe tenerse en cuenta que para realizar estas operaciones es
necesario que las expresiones sean todas con la función \texttt{seno}
o con \texttt{coseno}. Caso contrario, habrá que expresar todas las
magnitudes respecto a la misma función, siguiendo la relación:
\begin{equation*}
  \cos(\beta)=\sin(\beta+90^\circ)
\end{equation*}
\begin{remark}
  Si se tiene un fasor en forma rectangular $a+\mathrm{j}\,b$, para
  transformarlo a polar se debe calcular su módulo $\sqrt{a^2+b^2}$ y
  argumento $\arctan(b/a)$.
\end{remark}
	
\begin{remark}
  Si se tiene un fasor en forma polar $r\phase{\alpha^\circ}$, para
  transformarlo a rectangular se debe calcular $a$ como
  $r\cdot \cos(\alpha^\circ)$ y $b$ como $r\cdot \sin(\alpha^\circ)$.
\end{remark}
	
\begin{remark}
  Se recuerda que el empleo de fasores solo es válido cuando todas las
  ondas tienen la misma pulsación $\omega$
\end{remark}
	
\vspace{4mm}
\begin{example}
  \textbf{Dados $\overline{U_1}=25\phase{145}^\circ$ V y
    $\overline{U_2}=11\phase{25^\circ}$ V, calcular la relación
    $\overline{U_1}/\overline{U_2}$ y la suma
    $\overline{U_1}+\overline{U_2}$.}
		
  Para hacer el cociente se utiliza la forma polar:
  \begin{equation*}
    \dfrac{\overline{U_1}}{\overline{U_2}}=\dfrac{25\phase{145^\circ}}{11\phase{25^\circ}}=\dfrac{25}{11}\phase{145^\circ-25^\circ}=2.27\phase{120^\circ}
  \end{equation*}
		
  Para hacer la suma, es necesario usar la forma binómica:
  \begin{align*}
    \overline{U_1}&=25\phase{145^\circ}=-20.48+\mathrm{j} 14.34 \si{\volt}\\
    \overline{U_2}&=11\phase{30^\circ}=9.52+\mathrm{j} 5.5 \si{\volt}
  \end{align*}
		
  La suma de ambas tensiones es:
  \begin{equation*}
    \overline{U_1}+\overline{U_2}=(-20.48+\mathrm{j}14.34)+(9.52+\mathrm{j}5.5)=-10.96+\mathrm{j}19.84\;\si{\volt}
  \end{equation*}
\end{example}
	
\vspace{4mm}
\begin{example}
  \textbf{Sabiendo que $i_1(t)=2\cdot\sqrt{2}\cdot \sin(600\cdot t)$
    A;
    $i_2(t)=4\cdot\sqrt{2}\cdot \sin\left(600\cdot
      t+\frac{\pi}{2}\right)$ A; e
    $i_3(t)=10\cdot\sqrt{2}\cdot \cos(600\cdot t-\pi)$ A, determinar
    la suma de $i_1(t)+i_2(t)+i_3(t)$.}
		
  En primer lugar, se expresa todo en funciones seno:
  \begin{align*}
    i_1(t)&=2\cdot\sqrt{2}\cdot \sin(600\cdot t)\rightarrow \overline{I_1}=2\phase{0^\circ}\;\si{\ampere}\\
    i_2(t)&=4\cdot\sqrt{2}\cdot \sin\left(600\cdot t+\frac{\pi}{2}\right)\rightarrow \overline{I_2}=4\phase{90^\circ}\;\si{\ampere}\\
    i_3(t)&=10\cdot\sqrt{2}\cdot \cos(600\cdot t-\pi)=10\cdot\sqrt{2}\cdot \sin\left(600\cdot t-\frac{\pi}{2}\right)\rightarrow \overline{I_3}=10\phase{-90^\circ}\;\si{\ampere}\
  \end{align*}
		
  Así, la suma de $i_1(t)+i_2(t)+i_3(t)$ es:
  \begin{align*}
    \overline{I_1}&+\overline{I_2}+\overline{I_3}=(2\phase{0^\circ})+(4\phase{90^\circ})+(10\phase {-90^\circ})=6.32\phase{-71.5651^\circ}\;\si{\ampere} \\
                  &i_1(t)+i_2(t)+i_3(t)=6.32\cdot\sqrt{2}\cdot \sin(600\cdot t-1.249)\;\si{\ampere}
  \end{align*}
\end{example}
	
\subsection{Representación fasorial: diagramas fasoriales}
	
Considérense las ondas de tensión $u(t)$ y corriente $i(t)$ mostradas
en la Figura~\ref{fig:ondasTensionCorriente}, representadas en
notación fasorial como:
\begin{equation*}
  \overline{U} = U\phase{\theta_U};\qquad\qquad   \overline{I} = I\phase{\theta_I}
\end{equation*}
donde $U$ e $I$ son los valores eficaces de la tensión y corriente,
respectivamente. Su representación gráfica en el plano es la mostrada
en la Figura~\ref{fig:fasortensioncorriente}. A estos diagramas se los
conoce como \textbf{diagramas fasoriales}, y permiten también el
estudio y análisis de circuitos en corriente alterna como si de
vectores en el plano se tratara. Además, este procedimiento gráfico
ofrece la ventaja, respecto al procedimiento algebraico, de que las
relaciones de fase y amplitud entre todas las tensiones e intensidades
quedan expuestas de forma muy clara e intuitiva. Por tanto, a lo largo
de este Tema~\ref{chap:ca_mono} se irán realizando y analizando los
diagramas fasoriales correspondientes a los circuitos en estudio.
	
\begin{figure}[H]
  \centering
  \includegraphics[width=.9\linewidth]{../figs/ondasTensionCorriente.pdf}
  \caption{Tensión y corriente en notación fasorial}
  \label{fig:ondasTensionCorriente}
\end{figure}
	
	
\begin{figure}[H]
  \centering
  \includegraphics[width=0.3\linewidth]{../figs/fasorTensionCorriente.pdf}
  \caption{Diagrama fasorial de $\overline{U}$ e $\overline{I}$}
  \label{fig:fasortensioncorriente}
\end{figure}
	
\begin{remark}
  A partir de este momento, se utilizará siempre $U$ para referirse a
  tensión eficaz e $I$ para la corriente eficaz.
\end{remark}
	
\section{Respuesta de los elementos pasivos a una excitación
  sinusoidal}
	
La ley de Ohm también puede escribirse utilizando fasores, de manera
que:
\begin{equation}\label{eq:ohm_generalizada}
  \boxed{ \overline{U}=\overline{Z}\cdot\overline{I} }
\end{equation}
siendo la impedancia:
\begin{equation}\label{eq:impedancia}
  \boxed{\overline{Z} = \frac{U}{I}\phase{\theta_U - \theta_I} \Rightarrow 
    \begin{cases}
      Z = \frac{U}{I}\\
      \varphi = \theta_U - \theta_I
    \end{cases}}
\end{equation}
Por tanto, la impedancia $\overline{Z}=Z\phase{\varphi}$ es el cociente
entre tensión y corriente [$\Omega$]. De nuevo, la expresión para la
impedancia mostrada en la ecuación~\eqref{eq:impedancia} se presenta
en forma polar, siendo en forma binómica:
\begin{equation*}
  \overline{Z} = Z\cdot\cos(\varphi)+\mathrm{j}\,Z\cdot\sin(\varphi) %= R + \mathrm{j} X
\end{equation*}
cuyo resultado, según se demostrará más adelante, es igual a:
\begin{equation}
  \boxed{\overline{Z} =  R + \mathrm{j} X}
\end{equation}
donde $R$ es la parte resistiva de la impedancia (resistencia), y $X$
es la parte reactiva (bobina y/o condensador), como se muestra en la
Figura~\ref{fig:fasorimpedancia}. La parte imaginaria de
$\overline{Z}$ (la $X$) puede ser positiva (reactancia inductiva
$\rightarrow$ bobina) o negativa (reactancia capacitiva $\rightarrow$
condensador). Además, una impedancia puede ser puramente resistiva
($Z=R$), inductiva ($Z=+X$) o capacitiva ($Z=-X$).
\begin{figure}[H]
  \centering \includegraphics{../figs/fasorImpedancia.pdf}
  \caption{Fasor de una impedancia genérica $\overline{Z}$}
  \label{fig:fasorimpedancia}
\end{figure}
	
\begin{remark}
  Cuando los elementos pasivos son puramente reactivos, la impedancia
  se conoce como reactancia $X$ y, la admitancia, como susceptancia
  $B$.
\end{remark}
			
\subsection{Circuito resistivo}\label{sec:R-puro}
	
Considérese una resistencia $R$ por la que circula una corriente
alterna de forma de onda:
\begin{equation*}
  i(t)=I\,\sqrt{2}\cdot\sin(\omega t+\theta_I)\rightarrow\overline{I}=I\phase{\theta_I}
\end{equation*}
Aplicando la ley de Ohm, la tensión en los bornes de la resistencia
es:
\begin{equation*}
  u(t)=R\cdot i(t) ={\color{blue}R\cdot I}\,\sqrt{2}\cdot\sin(\omega t+{\color{red}\theta_I})={\color{blue}U}\,\sqrt{2}\cdot\sin(\omega t+{\color{red}\theta_U})\,,
\end{equation*}
función senoidal que va \textbf{en fase} con la intensidad
(Figura~\ref{fig:resistivo}), y cuyo valor eficaz es $U=R\cdot I$. Por
tanto, le corresponde el fasor
\begin{equation*}
  \overline{U}=(R\cdot I)\phase{0+\theta_I}=U\phase{\theta_U}
\end{equation*}
donde $\theta_I=\theta_U$.
\begin{figure}[H]
  \centering \includegraphics{../figs/resistivo.pdf}
  \caption{$u(t)$ e $i(t)$ en circuitos resistivos puros}
  \label{fig:resistivo}
\end{figure}
	
Así, la impedancia $\overline{Z_R}$, según la
expresión~\eqref{eq:impedancia}:
\begin{equation*}
  \overline{Z_R}=\dfrac{\overline{U}}{\overline{I}}\Rightarrow
  \begin{cases}
    Z_R=\dfrac{U}{I}=R\\
    \varphi=\theta_U-\theta_I=0
  \end{cases}
\end{equation*}
Es decir, que la impedancia de una resistencia tiene de módulo el
valor de la resistencia y un argumento nulo:
\begin{equation}\label{eq:resistencia}
  \boxed{ \overline{Z_R}=R+\mathrm{j}\,0=R\phase{0^\circ}}
\end{equation}
La representación fasorial de $\overline{U}$ e $\overline{I}$, así
como la de $\overline{Z_R}$, se muestran en la
Figura~\ref{fig:fasorResistencia}.
\begin{figure}[H]
  \centering \subfloat[$\overline{U}$ e
  $\overline{I}$]{\includegraphics[width=0.2\linewidth]{../figs/fasorResistencia_VI.pdf}}\hfil
  \subfloat[$\overline{Z_R}$]{\includegraphics[width=0.22\linewidth]{../figs/fasorResistencia.pdf}}
  \caption{Diagrama fasorial de un circuito resistivo puro}
  \label{fig:fasorResistencia}
\end{figure}
	
\subsection{Circuito inductivo puro}\label{sec:L-puro}
	
Considérese una bobina de inductancia $L$ por la que circula una
corriente alterna de forma de onda:
\begin{equation*}
  i(t)=I\,\sqrt{2}\cdot\sin(\omega t+\theta_I)\rightarrow\overline{I}=I\phase{\theta_I}
\end{equation*}
Según se indica en la expresión~\eqref{eq:u_L}, la relación entre
tensión y corriente en una bobina es:
\begin{equation*}
  u(t)=L\cdot\dfrac{di(t)}{dt}={\color{blue}L\cdot I \cdot \omega}\,\sqrt{2}\cdot  \cos(\omega t+{\color{red}\theta_I})= {\color{blue} U} \sqrt{2}\cdot \omega\cdot  \sin \left(\omega t+{\color{red}\theta_I+\frac{\pi}{2}}\right)= {\color{blue} U} \sqrt{2}\cdot \omega\cdot  \sin \left(\omega t+{\color{red}\theta_U}\right)\,
\end{equation*}
Así, un circuito inductivo puro genera señales en cuadratura entre
$u(t)$ e $i(t)$, estando la corriente \textbf{retrasada 90$^\circ$}
respecto a la tensión (Figura~\ref{fig:inductivoPuro}). A la tensión
le corresponde el fasor:
\begin{equation*}
  \overline{U}=(L\cdot I\cdot\omega)\phase{\theta_I+\frac{\pi}{2}}=U\phase{\theta_U}
\end{equation*}
\begin{figure}[H]
  \centering \includegraphics{../figs/inductivoPuro.pdf}
  \caption{$u(t)$ e $i(t)$ en circuitos inductivos puros}
  \label{fig:inductivoPuro}
\end{figure}
	
Por tanto, la impedancia $\overline{Z_L}$, según la
expresión~\eqref{eq:impedancia}:
\begin{equation*}
  \overline{Z_L}=\dfrac{\overline{U}}{\overline{I}}\Rightarrow
  \begin{cases}
    Z_L=\dfrac{U}{I}=\omega\cdot L\\
    \varphi=\theta_U-\theta_I=90^\circ
  \end{cases}
\end{equation*}
Es decir, que la impedancia de una bobina tiene de módulo el valor
$L\cdot\omega$, conocido como \textbf{reactancia inductiva}
(resistencia aparente que ofrece la bobina al paso de la corriente
alterna) y un argumento de $+90^\circ$:
\begin{equation}
  \boxed{0+\mathrm{j}\,L\omega=L\omega\phase{90^\circ}}
\end{equation}
La representación fasorial de $\overline{U}$ e $\overline{I}$, así
como la de $\overline{Z_L}$, se muestran en la
Figura~\ref{fig:fasorInductancia}.
\begin{figure}[H]
  \centering \subfloat[$\overline{U}$ e
  $\overline{I}$]{\includegraphics[width=0.28\linewidth]{../figs/fasorInductancia_VI.pdf}}\hfil
  \subfloat[$\overline{Z_L}$]{\includegraphics[width=0.19\linewidth]{../figs/fasorInductancia.pdf}}
  \caption{Diagrama fasorial de un circuito inductivo puro}
  \label{fig:fasorInductancia}
\end{figure}
	
\subsection{Circuito capacitivo puro}\label{sec:C-puro}
	
Considérese un condensador de capacidad $C$ por el que circula una
corriente alterna de forma de onda:
\begin{equation*}
  i(t)=I\,\sqrt{2}\cdot\sin(\omega t+\theta_I)\rightarrow\overline{I}=I\phase{\theta_I}
\end{equation*}
Según se indica en la expresión~\eqref{eq:u_C}, la relación entre
tensión y corriente en un condensador, (considerando que $t_i=-\infty$
y que $u(-\infty)=0$) es:
\begin{equation*}
  u(t)=\dfrac{1}{C}\cdot\int_{-\infty}^{t} i(t)\cdot dt=-{\color{blue}\dfrac{I}{\omega\,C}}\sqrt{2}\cdot\cos (\omega t+{\color{red}\theta_I})={\color{blue}U}\sqrt{2}\cdot\sin \left(\omega t+{\color{red}\theta_I-\frac{\pi}{2}}\right)
\end{equation*}
Así, un circuito capacitivo puro genera señales en cuadratura entre
$u(t)$ e $i(t)$, estando la corriente \textbf{adelantada 90$^\circ$}
respecto a la tensión (Figura~\ref{fig:capacitivoPuro}). A la tensión
le corresponde el fasor:
\begin{equation*}
  \overline{U}=\left( \dfrac{I}{\omega C} \right)\phase{\theta_I-\frac{\pi}{2}}=U\phase{\theta_U}
\end{equation*}
\begin{figure}[H]
  \centering \includegraphics{../figs/capacitivoPuro.pdf}
  \caption{$u(t)$ e $i(t)$ en circuitos capacitivos puros}
  \label{fig:capacitivoPuro}
\end{figure}
	
Por tanto, la impedancia $\overline{Z_C}$, según la
expresión~\eqref{eq:impedancia}:
\begin{equation*}
  \overline{Z_C}=\dfrac{\overline{U}}{\overline{I}}\Rightarrow
  \begin{cases}
    Z_C=\dfrac{U}{I}=\dfrac{1}{\omega C}\\
    \varphi=\theta_U-\theta_I=-90^\circ
  \end{cases}
\end{equation*}
Es decir, que la impedancia de un condensador tiene de módulo el valor
$\frac{1}{\omega C}$, conocido como \textbf{reactancia capacitiva}
(resistencia aparente que ofrece el condensador al paso de la
corriente alterna) y un argumento de $-90^\circ$:
\begin{equation}
  \boxed{0-\mathrm{j}\,\dfrac{1}{\omega C}=\dfrac{1}{\omega C}\phase{-90^\circ}}
\end{equation}
La representación fasorial de $\overline{U}$ e $\overline{I}$, así
como la de $\overline{Z_C}$, se muestran en la
Figura~\ref{fig:fasorCondensador}.
\begin{figure}[H]
  \centering \subfloat[$\overline{U}$ e
  $\overline{I}$]{\includegraphics[width=0.28\linewidth]{../figs/fasorCondensador_VI.pdf}}\hfil
  \subfloat[$\overline{Z_C}$]{\includegraphics[width=0.19\linewidth]{../figs/fasorCondensador.pdf}}
  \caption{Diagrama fasorial de un circuito capacitivo puro}
  \label{fig:fasorCondensador}
\end{figure}
	
\section{Respuesta de los circuitos serie a una excitación
  senoidal} \label{sec:respuesta_serie}
	
\subsection{Circuito $RL$}\label{sec:RL}
	
Este circuito se corresponde con el mostrado en la
Figura~\ref{fig:RL}, equivalente a un circuito inductivo con pérdidas.
\begin{figure}[H]
  \centering \includegraphics[width=0.3\linewidth]{../figs/RL.pdf}
  \caption{Circuito $RL$ serie}
  \label{fig:RL}
\end{figure}
	
La corriente que circula por el circuito es:
\begin{equation*}
  i(t)=I\,\sqrt{2}\cdot\sin(\omega t+\theta_I)\rightarrow\overline{I}=I\phase{\theta_I}
\end{equation*}
por lo que las tensiones en la resistencia $R$ y bobina $L$:
\begin{align*}
  u_R(t)=R\, I\,\sqrt{2}\cdot\sin(\omega t+\theta_I)&\rightarrow \overline{U_R} = R \overline{I}=R\,I\phase{\theta_I}\\ 
  u_L(t)= \omega\,L\,I \sqrt{2}\cdot \sin \left(\omega t+\theta_I+\frac{\pi}{2}\right)&\rightarrow \overline{U_L}=\overline{X_L}\cdot\overline{I}= \omega\,L\,I\phase{\theta_I+90^\circ}
\end{align*}
donde $u_R(t)$ va en fase con $i(t)$ y $u_L(t)$ va 90$^\circ$
adelantada respecto a $i(t)$.
	
La impedancia del circuito (resistencia aparente que ofrece al paso de
la corriente alterna) es el conjunto de $R$ y $L$, que se corresponde
con una magnitud compleja:
\begin{equation}
  \boxed{ \overline{Z} = R + \mathrm{j}\,X_L = R+ \mathrm{j}\,\omega L \Rightarrow 
    \begin{cases}
      Z=\sqrt{R^2+(\omega L)^2}\\
      \varphi=\arctan\left(\dfrac{\omega\,L}{R} \right)
    \end{cases}}
\end{equation}
que puede representarse en el plano complejo como se muestra en la
Figura~\ref{fig:fasorinductanciareal}, donde es inmediato comprobar
que:
\begin{align*}
  R&=Z\cdot\cos(\varphi)\\
  X_L&=Z\cdot\sin(\varphi)
\end{align*} 
\begin{figure}[H]
  \centering \includegraphics{../figs/fasorInductanciaReal.pdf}
  \caption{Representación gráfica de la impedancia de un circuito
    $RL$}
  \label{fig:fasorinductanciareal}
\end{figure}
	
	
La tensión total del circuito, según la 2LK, es:
\begin{equation*}
  \overline{U} = \overline{U_R} + \overline{U_L} =(R + \mathrm{j}\,\omega L) \cdot \overline{I}\Rightarrow 
  \begin{cases}
    U=\sqrt{U_R^2+U_L^2}=I\sqrt{R^2+(\omega L)^2}=I\cdot Z\\
    \theta=\arctan\left( \dfrac{U_L}{U_R}\right)=\arctan\left( \dfrac{\omega L}{R}\right)
  \end{cases}
\end{equation*}
Por tanto, la intensidad $i(t)$ va \textbf{retrasada} respecto a la
tensión total $u(t)$, pero \textbf{no en cuadratura} con ésta
(Figura~\ref{fig:fasorInductanciaReal_VI}).
	
\begin{figure}[H]
  \centering \subfloat[Evolución
  temporal]{\includegraphics[width=0.7\linewidth]{../figs/inductivo.pdf}}\hfil
  \subfloat[Diagrama fasorial]{
    \includegraphics{../figs/fasorInductanciaReal_VI.pdf}}
  \caption{Evolución temporal y diagrama fasorial de $u(t)$ e $i(t)$
    en circuitos $RL$}
  \label{fig:fasorInductanciaReal_VI}
\end{figure}
	
	
	
\subsection{Circuito $RC$}\label{sec:RC}
	
Este circuito se corresponde con el mostrado en la
Figura~\ref{fig:RC}, equivalente a un circuito capacitivo con
pérdidas.
\begin{figure}[H]
  \centering \includegraphics{../figs/RC.pdf}
  \caption{Circuito $RC$ serie}
  \label{fig:RC}
\end{figure}
	
La corriente que circula por el circuito es:
\begin{equation*}
  i(t)=I\,\sqrt{2}\cdot\sin(\omega t+\theta_I)\rightarrow\overline{I}=I\phase{\theta_I}
\end{equation*}
por lo que las tensiones en la resistencia $R$ y el condensador $C$:
\begin{align*}
  u_R(t)=R\, I\,\sqrt{2}\cdot\sin(\omega t+\theta_I)&\rightarrow \overline{U_R} = R \overline{I}=R\,I\phase{\theta_I}\\ 
  u_C(t)=\dfrac{I\,\sqrt{2}}{\omega\,C} \cdot \sin \left(\omega t+\theta_I-\frac{\pi}{2}\right)&\rightarrow \overline{U_C}=\overline{X_C}\cdot\overline{I}=\dfrac{I}{\omega C}\phase{\theta_I-90^\circ}
\end{align*}
donde $u_R(t)$ va en fase con $i(t)$ y $u_C(t)$ va 90$^\circ$
retrasada respecto a $i(t)$.
	
La impedancia del circuito (resistencia aparente que ofrece al paso de
la corriente alterna) es el conjunto de $R$ y $C$, que se corresponde
con una magnitud compleja:
\begin{equation}
  \boxed{ \overline{Z} = R + \mathrm{j}\,X_C = R- \mathrm{j}\dfrac{1}{\omega C} \Rightarrow 
    \begin{cases}
      Z=\sqrt{R^2-\left(\dfrac{1}{\omega C} \right)^2}\\
      \varphi=-\arctan\left(\dfrac{\frac{1}{\omega\,C}}{R} \right)
    \end{cases}}
\end{equation}
que puede representarse en el plano complejo como se muestra en la
Figura~\ref{fig:fasorcondensadorreal}, donde es inmediato comprobar
que:
\begin{align*}
  R&=Z\cdot\cos(\varphi)\\
  X_C&=Z\cdot\sin(\varphi)
\end{align*} 
\begin{figure}[H]
  \centering \includegraphics{../figs/fasorCondensadorReal.pdf}
  \caption{Representación gráfica de la impedancia de un circuito
    $RC$}
  \label{fig:fasorcondensadorreal}
\end{figure}
	
La tensión total del circuito, según la 2LK, es:
\begin{equation*}
  \overline{U} = \overline{U_R} + \overline{U_C} =\left(R - \mathrm{j}\,\dfrac{1}{\omega\,C}\right) \cdot \overline{I}\Rightarrow 
  \begin{cases}
    U=\sqrt{U_R^2+U_C^2}=I\sqrt{R^2+\left(\dfrac{1}{\omega\,C}\right)^2}=I\cdot Z\\
    \theta=-\arctan\left( \dfrac{U_C}{U_R}\right)=-\arctan\left( \dfrac{1}{R\,\omega\,C}\right)
  \end{cases}
\end{equation*}
Por tanto, la intensidad $i(t)$ va \textbf{adelantada} respecto a la
tensión total $u(t)$, pero \textbf{no en cuadratura} con ésta
(Figura~\ref{fig:fasorCapacitivoReal_VI}).
\begin{figure}[H]
  \centering \subfloat[Evolución
  temporal]{\includegraphics[width=0.7\linewidth]{../figs/capacitivo.pdf}}\hfil
  \subfloat[Diagrama fasorial]{
    \includegraphics[width=0.28\linewidth]{../figs/fasorCondensadorReal_VI.pdf}}
  \caption{Evolución temporal y diagrama fasorial de $u(t)$ e $i(t)$
    en circuitos $RC$}
  \label{fig:fasorCapacitivoReal_VI}
\end{figure}
	
\begin{remark}
  Se quiere destacar que, pese a que a una impedancia se le puede
  asociar un número complejo, no se trata de \textbf{un fasor} (señal
  que varía en el tiempo).
\end{remark}
	
\subsection{Circuito $RLC$}\label{sec:RLC}
	
Este circuito se corresponde con el mostrado en la
Figura~\ref{fig:RLC}.
\begin{figure}[H]
  \centering \includegraphics[width=0.4\linewidth]{../figs/RLC.pdf}
  \caption{Circuito $RLC$ serie}
  \label{fig:RLC}
\end{figure}
	
La corriente que circula por el circuito es:
\begin{equation*}
  i(t)=I\,\sqrt{2}\cdot\sin(\omega t+\theta_I)\rightarrow\overline{I}=I\phase{\theta_I}
\end{equation*}
por lo que las tensiones en resistencia $R$, bobina $L$ y condensador
$C$:
\begin{align*}
  u_R(t)=R\, I\,\sqrt{2}\cdot\sin(\omega t+\theta_I)&\rightarrow \overline{U_R} = R \overline{I}=R\,I\phase{\theta_I}\\ 
  u_L(t)= \omega\,L\,I \sqrt{2}\cdot \sin \left(\omega t+\theta_I+\frac{\pi}{2}\right)&\rightarrow \overline{U_L}=\overline{X_L}\cdot\overline{I}= \omega\,L\,I\phase{\theta_I+90^\circ}\\
  u_C(t)=\dfrac{I\,\sqrt{2}}{\omega\,C} \cdot \sin \left(\omega t+\theta_I-\frac{\pi}{2}\right)&\rightarrow \overline{U_C}=\overline{X_C}\cdot\overline{I}=\dfrac{I}{\omega C}\phase{\theta_I-90^\circ}
\end{align*}
donde $u_R(t)$ va en fase con $i(t)$, $u_L(t)$ va 90$^\circ$
adelantada respecto a $i(t)$ y $u_C(t)$ va 90$^\circ$ retrasada
respecto a $i(t)$.
	
La impedancia del circuito (resistencia aparente que ofrece al paso de
la corriente alterna) es el conjunto de $R$, $L$ y $C$, que se
corresponde con una magnitud compleja:
\begin{equation}
  \boxed{ \overline{Z} = R + \mathrm{j}\,(X_L-X_C) = R+ \mathrm{j}\left(\omega\,L-\dfrac{1}{\omega C}\right) \Rightarrow 
    \begin{cases}
      Z=\sqrt{R^2+\left(\omega L -\frac{1}{\omega C} \right)^2}\\
      \varphi=\arctan\left(\dfrac{\omega L-\frac{1}{\omega\,C}}{R} \right)
    \end{cases}}
\end{equation}
que puede representarse en el plano complejo como se muestra en la
Figura~\ref{fig:fasorRLC}, donde es inmediato comprobar que:
\begin{align*}
  R&=Z\cdot\cos(\varphi)\\
  X&=Z\cdot\sin(\varphi)
\end{align*} 
\begin{figure}[H]
  \centering \includegraphics{../figs/fasorRLC.pdf}
  \caption{Representación gráfica de la impedancia de un circuito
    $RLC$}
  \label{fig:fasorRLC}
\end{figure}
	
La tensión total del circuito, según la 2LK, es:
\begin{equation*}
  \overline{U} = \overline{U_R} +\overline{U_L} + \overline{U_C} =\left(R+\mathrm{j}\,\omega\,L - \mathrm{j}\,\dfrac{1}{\omega\,C}\right) \cdot \overline{I}\Rightarrow 
  \begin{cases}
    U=I\sqrt{R^2 + \left(\omega L - \dfrac{1}{\omega C}\right)^2}=I\cdot Z\\
    \theta=\arctan\left( \dfrac{\omega\,L-\frac{1}{\omega\,C}}{R}\right)
  \end{cases}
\end{equation*}
Por tanto, \textit{a priori}, no se puede saber si la intensidad
$i(t)$ va {adelantada o retrasada} respecto a la tensión total $u(t)$,
puesto que dependerá de si
$\overline{U_L}>\overline{U_C}\Rightarrow \omega L>\frac{1}{\omega C}$
(\textbf{carácter inductivo}) o
$\overline{U_L}<\overline{U_C}\Rightarrow \omega L<\frac{1}{\omega C}$
(\textbf{carácter capacitivo}); además, también puede darse el caso de
que
$\overline{U_L}=\overline{U_C}\Rightarrow \omega L=\frac{1}{\omega C}$
(\textbf{carácter resistivo}), diciendo entonces que el circuito se
encuentra \textbf{en resonancia} (ver
Sección~\ref{sec:resonancia_serie}). En la
Figura~\ref{fig:fasorRLC_VI} ha supuesto que la corriente va en
retraso respecto a la tensión.
\begin{figure}[H]
  \centering
  \includegraphics[width=0.3\linewidth]{../figs/fasorRLC_VI.pdf}
  \caption{Diagrama fasorial de un circuito $RLC$, suponiendo carácter
    inductivo}
  \label{fig:fasorRLC_VI}
\end{figure}
	
	
	
\subsection{Circuito serie general}
Considérese un circuito en serie formado por $n$ impedancias, donde
cada impedancia es de la forma $\overline{Z_i}=R_i+\mathrm{j}\,X_i$,
como se muestra en la Figura~\ref{fig:serie-general-inicio}. Este
circuito se alimenta con una tensión $u(t)$, de valor eficaz $U$, de
manera que circula por el mismo una intensidad $i(t)$ de valor eficaz
$I$. Se dice que la impedancia equivalente a las $n$ impedancias en
serie es aquella que, al aplicarle la misma tensión $u(t)$, origina la
misma intensidad $i(t)$, es decir, la que \textbf{conserva el módulo
  de $\overline{I}$ y el ángulo de fase} entre $\overline{U}$ e
$\overline{I}$ del circuito serie original
\begin{figure}[H]
  \centering
  \subfloat[Real]{\includegraphics[height=2cm]{../figs/serie_general.pdf}\label{fig:serie-general-inicio}}\hfil
  \subfloat[Equivalente]{\includegraphics[height=2cm]{../figs/serie_general_eq.pdf}\label{fig:serie-general-eq}}
  \caption{Circuito serie general alimentado por corriente alterna}
  \label{fig:serie-general}
\end{figure}
	
En el acoplamiento de la Figura~\ref{fig:serie-general-inicio} se
cumple que:
\begin{equation*}
  \overline{U}=\overline{U_1}+\overline{U_2}+\overline{U_3}+...+\overline{U_n}
\end{equation*}
donde cada tensión es igual a
$\overline{U_i}=\overline{I}\cdot\overline{Z_i}$, luego:
\begin{equation*}
  \overline{U}=\overline{U_1}+\overline{U_2}+\overline{U_3}+...+\overline{U_n}=\overline{I} \cdot(\overline{Z_1}+\overline{Z_2}+\overline{Z_3}+...+\overline{Z_n})
\end{equation*}
Puesto que en el circuito equivalente se cumple que:
\begin{equation*}
  \overline{U}=\overline{I}\cdot\overline{Z_{eq}}
\end{equation*}
se llega a la conclusión de que:
\begin{equation}
  \overline{Z_{eq}}=\overline{Z_1}+\overline{Z_2}+\overline{Z_3}+...+\overline{Z_n}\Rightarrow \boxed{\overline{Z_{eq}}=\sum_{i=1}^n \overline{Z_i}}
\end{equation}
que equivale a decir que:
\begin{equation*}
  R_{eq}=\sum_{i=1}^n R_i\,;\qquad \qquad X_{eq}=\sum_{i=1}^n X_i
\end{equation*}
siendo el ángulo de la impedancia equivalente:
\begin{equation*}
  \varphi=\arctan\left(\dfrac{X_{eq}}{R_{eq}}\right)
\end{equation*}
	
\vspace{4mm}
\begin{example}\label{ej.2-3}
  \textbf{Un circuito serie formado por $R=\qty{10}{\ohm}$,
    $L=\qty{20}{\milli\henry}$ y $C=\qty{100}{\micro\farad}$ es
    alimentado con una tensión
    $u(t)=200\cdot\sin(1000t+\frac{\pi}{4})\,\si{\volt}$. Calcular
    $\overline{I}$, ${u_R(t)}$, $u_L(t)$ y $u_C(t)$, y dibujar el
    diagrama fasorial de tensiones y corrientes.}

  \vspace{4mm} El valor eficaz de la tensión y su fase inicial son:
  \begin{equation*}
    U = \dfrac{U_{max}}{\sqrt{2}} =\dfrac{200}{\sqrt{2}} = 100\sqrt2\,\si{\volt} \;;\;\;\;\theta_U=\dfrac{\pi}{4}=45^\circ \quad \Rightarrow \quad\overline{U}=100\sqrt2\phase{\ang{45}}
  \end{equation*}
  Los valores de las impedancias $X_L$, $X_C$ y la impedancia
  equivalente son:
  \begin{align*}
    \overline{X}_L&=\mathrm{j}\,\omega\,L=\mathrm{j}\,1000\cdot20\cdot 10^{-3}= \mathrm{j}\,\qty{20}{\ohm}=20\phase{\ang{90}}\,\si{\ohm}\\
    \overline{X}_C&=-\mathrm{j}\,\dfrac{1}{\omega\,C}=\dfrac{1}{1000\cdot100\cdot 10^{-6}}= -\mathrm{j}\,\qty{10}{\ohm}=10\phase{\ang{-90}}\,\si{\ohm}		\\
    \overline{Z}_{eq}&=R+\overline{X}_L+\overline{X}_C=10+\mathrm{j}\,20-\mathrm{j}\,10=10+\mathrm{j}\,10\,\si{\ohm}=10\sqrt2 \phase{\ang{45}}\,\si{\ohm}
  \end{align*}
  Aplicando la ley de Ohm, se obtiene la corriente y, con ella, las
  tensiones de cada elemento:
  \begin{align*}
    \overline{I}&=\dfrac{\overline{U}}{\overline{Z}_{eq}}=\dfrac{100\sqrt2\phase{45^\circ}}{10\sqrt2\phase{45^\circ}}=10\phase{0^\circ}\,\si{\ampere}\\
    \overline{U}_R&=\overline{I}\cdot R=10\phase{0^\circ}\cdot 10=100\phase{0^\circ}\,\si{\volt} \quad \Rightarrow \quad u_R(t)=100\sqrt{2}\cdot\sin(1000t)\,\si{\volt}\\
    \overline{U}_L&=\overline{I}\cdot \overline{X}_L=10\phase{0^\circ}\cdot 20\phase{90^\circ}=200\phase{90^\circ}\,\si{\volt} \quad \Rightarrow \quad u_L(t)=200\sqrt{2}\cdot\sin\left(1000t+\tfrac{\pi}{2}\right)\,\si{\volt}\\
    \overline{U}_C&=\overline{I}\cdot \overline{X}_L=10\phase{0^\circ}\cdot 10\phase{-90^\circ} =100\phase{-90^\circ}\,\si{\volt} \quad \Rightarrow \quad u_C(t)=100\sqrt{2}\cdot\sin\left(1000t-\tfrac{\pi}{2}\right)\,\si{\volt}
  \end{align*}
  \begin{figure}[H]
    \centering
    \includegraphics[]{../figs/diagrama_fasorial_ejemplo2_3.pdf}
    \caption{Diagrama fasorial del Ejemplo~\ref{ej.2-3}}
    \label{fig:diagrama_fasorial_ejemplo2-3}
  \end{figure}
		
\end{example}
	
\subsection{Resonancia}\label{sec:resonancia_serie}
	
En un circuito serie $RLC$ se dice que se produce \textbf{resonancia}
cuando la reactancia es nula:
\begin{equation*}
  X=\omega\,L-\dfrac{1}{\omega\,C}=0\rightarrow \omega\,L=\dfrac{1}{\omega\,C}
\end{equation*}
Por tanto, cuando se produce resonancia, se tiene que
$\overline{Z}=R$, siendo un circuito resistivo puro y $Z$ alcanzando
su mínimo valor. Como, por ley de Ohm,
$\overline{I}=\overline{U}/\overline{Z}$, la intensidad alcanzará su
máximo valor y estará en fase con la tensión.
	
Para alcanzar resonancia, se puede llegar variando la autoinducción
$L$, la capacidad $C$ o la pulsación $\omega$. La pulsación $\omega_0$
necesaria para que se produzca resonancia, manteniendo constantes $L$
y $C$ es:
\begin{equation}
  \omega_0\,L=\dfrac{1}{\omega_0\,C}\rightarrow \boxed{\omega_0=\dfrac{1}{\sqrt{L\,C}}}
\end{equation}
\begin{remark}
  Que la impedancia $X$ tenga un valor nulo, no implica que la tensión
  en la bobina y el condensador sean 0. De hecho, pueden presentarse
  tensiones elevadas en éstos.
\end{remark}
	
\vspace{4mm}
\begin{example}\label{ej.2-4}
  \textbf{Un circuito en serie formado por $R=1\,\Omega$,
    ${X_L}=100\,\Omega$ y ${X_C}=-100\,\Omega$ se conecta a una
    tensión alterna
    $\overline{U}=100\phase{0^\circ}\,\si{\volt}$. Calcular las
    tensiones de cada elemento.}

  \vspace{4mm} La reactancia total $\overline{X}$ es:
  \begin{equation*}
    \overline{X}= \overline{X}_L+\overline{X}_C=\mathrm{j}\,100-\mathrm{j}\,100=0
  \end{equation*}
		
  Se trata de un circuito resonante, siendo la impedancia equivalente:
  \begin{equation*}
    \overline{Z}=R+\mathrm{j}\,X=1+\mathrm{j}\,0=1\,\Omega
  \end{equation*}
		
  Por la ley de Ohm, la corriente resulta:

  \vspace{-2mm}
  \begin{equation*}
    \overline{I}=\dfrac{\overline{U}}{\overline{Z}}=\dfrac{100\phase{0^\circ}}{1\phase{0^\circ}}=100\phase{0^\circ}\,\si{\ampere}
  \end{equation*}
		
  Por tanto, la tensión en cada elemento es:
  \begin{align*}
    \overline{U}_R&=R\cdot\overline{I}=1\cdot 100\phase{0^\circ}=100\phase{0^\circ}\,\si{\volt}\\
    \overline{U}_L&=\overline{X}_L\cdot\overline{I}=100\phase{90^\circ}\cdot 100\phase{0^\circ}=10^4\phase{90^\circ}\,\si{\volt}=10\phase{90^\circ}\,\si{\kilo\volt}\\
    \overline{U}_C&=\overline{X}_C\cdot\overline{I}=100\phase{-90^\circ}\cdot 100\phase{0^\circ}=10^4\phase{-90^\circ}\,\si{\volt}=10\phase{-90^\circ}\,\si{\kilo\volt}
  \end{align*}
\end{example}
	
	
\section{Respuesta de los circuitos paralelo a una excitación
  senoidal}
\begin{figure}[bp]
  \centering
  \subfloat[Real]{\includegraphics[height=4cm]{../figs/paralelo_general.pdf}\label{fig:paralelo-general-inicio}}\hfil
  \subfloat[Equivalente]{\includegraphics[height=4cm]{../figs/paralelo_general_eq.pdf}\label{fig:paralelo-general-eq}}
  \caption{Circuito paralelo general alimentado por corriente alterna}
  \label{fig:paralelo-general}
\end{figure}
	
Para el caso de circuitos en paralelo, se analiza directamente el
circuito general.
	%
% \subsection{Circuito paralelo general}
Considérese un circuito en paralelo formado por $n$ impedancias, donde
cada impedancia es de la forma $\overline{Z_i}=R_i+\mathrm{j}\,X_i$,
como se muestra en la Figura~\ref{fig:paralelo-general-inicio}. Este
circuito se alimenta con una tensión $u(t)$, de valor eficaz $U$, de
manera que circula por cada impedancia una intensidad $i_i(t)$
obtenidas según:
\begin{equation*}
  \overline{I_i}=\dfrac{\overline{U}}{\overline{Z_i}}
\end{equation*}
cumpliéndose, según la 1LK, que:
\begin{equation*}
  \overline{I}=\overline{I_1}+\overline{I_2}+\overline{I_3}+...+\overline{I_n}
\end{equation*}
Por tanto, la corriente total $\overline{I}$ es:
\begin{equation*}
  \overline{I}=\overline{I_1}+\overline{I_2}+\overline{I_3}+...+\overline{I_n}=\overline{U} \cdot\left(\dfrac{1}{\overline{Z_1}}+\dfrac{1}{\overline{Z_2}}+\dfrac{1}{\overline{Z_3}}+...+\dfrac{1}{\overline{Z_n}}\right)
\end{equation*}
Se dice que la impedancia equivalente a las $n$ impedancias en
paralelo es aquella que, al aplicarle la misma tensión $u(t)$, origina
la misma intensidad $i(t)$, es decir, la que \textbf{conserva el
  módulo de $\overline{I}$ y el ángulo de fase} entre $\overline{U}$ e
$\overline{I}$ del circuito paralelo original. Puesto que en el
circuito equivalente se cumple que:
\begin{equation*}
  \overline{I}=\dfrac{\overline{U}}{\overline{Z_{eq}}}
\end{equation*}
se llega a la conclusión de que:
\begin{equation}
  \dfrac{1}{\overline{Z_{eq}}}=\dfrac{1}{\overline{Z_1}}+\dfrac{1}{\overline{Z_2}}+\dfrac{1}{\overline{Z_3}}+...+\dfrac{1}{\overline{Z_n}}\Rightarrow \boxed{\dfrac{1}{\overline{Z_{eq}}}=\dfrac{1}{\displaystyle\sum_{i=1}^n \overline{Z_i}}}
\end{equation}
	
\subsection{Admitancia}
	
Puesto que el cálculo de la impedancia equivalente operando de esta
forma es, con frecuencia, complicado y no exento de errores, es más
frecuente hablar de \textbf{admitancia}. Conceptualmente, la
admitancia representa la \textit{facilidad} que ofrece el circuito al
paso de la corriente alterna, y es el recíproco (inversa) de la
impedancia:
\begin{equation*}
  \overline{Y}=\dfrac{1}{\overline{Z}}=\dfrac{1\phase{0^\circ}}{Z\phase{\varphi^\circ}}=\dfrac{1}{Z}\phase{-\varphi^\circ}=Y\phase\psi^\circ=G+\mathrm{j}\,B
\end{equation*}
donde $G$ es la conductancia y $B$ es la susceptancia. En función del
valor de $B$, se tienen tres casos (al igual que con las impedancias):
\begin{itemize}
\item $B>0$: admitancia capacitiva
\item $B<0$: admitancia inductiva
\item $B=0$: admitancia resistiva
\end{itemize}
En los elementos simples, las admitancias se pueden calcular como:
\begin{align}
  \Aboxed{\overline{Y_R}&=\dfrac{1}{R}=G}\Rightarrow G\geq 0\\
  \Aboxed{\overline{Y_L}&=\dfrac{1}{\overline{X_L}}=\dfrac{1}{\mathrm{j}\,\omega\,L}=\dfrac{-\mathrm{j}}{\omega\,L}=B_L}\Rightarrow B_L\leq 0\\
  \Aboxed{\overline{Y_C}&=\dfrac{1}{\overline{X_C}}=\dfrac{1}{\frac{1}{\mathrm{j}\,\omega\,C}}=\mathrm{j}\,\omega\,C=B_C}\Rightarrow B_C\geq 0
\end{align}
Las admitancias se asocian en serie y paralelo de la misma forma que
las impedancias, pero el cálculo de su valor equivalente es justo a la
inversa: es decir, en paralelo su equivalente es la suma de las
admitancias, y en serie el inverso de la admitancia equivalente es la
suma de los inversos de las admitancias.
	
\begin{figure}[H]
  \centering
  \subfloat[Impedancia]{\includegraphics{../figs/Z.pdf}}\hfil
  \subfloat[Admitancia]{\includegraphics{../figs/Y.pdf}}
  \caption{Equivalencias entre impedancia y admitancia}
  \label{fig:equivalencias_impedancia_admitancia}
\end{figure}
De la Figura~\ref{fig:equivalencias_impedancia_admitancia}, se extraen
las siguientes relaciones para que impedancia y admitancia sean
equivalentes (es decir, se cumpla la igualdad
$\overline{Z}\cdot \overline{Y}=1$):
\begin{equation}\label{eq:impedancia-admitancia}
  R+\mathrm{j}\,X=\dfrac{1}{G+\mathrm{j}\,B}=\frac{G-\mathrm{j}\,B}{(G+\mathrm{j}\,B)(G-\mathrm{j}\,B)}=\frac{G-\mathrm{j}\,B}{G^2+B^2} \Rightarrow 
  \boxed{\begin{cases}
    R=\dfrac{G}{G^2+B^2}\\[6pt]
    X=-\dfrac{B}{G^2+B^2}
  \end{cases}}
\end{equation}
\begin{equation}\label{eq:admitancia-impedancia}
  G+\mathrm{j}\,B=\dfrac{1}{R+\mathrm{j}\,X}=\dfrac{R-\mathrm{j}\,X}{(R+\mathrm{j}\,X)(R-\mathrm{j}\,X)}=\dfrac{R-\mathrm{j}\,X}{R^2+X^2} \Rightarrow 
  \boxed{\begin{cases}
    G=\dfrac{R}{R^2+X^2}\\[6pt]
    B=-\dfrac{X}{R^2+X^2}
  \end{cases}}
\end{equation}
	
	
\begin{remark}
  Las dimensiones de la admitancia y sus componentes son la inversa de
  $\Omega$ [S].
\end{remark}
	
La Figura~\ref{fig:impedancia_admitancia} muestra gráficamente una
impedancia $\overline{Z}=R+\mathrm{j}\,X=Z\phase{\varphi}$, donde su
admitancia correspondiente es
$\overline{Y}=G-\mathrm{j}\,B=Y\phase{-\psi}$.
\begin{figure}[H]
  \centering
  \subfloat[Impedancia]{\includegraphics{../figs/impedancia.pdf}}\hfil
  \subfloat[Admitancia]{\includegraphics{../figs/admitancia.pdf}}
  \caption{Representación gráfica de la impedancia y admitancia}
  \label{fig:impedancia_admitancia}
\end{figure}
A partir de este gráfico, se pueden verificar las siguientes
relaciones:
\begin{align*}
  Z=\sqrt{R^2+X^2}\qquad & \qquad Y=\sqrt{G^2+B^2}\\
  \varphi=\arctan\left(\dfrac{X}{R}\right)\qquad &  \qquad \psi=\arctan\left(\dfrac{B}{G}\right)\\
  R=Z\cdot \cos(\varphi)\qquad & \qquad G=Y\cdot \cos(\psi)\\
  X=Z\cdot \sin(\varphi)\qquad & \qquad B=Y\cdot \sin(\psi)
\end{align*}
	
\begin{remark}
  Nótese que la susceptancia $B$ \textbf{siempre} tiene signo
  contrario a la reactancia $X$:
  \begin{align*}
    \text{Circuito inductivo}&: X > 0 \Rightarrow B<0\\
    \text{Circuito capacitivo}&: X < 0 \Rightarrow B>0
  \end{align*}
\end{remark}
	
\begin{example}\label{ex.impedancia_admitancia_eq}
  \textbf{Calcular la impedancia y admitancia compleja equivalentes
    del circuito de la Figura~\ref{fig:impedancia_admitancia_eq}.}
  \begin{figure}[H]
    \centering \includegraphics{../figs/impedancia_admitancia_eq.pdf}
    \caption{Ejemplo~\ref{ex.impedancia_admitancia_eq}}
    \label{fig:impedancia_admitancia_eq}
  \end{figure}
	    
  La impedancia equivalente de la resistencia y la bobina es:
  \begin{equation*}
    \overline{Z}_{R,L}=R+\overline{X}_L=10+\mathrm{j}20\,\Omega
  \end{equation*}
  y la de la resistencia y el condensador:
  \begin{equation*}
    \overline{Z}_{R,C}=R+\overline{X}_C=15-\mathrm{j}15\,\Omega
  \end{equation*}
  siendo por tanto la impedancia equivalente total:
  \begin{equation*}
    \overline{Z}_{eq}=\dfrac{1}{\dfrac{1}{\overline{Z}_{R,L}}+\dfrac{1}{\overline{Z}_{R,C}}}= \dfrac{1}{\dfrac{1}{10+\mathrm{j}20}+\dfrac{1}{15-\mathrm{j}15}}=18.61\phase{7.1250^\circ}\,\Omega
  \end{equation*}
	    
  A partir de la impedancia equivalente se determina la admitancia:
  \begin{equation*}
    \overline{Y}_{eq}=\dfrac{1}{\overline{Z}_{eq}}=\dfrac{1}{18.61\phase{7.1250^\circ}}=0.05\phase{-7.1250^\circ}\,\si{\siemens}
  \end{equation*}
\end{example}
	
\subsection{Antirresonancia}
En un circuito paralelo $RLC$ se dice que se produce
\textbf{antirresonancia} cuando la susceptancia es nula:
\begin{equation*}
  B=\omega\,C-\dfrac{1}{\omega\,L}=0\rightarrow \omega\,C=\dfrac{1}{\omega\,L}
\end{equation*}
Por tanto, cuando se produce resonancia, se tiene que
$\overline{Y}=G$, siendo un circuito resistivo puro. Como en el caso
de la resonancia, se puede llegar a la antirresonancia variando la
autoinducción $L$, la capacidad $C$ o la pulsación $\omega$. La
pulsación $\omega_0$ necesaria para que se produzca antirresonancia,
manteniendo constantes $L$ y $C$ es:
\begin{equation}
  \omega_0\,C=\dfrac{1}{\omega_0\,L}\rightarrow \boxed{\omega_0=\dfrac{1}{\sqrt{L\,C}}}
\end{equation}
\begin{remark}
  Que la susceptancia $B$ tenga un valor nulo, no implica que la
  corriente en la bobina y el condensador sean 0. De hecho, pueden
  presentarse corrientes elevadas en éstos.
\end{remark}
	
\vspace{4mm}
\begin{example}\label{ej.2-5}
  \textbf{Un circuito en paralelo formado por $R=1\,\Omega$,
    ${X_L}=0.01\,\Omega$ y ${X_C}=100\,\Omega$ se conecta a una
    tensión alterna $\overline{U}=100\phase{0^\circ}$ V. Calcular las
    corrientes de cada uno de los elementos.}
		
  En primer lugar, se calculan la conductancia de la resistencia y las
  susceptacias de bobina y condensador:
  \begin{align*}
    \overline{G}&=\dfrac{1}{R}=\dfrac{1}{1}=1\;\si{\siemens}\\
    \overline{B_L}&=-\dfrac{\mathrm{j}}{X_L}=-\dfrac{\mathrm{j}}{0.01}=-\mathrm{j}\,100\;\si{\siemens}\\
    \overline{B_C}&=\mathrm{j}\,X_C=\mathrm{j}\,100\;\si{\siemens}
  \end{align*}
		
  La susceptancia total se calcula como:
  \begin{equation*}
    \overline{B}= \overline{B_L}+\overline{B_C}=-\mathrm{j}\,100+\mathrm{j}\,100=0
  \end{equation*}
		
  Se trata de un circuito antirresonante, siendo la admitancia
  equivalente:
  \begin{equation*}
    \overline{Y}=G+\mathrm{j}\,B=1+\mathrm{j}\,0=1\;\si{\siemens}
  \end{equation*}
		
  La corriente en cada elemento es:
  \begin{align*}
    \overline{I_R}&=\overline{G}\cdot \overline{U}={1\phase{0^\circ}}\cdot {100\phase{0^\circ}}=100\phase{0^\circ}\,\si{\ampere}\\
    \overline{I_L}&=\overline{B_L}\cdot\overline{U}=100\phase{-90^\circ}\cdot 100\phase{0^\circ}=10000\phase{-90^\circ}\,\si{\ampere}\\
    \overline{I_C}&=\overline{B_C}\cdot\overline{U}=100\phase{90^\circ}\cdot 100\phase{0^\circ}=10000\phase{90^\circ}\,\si{\ampere}
  \end{align*}
\end{example}
	
	

\section{Potencia en corriente alterna}\label{sec.potencia_CA}
	
Cuando se conecta una impedancia a una tensión alterna de expresión
$u(t) = U\sqrt{2} \cdot\cos (\omega t)$, la impedancia es recorrida
por una corriente:
\begin{equation*}
  i(t) = I\sqrt{2} \cdot \cos (\omega t -\varphi)
\end{equation*}
siendo $\varphi>0$ si la impedancia es inductiva y $\varphi<0$ si es
capacitiva. La \textbf{potencia instantánea} entregada al circuito
está definida por:
\begin{equation*}
  p(t)=u(t)\cdot i(t)=\left(U\sqrt{2}\cdot \cos (\omega t) \right)\cdot \left(I\sqrt{2} \cdot \cos (\omega t -\varphi)\right)=2\cdot U\cdot I\cdot \cos(\omega t)\cdot\cos(\omega t-\varphi)
\end{equation*}
donde, teniendo en cuenta:
\begin{align*}
  \cos(\alpha-\beta)&=\cos(\alpha)\cdot\cos(\beta)+\sin(\alpha)\cdot\sin(\beta)\\
  \cos(\alpha+\beta)&=\cos(\alpha)\cdot\cos(\beta)-\sin(\alpha)\cdot\sin(\beta)\\[-10pt]
  \cline{1-2}
  \cos(\alpha-\beta)+\cos(\alpha+\beta)&=2\cdot\cos(\alpha)\cdot\cos(\beta)%\rightarrow \sin(\alpha)\cdot\sin(\beta)=\dfrac{1}{2}\cdot \left[ \cos(\alpha-\beta)-\cos(\alpha+\beta)\right]
\end{align*}
y considerando como $\alpha=\omega t$ y $\beta=\omega t-\varphi$:
\begin{align*}
  \alpha-\beta&=\varphi\\
  \alpha +\beta&=2\omega t-\varphi
\end{align*}
por lo que la potencia instantánea resulta:
\begin{equation}\label{eq:pot_inst}
  p(t)=2\cdot U\cdot I\cdot \overbrace{\cos(\omega t)}^{\cos(\alpha)}\cdot\overbrace{\cos(\omega t-\varphi)}^{\cos(\beta)}\Rightarrow \boxed{p(t)
    =U\cdot I \cdot \cos(\varphi)+U\cdot I \cdot\cos(2\omega t-\varphi)}
\end{equation}
Según esta ecuación, la potencia instantánea consta de un {valor
  constante} (el primer término, $U\cdot I\cdot \cos(\varphi)$) y una
componente sinusoidal (el segundo término,
$U\cdot I\cdot \cos(2 \omega t-\varphi)$) de frecuencia $2\omega$
(\textbf{doble} de $u(t)$ o $i(t)$). Como puede verse en la
Figura~\ref{fig.inductivoPuroPotencia}, $p(t)$ es negativa para los
intervalos de tiempo en los que $u(t)$ e $i(t)$ tienen signos
opuestos. En los periodos de tiempo en que la potencia es negativa, la
impedancia \textbf{devuelve energía} a la red, algo que solo es
posible si contiene elementos almacenadores de energía (es decir,
bobinas o condensadores).
	
	\begin{figure}[H]
          \centering
          \includegraphics{../figs/inductivoPuroPotencia.pdf}
          \caption{Ondas de tensión, corriente y potencia
            instantáneas}
          \label{fig.inductivoPuroPotencia}
	\end{figure}
	
	Por tanto, la potencia instantánea cambia con el tiempo y es
        difícil de medir. Si se hace el valor medio de la
        expresión~\eqref{eq:pot_inst} en un periodo, mediante la
        ecuación~\eqref{eq:valor_medio}:
	\begin{equation*}
          P_m=\dfrac{1}{T}\int_{0}^{T}p(t)\,dt=\dfrac{1}{T}\int_{0}^{T}U\,I\,\cos(\varphi)\,dt+\cancelto{0}{\dfrac{1}{T}\int_{0}^{T}U\,I\,\cos(2\omega t-\varphi)\,dt}
	\end{equation*}
	siendo el primer integrando constante y el segundo integrando
        una sinusoide. Dado que el promedio de una sinusoide a lo
        largo de un periodo es nulo (el área bajo la sinusoide durante
        medio ciclo positivo es cancelada por el área bajo ella
        durante el siguiente medio ciclo negativo), este término se
        anula y la potencia promedio se convierte en
        $P=U\,I\,\cos(\varphi)$, que coincide con el término fijo de
        $p(t)$ y es la \textbf{potencia real} consumida en los
        elementos disipativos de la impedancia. El término
        $-U\,I\,\cos(2\omega t-\varphi)$ es el responsable de que
        $p(t)$ fluctúe (oscile) en torno a su valor medio ($P$); de
        ahí que se conozca como \textbf{potencia fluctuante}. Si se
        desarrolla $p(t)$ teniendo en cuenta la relación para el
        coseno de una resta, resulta:
	\begin{align*}
          p(t)&=U\,I\,\cos(\varphi)+U\,I\,\cos(\overbrace{2\omega\,t}^{\alpha}-\overbrace{\varphi}^{\beta})=U\,I\,\cos(\varphi)+\left[ U\,I\,\cos(\varphi)\,\cos(2\omega t) + U\,I\sin(\varphi)\,\sin(2\omega t)\right]=\\
              &=\underbrace{{\color{blue}U\,I\,\cos(\varphi)}\left[1+\cos(2\omega t)\right]}_{p_1(t)} + \underbrace{{\color{red}U\,I\sin(\varphi)}\,\sin(2\omega t)}_{p_2(t)}={\color{blue}P}\left[1+\cos(2\omega t)\right] + {\color{red}Q}\,\sin(2\omega t)
	\end{align*}
	Por tanto, la potencia eléctrica instantánea absorbida por una
        impedancia consta de dos términos variables en el tiempo con
        frecuencia $2\omega$:
	\begin{itemize}
        \item $p_1(t)$, positivo y oscilante en torno al valor medio
          $P=U\,I\,\cos(\varphi)$. Es la potencia instantánea que
          consumen los elementos resistivos, y se denomina
          \textbf{potencia activa} [W]:
          \begin{equation}
            \boxed{P=U\,I\,\cos(\varphi)}
          \end{equation}
        \item $p_2(t)$, es la potencia instantánea que almacena o
          devuelve el circuito (no implica transformación en trabajo
          útil), razón por la que se denomina \textbf{potencia
            entretenida}. La máxima potencia que almacena/devuelve el
          circuito se identifica con la letra $Q$ y se denomina
          \textbf{potencia reactiva}. Al no ser ``potencia
          consumida'', para diferenciarla de $P$, su unidad se
          denomina \textbf{voltamperio reactivo} [var]:
          \begin{equation}
            \boxed{Q=U\,I\,\sin(\varphi)}
          \end{equation}
	\end{itemize}
	
	\subsection{Circuito resistivo}\label{sec.potencia_R}
	Una resistencia, como ya se ha indicado en la
        Sección~\ref{sec:R-puro}, presenta una impedancia
        $\overline{Z_R}=R\phase{0^\circ}\rightarrow
        \varphi=0^\circ$. Por tanto, las potencias activa y reactiva:
	\begin{equation}
          \varphi = 0 \rightarrow
          \boxed{\begin{cases}
            P = U\cdot I = \dfrac{U^2}{R} = I^2\cdot R\\
            Q = 0
          \end{cases}}
      \end{equation}
      puesto que $\cos(0^\circ)=1$ y $\sin(0^\circ)=0$. Dibujando las
      ondas de $u(t)$, $i(t)$ y $p(t)$
      (Figura~\ref{fig.resistivoPotencia}), se observa que $p(t)$
      fluctúa al doble de frecuencia que $u(t)$ e $i(t)$, y que
      \textbf{siempre es positiva}, puesto que tensión y corriente van
      en fase y, por tanto, siempre tienen el mismo signo. Además, su
      valor medio es igual a $P=U\cdot I$.
      \begin{figure}[H]
        \centering \includegraphics{../figs/resistivoPotencia.pdf}
        \caption{Ondas de tensión, corriente y potencia instantáneas
          en un circuito resistivo}
        \label{fig.resistivoPotencia}
      \end{figure}
	
	\subsection{Circuito inductivo puro}\label{sec.potencia_L}
	Una bobina, como ya se ha indicado en la
        Sección~\ref{sec:L-puro}, presenta una impedancia
        $\overline{Z_L}=\omega\cdot L\phase{90^\circ}\rightarrow
        \varphi=90^\circ$. Por tanto, las potencias activa y reactiva:
	\begin{equation}
          \varphi = 90^\circ \rightarrow
          \boxed{\begin{cases}
            P = 0\\
            Q = U\cdot I = \dfrac{U^2}{\omega L} = I^2\cdot  \omega L
          \end{cases}}
      \end{equation}
      puesto que $\cos(90^\circ)=0$ y $\sin(90^\circ)=1$. Dibujando
      las ondas de $u(t)$, $i(t)$ y $p(t)$
      (Figura~\ref{fig.inductivoPotencia}), se observa que $p(t)$
      fluctúa al doble de frecuencia que $u(t)$ e $i(t)$, y que
      \textbf{tiene periodos positivos y negativos}, pasando por los
      ceros de tensión y corriente. Además, su valor medio es nulo,
      coincidiendo con $P=0$.
      \begin{figure}[H]
        \centering \includegraphics{../figs/inductivoPuroPotencia.pdf}
        \caption{Ondas de tensión, corriente y potencia instantáneas
          en un circuito inductivo puro}
        \label{fig.inductivoPotencia}
      \end{figure}
	
	\subsection{Circuito capacitivo puro}\label{sec.potencia_C}
	Un condensador, como ya se ha indicado en la
        Sección~\ref{sec:C-puro}, presenta una impedancia
        $\overline{Z_C}=\frac{1}{\omega\cdot
          C}\phase{-90^\circ}\rightarrow \varphi=-90^\circ$. Por tanto,
        las potencias activa y reactiva:
	\begin{equation}
          \varphi = -90^\circ \rightarrow
          \boxed{\begin{cases}
            P = 0\\
            Q = -U\cdot I = -U^2\cdot \omega C = -\dfrac{I^2}{\omega C}
          \end{cases}}
      \end{equation}
      puesto que $\cos(-90^\circ)=0$ y $\sin(-90^\circ)=-1$. Dibujando
      las ondas de $u(t)$, $i(t)$ y $p(t)$
      (Figura~\ref{fig.capacitivoPotencia}), se observa que $p(t)$
      fluctúa al doble de frecuencia que $u(t)$ e $i(t)$, y que
      \textbf{tiene periodos positivos y negativos}, pasando por los
      ceros de tensión y corriente. Además, su valor medio es nulo,
      coincidiendo con $P=0$.
      \begin{figure}[H]
        \centering
        \includegraphics{../figs/capacitivoPuroPotencia.pdf}
        \caption{Ondas de tensión, corriente y potencia instantáneas
          en un circuito capacitivo puro}
        \label{fig.capacitivoPotencia}
      \end{figure}
	
	\subsection{Triángulo de potencias}
	Supóngase un circuito con carácter inductivo, con resistencia
        $R$ y reactancia $X$. El módulo de la impedancia global del
        circuito puede calcularse como $\overline{Z}=\sqrt{R^2+X^2}$,
        obteniendo un diagrama fasorial conocido como
        \textbf{triángulo de impedancias}. Si se multiplica cada lado
        de dicho triángulo por el módulo de $\overline{I}$, se obtiene
        el \textbf{triángulo de tensiones}, donde el eje $\Re$ es el
        módulo de la tensión en la resistencia $U_R=R\cdot I$ y el eje
        $\Im$ es el módulo de la tensión en la reactancia
        $U_X=X\cdot I$, siendo el módulo de la tensión total
        $U=Z\cdot I=\sqrt{U_R^2+U_X^2}$. Si, de nuevo, vuelve a
        multiplicarse dicho triángulo por el módulo de $\overline{I}$,
        se llega al \textbf{triángulo de potencias}, siendo el eje
        $\Re$ la potencia activa $P=R\cdot I^2$ y, el eje $\Im$, la
        potencia reactiva $Q=X\cdot I^2$. La hipotenusa de dicho
        triángulo es conocida como \textbf{potencia aparente}. Estos
        triángulos se presentan en la
        Figura~\ref{fig.triangulo_potencias}.
	\begin{figure}[H]
          \centering
          \subfloat[Impedancias]{\includegraphics[height=4cm]{../figs/Z_ind_corriente.pdf}}\hfill
          \subfloat[Tensiones
          ]{\includegraphics[height=4cm]{../figs/tension_Z_ind.pdf}}\hfill
          \subfloat[Potencias]{\includegraphics[height=4cm]{../figs/triangulo_potencias.pdf}\label{fig.triangulo_potencias_pqs}}\hfil
          \caption{Triángulo de potencias de un circuito inductivo
            $RL$}
          \label{fig.triangulo_potencias}
	\end{figure}
	
	Se detalla a continuación el significado de cada una de las
        potencias representadas en el triángulo:
	\begin{itemize}
        \item \textbf{Potencia activa.} Cateto contiguo en la
          Figura~\ref{fig.triangulo_potencias_pqs}. Suele denominarse
          únicamente como \textit{potencia}, al tratarse de la que es
          \textbf{realmente consumida} por el circuito. Su unidad es
          el [W]:
          \begin{equation}\label{eq.Pactiva}
            \boxed{P = U\cdot I\cdot\cos(\varphi) = R \cdot I^2}
          \end{equation}
        \item \textbf{Potencia reactiva.} Cateto opuesto en la
          Figura~\ref{fig.triangulo_potencias_pqs}. Es el
          \textbf{valor máximo de la potencia entretenida} (almacenada
          y cedida) por los elementos almacenadores de energía
          (bobinas y condensadores). Se considera positiva $+$ si el
          circuito es inductivo ($\varphi>0^\circ$) y negativa $-$ si
          el circuito es capacitivo ($\varphi<0^\circ$). Su unidad es
          el [var] y se calcula mediante:
          \begin{equation}\label{eq.Qreactiva}
            \boxed{Q = U\cdot I\cdot\sin(\varphi) = X \cdot I^2}
          \end{equation}
        \item \textbf{Potencia aparente.} Hipotenusa en la
          Figura~\ref{fig.triangulo_potencias_pqs}. No es ``potencia
          consumida'' en sentido estricto (excepto cuando
          $\cos(\varphi)=1$), pero representa la \textbf{potencia
            demandada} al generador/red; se denomina \textit{potencia
            aparente}, puesto que es la potencia que, ``en
          apariencia'', la red entrega a las cargas. Para
          diferenciarla de $P$ y $Q$, su unidad es el
          \textbf{voltamperio} [VA] y puede expresarse como:
          \begin{equation}\label{eq.Saparente}
            \boxed{S = U\cdot I= Z \cdot I^2=\sqrt{P^2+Q^2}}
          \end{equation}
          donde la última igualdad se obtiene al aplicar el teorema de
          Pitágoras a la
          Figura~\ref{fig.triangulo_potencias_pqs}. Además, se observa
          que puede expresarse también mediante un \textbf{número
            complejo} (al igual que $\overline{Z}$ y $\overline{U}$):
          \begin{equation}
            \boxed{\overline{S}=P+\mathrm{j}\,Q=\overline{U}\cdot \overline{I^*}=S\phase{\varphi}}
          \end{equation}
          conociéndose entonces como \textbf{potencia compleja}. La
          igualdad $\overline{S}=\overline{U}\cdot \overline{I^*}$ se
          obtiene, considerando que $\overline{U} = U\phase{0}$ y
          $\overline{I} = I\phase{-\varphi}$, de la siguiente forma:
          \begin{equation*}
            \overline{U} \overline{I}^* = U\phase{0} \cdot I\phase{\varphi} = UI\phase{\varphi}= U I (\cos\varphi + \mathrm{j} \sin\varphi) = P + \mathrm{j} Q
          \end{equation*}
          \begin{remark}
            Nótese que la fase de $\overline{S}$ es \textbf{igual} a
            la fase de la impedancia $\overline{Z}$:
            \begin{equation*}
              \varphi_S = \varphi_Z = \varphi
            \end{equation*}
          \end{remark}
          La Figura~\ref{fig.trianguloPotencias} muestra la potencia
          compleja y su descomposición en $P$ y $Q$.
          \begin{figure}[H]
            \centering
            \includegraphics{../figs/trianguloPotencias.pdf}
            \caption{Triángulo de potencias}
            \label{fig.trianguloPotencias}
          \end{figure}
	\end{itemize}
	
	
        \subsection{Resumen de potencia de los elementos pasivos}
	
        Se presenta aquí un resumen de los tipos de potencia consumida
        por cada elemento pasivo básico:
        \begin{itemize}
        \item \textbf{Resistencia:}
          \begin{equation*}
            \varphi = 0^\circ \Rightarrow 
            \begin{cases}
              P_R = R I^2\\
              Q_R = 0\\
              \overline{S_R} = R I^2\phase{0^\circ}
            \end{cases}
          \end{equation*}
		
		
		
          \begin{itemize}
          \item Consume potencia activa
          \item No consume potencia reactiva
          \end{itemize}
		
        \item \textbf{Inductancia:}
          \begin{equation*}
            \varphi = 90^\circ \Rightarrow 
            \begin{cases}
              P_L = 0\\
              Q_L = \omega L I^2\\
              \overline{S}_L = \omega L I^2 \phase{90^\circ}
            \end{cases}
          \end{equation*}
		
		
          \begin{itemize}
          \item No consume potencia activa
          \item Consume potencia reactiva ($Q > 0$)
          \end{itemize}
		
        \item \textbf{Condensador:}
          \begin{equation*}
            \varphi = - 90^\circ \Rightarrow 
            \begin{cases}
              P_L = 0\\
              Q_C = - \omega C U^2\\
              \overline{S}_C = \omega C U^2 \phase{- 90^\circ}
            \end{cases}   
          \end{equation*}
		
		
          \begin{itemize}
          \item No consume potencia activa
          \item Genera potencia reactiva ($Q < 0$)
          \end{itemize}
		
		
        \end{itemize}
	
\subsection{Teorema de Boucherot}\label{sec:boucherot}
El teorema de Boucherot es una consecuencia del principio de
conservación de la energía y, de hecho, puede encontrarse también como
\textit{principio de conservación de la potencia compleja}. Se
demuestra aquí para el caso de un circuito en serie.
	
Sea un circuito en serie formado por 3 impedancias:
$\overline{Z_1}=R_1+\mathrm{j}\,X_1$,
$\overline{Z_2}=R_2+\mathrm{j}\,X_2$ y
$\overline{Z_3}=R_3-\mathrm{j}\,X_3$ (es decir, $\overline{Z_1}$ y
$\overline{Z_2}$ tienen carácter inductivo y $\overline{Z_3}$ tiene
carácter capacitivo). Por comodidad, se supondrá que
$\overline{I}=I\phase{0^\circ}$. Por la 2LK se cumple que:
\begin{equation*}
  \overline{U}=\sum_{i=1}^3 \overline{U_i}\Rightarrow
  \begin{cases}
    U\,\cos(\varphi)=\displaystyle\sum_{i=1}^3 U_i\,\cos(\varphi_i)\\
    U\,\sin(\varphi)=\displaystyle\sum_{i=1}^3 U_i\,\sin(\varphi_i)
  \end{cases}
\end{equation*}
Multiplicando las dos expresiones anteriores por la corriente (la
misma en todo el circuito, al tratarse de una conexión en serie), se
obtienen las relaciones entre las potencias, que puede verse
gráficamente en la Figura~\ref{fig:boucherot}:
\begin{align*}
  U\,I\,\cos(\varphi)&=P_T=\displaystyle\sum_{i=1}^3 U_i\,I\,\cos(\varphi_i)=\displaystyle\sum_{i=1}^3 P_i=\displaystyle\sum_{i=1}^3 R_i\cdot I^2\\
  U\,I\,\sin(\varphi)&=Q_T=\displaystyle\sum_{i=1}^3 U_i\,I\,\sin(\varphi_i)=\displaystyle\sum_{i=1}^3 Q_i=\displaystyle\sum_{i=1}^3 X_i\cdot I^2\\
\end{align*}
\begin{figure}[H]
  \centering
  \includegraphics[width=0.7\linewidth]{../figs/boucherot.pdf}
  \caption{Teorema de Boucherot}
  \label{fig:boucherot}
\end{figure}
	
Es decir, se cumple que:
\begin{itemize}
\item La potencia activa total es la suma aritmética (suma de números
  naturales) de las potencias activas de cada receptor
\item La potencia reactiva total es la suma algebraica (suma de
  números enteros, considerando el signo) de las potencias reactivas
  de cada receptor
\end{itemize}
Estas dos afirmaciones expresan el teorema de Boucherot, de manera que
se cumple que la potencia activa y reactiva total es la suma de las
potencias activas y reactivas individuales (respectivamente) y, por
tanto, que la potencia compleja total es la suma de las potencias
aparentes individuales:
\begin{equation}\label{eq:S_compleja_mono}
  \boxed{\overline{S} =P+\mathrm{j}\,Q= \sum^n_{i = 1} (P_i + jQ_i)=\sum_{i = 1}^{n} \overline{S}_i}
\end{equation}
	
\vspace{4mm}
\begin{example}\label{ej.2-7}
  \textbf{Sabiendo que las fuentes de tensión del circuito de la
    Figura \ref{fig:problema9_garri} vienen definidas por las formas
    de onda $u_1(t)=10\sqrt{2}\cdot \cos(1000\cdot t) \,\si{\volt}$ y
    $u_2(t)=5\sqrt{2}\cdot \sin(1000\cdot t) \,\si{\volt}$, calcular
    las potencias de cada elemento, así como el balance de potencias
    del circuito. }
  \begin{figure}[H]
    \centering
    \includegraphics[width=0.6\linewidth]{../figs/ej7_BT2.pdf}
    \caption{Ejemplo \ref{ej.2-7}}
    \label{fig:problema9_garri}
  \end{figure}
		
  Se convierte $u_1(t)$ en función senoidal, obteniéndose
  $u_1(t)=10\sqrt{2}\cdot \sin(1000\cdot t+\frac{\pi}{2})
  \,\si{\volt}$. Así, los fasores de $\overline{U}_1$ y
  $\overline{U}_2$ son:
  \begin{align*}
    \overline{U}_1&=10\phase{90^\circ}\,\si{\volt}\\
    \overline{U}_2&=5\phase{0^\circ}\,\si{\volt}
  \end{align*}
		
  Se calcula también el valor de $\overline{X}_L$:
  \begin{equation*}
    \overline{X}_L=\mathrm{j}\,\omega L=\mathrm{j}\,\Omega
  \end{equation*}
		
  Con esto, el sistema matricial por el método de mallas es:
  \begin{equation*}
    \begin{bmatrix}
      \vphantom{\dfrac{1}{1}}10\phase{90^\circ} \\[4pt]
      \vphantom{\dfrac{1}{1}}5\phase{180^\circ} 
    \end{bmatrix}
    =
    \begin{bmatrix}
      \vphantom{\dfrac{1}{1}}1-\mathrm{j}\,2 & \quad\mathrm{j}\,2 \\[4pt]
      \vphantom{\dfrac{1}{1}}\mathrm{j}\,2 & -\mathrm{j} \\
    \end{bmatrix}
    \cdot 
    \begin{bmatrix}
      \vphantom{\dfrac{1}{1}}\overline{I}_a\\[4pt]
      \vphantom{\dfrac{1}{1}}\overline{I}_b
    \end{bmatrix}
  \end{equation*}          
  donde el valor de $5\phase{\ang{180}}$ resulta de tomar el negativo
  del fasor $\overline{U}_2$, dado que la corriente de malla
  $\overline{I}_b$ ``entra'' por el polo $+$ de la fuente $u_2$ (tomar
  el negativo de un número complejo en forma polar equivale a
  adelantar su ángulo en $180^\circ$)

  \vspace{3mm} Resolviendo el sistema, se obtiene:
  \begin{align*}
    \overline{I}_a&= 2 + 6\mathrm{j}\,\unit{\ampere}\\
    \overline{I}_b&= 4 + 7\mathrm{j}\,\unit{\ampere}
  \end{align*}          
  y reemplazando en el circuito:
  \begin{align*}
    \overline{I}&=\overline{I}_a= 2 + 6\mathrm{j}\,\unit{\ampere}\\
    \overline{I}_1&=\overline{I}_a - \overline{I}_b= -2 -\mathrm{j}\,\unit{\ampere}\\
    \overline{I}_2&=-\overline{I}_b= -4 - 7\mathrm{j}\,\unit{\ampere}\\
  \end{align*}	

  \vspace{-3mm} Con las corrientes y los valores de las impedancias,
  se calculan las potencias activas y reactivas en los elementos
  pasivos:
  \begin{align*}
    P_R&=R\cdot I^2=\qty{40}{\watt}\\
    Q_L &= X_L\cdot I_2^2=\qty{65}{\var}\\
    Q_C&=X_C\cdot I_1^2=-\qty{10}{\var}
  \end{align*}
  siendo la potencia aparente total consumida por los receptores:
  \begin{equation*}
    \overline{S}=P+\mathrm{j}\,Q= 40 + 55\mathrm{j}\,\unit{\voltampere}
  \end{equation*}
		
  Se calcula también la potencia aparente entregada por las fuentes de
  alimentación:
  \begin{align*}
    \overline{S}_{u1}&=\overline{U}_1\cdot \overline{I}^*= 60 + 20\mathrm{j}\,\unit{\voltampere}\\
    \overline{S}_{u2}&=\overline{U}_2\cdot \overline{I}_2^*= -20 + 35\mathrm{j}\,\unit{\voltampere}\\
    \overline{S}_g &= \overline{S}_{u1} + \overline{S}_{u2} = 40 + 55\mathrm{j}\,\unit{\voltampere}
  \end{align*}
  Comprobamos que coincide con el triángulo de potencias de los
  receptores.
                        
\end{example}
	
\subsection{Medida de potencia: vatímetro}\label{sec:medida_potencia}
	
La medida de potencia activa y reactiva es de gran importancia para
determinar el comportamiento de los circuitos de corriente alterna. La
expresión de la potencia activa $P$ se corresponde con la
\textbf{parte real} de la potencia aparente compleja, según se indicó
en la expresión~\eqref{eq:S_compleja_mono}. Para medir $P$, se utiliza
un instrumento de medida denominado \textbf{vatímetro}, puesto que,
debido al $\cos(\varphi)$, no es posible hacerlo únicamente con
voltímetro y amperímetro. El vatímetro es un equipo que consta de dos
bobinas (una de intensidad, o circuito amperimétrico; y otra de
tensión, o circuito voltimétrico) y, por tanto, cuatro terminales (dos
para la tensión y otros dos para la corriente, como se muestra en la
Figura~\ref{fig:vatimetro_2}), cuya lectura da como resultado
directamente el valor de la potencia:
\begin{equation*}
  W=\Re(\overline{U}_{3,4} \cdot \overline{I}_{1,2}^*)=I_{1,2}\cdot U_{3,4}\cdot \cos\widehat{(I_{1,2}, U_{3,4})}
\end{equation*}%En la actualidad, hay aparatos digitales de medida que captan la tensión y la intensidad y, con estas muestras, pueden obtenerse las potencias activas y reactivas, así como otras magnitudes. Sin embargo, los medidores analógicos de potencia activa y reactiva se siguen utilizando ampliamente. 
	
\begin{figure}[H]
  \centering \includegraphics{../figs/vatimetro_2.pdf}
  \caption{Conexiones del vatímetro}
  \label{fig:vatimetro_2}
\end{figure}
Considerando las bornas 1 y 3 como entradas, si el ángulo formado por
$\overline{I_{1,2}}$ y $\overline{U_{3,4}}$ es menor que $90^\circ$ o
mayor que $270^\circ$, el valor de $\cos\widehat{(I_{1,2}, U_{3,4})}$
será positivo. Pero si el ángulo es mayor que $90^\circ$ o menor de
$270^\circ$, el valor de $\cos\widehat{(I_{1,2}, U_{3,4})}$ será
negativo y el vatímetro tratará de marcar en sentido contrario,
clavándose en $0$ la aguja (en caso de ser analógico) o apareciendo el
signo $-$ (en los digitales). En ese caso, basta con invertir las
conexiones en uno de los dos circuitos (generalmente el de tensión,
por no cortar la continuidad de la alimentación a los receptores) para
obtener una lectura positiva. No obstante, deberá considerarse esta
lectura como \textbf{negativa} a efectos del cómputo de la potencia
total.
	
	
\subsection{Factor de potencia: importancia y
  mejora}\label{sec:mejora_fdp_monofasica}
	
El factor de potencia, $fdp$ o $\cos(\varphi)$, representa la
aportación de potencia activa dentro de la potencia aparente:
\begin{equation}
  \boxed{\cos(\varphi)=\dfrac{P}{S}}
\end{equation}
y es igual al coseno del ángulo entre $\overline{U}$ e
$\overline{I}$. Se dice que:
\begin{itemize}
\item $\cos(\varphi)$ es \textbf{en retraso} cuando el circuito tiene
  carácter inductivo (la intensidad va retrasada respecto a la
  tensión)
\item $\cos(\varphi)$ es \textbf{en adelanto} cuando el circuito tiene
  carácter capacitivo (la intensidad va adelantada respecto a la
  tensión)
\end{itemize}
\begin{remark}
  Como $\cos(\varphi)\leq 1$, su valor supone un límite a la potencia
  activa que se puede consumir en una instalación: ésta será máxima
  para $\cos(\varphi)=1\Rightarrow \varphi=0^\circ$ pero, para cualquier
  otro valor de $\varphi$, aún con los mismos valores de $U$ e $I$, la
  potencia activa consumida será inferior.
\end{remark}
	
Sean dos sistemas con la \textbf{misma tensión y potencia activa},
pero con diferentes factores de potencia
$\cos(\varphi_2) < \cos(\varphi_1)$, lo que implica que $Q_2 > Q_1$
(Figura~\ref{fig:fasorescompensacionreactiva}). Se observa que el
sistema 2 requiere una \textbf{mayor potencia aparente} (es decir, un
generador mayor) para alimentar la misma potencia activa:
\begin{equation*}
  \left(\dfrac{P}{\cos(\varphi_1)} = S_1 \right) < \left( S_2 = \dfrac{P}{\cos (\varphi_2)}\right) 
\end{equation*}
Además, el sistema 2 requiere también una \textbf{mayor sección} de
cable para transportar la misma potencia activa, dado que la sección
del conductor está relacionada con la intensidad que circula por él;
esto implica un coste adicional en la instalación:
\begin{equation*}
  \left(\frac{P}{U \cos (\varphi_1)} = I_1 \right) < \left( I_2 = \frac{P}{U \cos (\varphi_2)}\right) 
\end{equation*}
\begin{figure}[H]
  \centering \includegraphics{../figs/fasorescompensacionreactiva.pdf}
  \caption{Fasores de potencias para dos sistemas con $U$ y $P$, pero
    diferentes $\cos(\varphi)$}
  \label{fig:fasorescompensacionreactiva}
\end{figure}
	
Todo ello hace que las compañías suministradoras penalicen a los
consumidores que tienen factores de potencia bajos, haciéndoles pagar
una tasa adicional. De hecho, en determinados casos, se obliga a
instalar elementos que mejoren dicho factor de potencia (aumenten a
$\cos(\varphi)\approx 1$). Dado que la mayoría de los receptores tienen
un \textbf{carácter inductivo} (máquinas eléctricas industriales),
para mejorar el $\cos(\varphi)$ se conectan \textbf{bancos de
  condensadores} en paralelo con los receptores hasta lograr el factor
de potencia deseado.
	
Sea una carga de potencia activa $P_Z$, potencia reactiva $Q_Z$ y
factor de potencia $\cos(\varphi)$, a la que se le quiere
\textbf{mejorar el factor de potencia} de manera que
$\cos (\varphi') > \cos (\varphi)$, pero manteniendo el valor de
$P_Z$. Así, hay que introducir \textbf{en paralelo} una capacidad $C$
capaz de generar un valor $Q_C$ que compense la reactiva inicial $Q_Z$
hasta el valor deseado $Q'$. Se cumple entonces que:
\begin{align*}
  P' &= P_Z\\
  Q' &= Q_C + Q_Z \quad (Q_C<0\Rightarrow Q' < Q_Z)\\
  \overline{I'} &= \overline{I_C} + \overline{I_Z}
\end{align*}
donde las magnitudes con $'$ hacen referencia a la situación una vez
introducido el condensador (ver
Figura~\ref{fig:circuito_compensacion}). A partir del triángulo de
potencias (Figura~\ref{fig:triangulocompensacionQ}) se deduce que:
\begin{align*}
  Q_Z &= P_Z \tan (\varphi)\\
  Q'&= P_Z \tan (\varphi')
\end{align*}
\begin{equation}\label{eq:compensacion_Q_mono}
  |Q_C| = Q_Z - Q' = P_Z\cdot \left[\tan (\varphi) - \tan (\varphi')\right]=\dfrac{U^2}{X_C}={U^2\,\omega\,C}\rightarrow \boxed{C=\frac{P_Z \left[\tan (\varphi) - \tan (\varphi')\right]}{\omega U^2}}
\end{equation}
	
	
\begin{figure}
  \centering
  \subfloat[Circuito]{\includegraphics[height=3cm]{../figs/circuitocompensacionreactiva.pdf}\label{fig:circuito_compensacion}}
  \hfil \subfloat[Triángulo de
  potencias]{\includegraphics{../figs/trianguloCompensacionQ.pdf}\label{fig:triangulocompensacionQ}}
  \caption{Circuito de compensación de potencia reactiva y triángulo
    de potencias}
  \label{fig:circuitocompensacionreactiva}
\end{figure}
	
\begin{example}\label{ex.condensador_Q}
  \textbf{Una instalación de 230 V, 50 Hz consume una potencia activa
    de $5.2$ kW con un factor de potencia $0.8$ en retraso. Calcular
    la capacidad necesaria para obtener un factor de potencia de
    $0.95$.}
	    
  A partir de la fórmula~\eqref{eq:compensacion_Q_mono}, se obtiene
  que la capacidad es:
  \begin{equation*}
    C=\frac{P \left[\tan (\varphi) - \tan (\varphi')\right]}{\omega U^2}=\dfrac{5200\cdot(\tan(\arccos(0.8))-\tan(\arccos(0.95))}{2\cdot\pi\cdot 50\cdot 230^2}=131.82\,\si{\micro\farad}
  \end{equation*}
	    
  A este mismo resultado se puede llegar sin necesiadad de aprenderse
  la fórmula anterior, de la interpretación de los triángulos de
  potencias. La potencia activa inicial y final connsumida por la
  carga es:
  \begin{equation*}
    P=P'=5200\,\si{\watt}
  \end{equation*}
  La potencia reactiva inicial y final:
  \begin{align*}
    Q&=P\,\tan(\varphi)=5200\cdot\tan(\arccos(0.8))=3900\,\si{\var}\\
    Q'&=P'\,\tan(\varphi')=5200\cdot\tan(\arccos(0.95))=1709.16\,\si{\var}
  \end{align*}
  por lo que la potencia reactiva que genera el condensador es:
  \begin{equation*}
    Q_c=Q'-Q=1709.16-3900=-2190.84\,\si{\var}
  \end{equation*}
  a partir de la cual se determina que la capacidad de dicho
  condensador es:
  \begin{equation*}
    Q_c=X_c\,I^2=\dfrac{U^2}{X_c}=\dfrac{U^2}{\frac{1}{\omega\,C}}\Rightarrow C=\dfrac{Q_c}{\omega\,U^2}=\dfrac{2190.84}{2\cdot\pi\cdot 50\cdot 230^2}=131.83\,\si{\micro\farad}
  \end{equation*}
\end{example}
	
\section{\textsuperscript{TC2} Admitancia e impedancia generalizadas}
\label{sec:admitancia-impedancia-generalizada}

\subsection{Admitancia generalizada}
\label{sec:admitancia-generalizada}

Sea un circuito de corriente alterna adecuado para su análisis
mediante el método de mallas, como el ejemplo recogido en la figura
\ref{fig:mallas-alterna}. Las ecuaciones de este método aplicadas a
este circuito de 3 mallas son:

\begin{equation*}
  \begin{bmatrix}
    {\color{BrickRed}\sum \overline{Z}_{aa}} &  - {\color{MidnightBlue}\sum \overline{Z}_{ab}} & - {\color{MidnightBlue}\sum \overline{Z}_{ca}} \\
    - {\color{MidnightBlue}\sum \overline{Z}_{ab}} & {\color{BrickRed}\sum \overline{Z}_{bb}} & - {\color{MidnightBlue}\sum \overline{Z}_{bc}} \\
    - {\color{MidnightBlue}\sum \overline{Z}_{ca}} & - {\color{MidnightBlue}\sum \overline{Z}_{bc}} &  {\color{BrickRed}\sum \overline{Z}_{cc}}
  \end{bmatrix} \cdot %
  \begin{bmatrix}
    \overline{I}_a\\
    \overline{I}_b\\
    \overline{I}_c\\
  \end{bmatrix} = %
  \begin{bmatrix}
    {\color{OliveGreen}\sum\overline{\epsilon}_a}\\
    {\color{OliveGreen}\sum\overline{\epsilon}_b}\\
    {\color{OliveGreen}\sum\overline{\epsilon}_c}
  \end{bmatrix}
\end{equation*}
siendo:
\begin{itemize}
\item[{\({\color{BrickRed}\sum \overline{Z}_{aa}}\)}] suma de las
  impedancias incluidas en la malla de \(\overline{I}_a\).
\item[{\({\color{MidnightBlue}\sum \overline{Z}_{ab}}\)}] suma de las
  impedancias incluidas en las ramas compartidas por las mallas de
  \(\overline{I}_a\) e \(\overline{I}_b\).
\item[{\({\color{OliveGreen}\sum \overline{\epsilon}_a}\)}] suma
  algebraica de las fuerzas electromotrices de los generadores de la
  malla de \(\overline{I}_a\). Su signo es positivo si contribuyen al
  giro de la corriente.
\end{itemize}

\begin{figure}[H]
  \centering \includegraphics[height=5cm]{../figs/mallas_alterna.pdf}
  \caption{Circuito de corriente alterna con las corrientes de malla
    indicadas.}
  \label{fig:mallas-alterna}
\end{figure}

Si generalizamos estas ecuaciones para un circuito de $n$ mallas
obtenemos el siguiente sistema:
\begin{equation}
  \begin{bmatrix}
    \overline{Z}_{11} & \overline{Z}_{12} & \dots & \overline{Z}_{1n} \\
    \overline{Z}_{21} & \overline{Z}_{22} & \dots & \overline{Z}_{2n} \\
    \vdots & \vdots & \ddots & \vdots \\
    \overline{Z}_{n1} & \overline{Z}_{n2} &  \dots & \overline{Z}_{nn}
  \end{bmatrix} \cdot %
  \begin{bmatrix}
    \overline{I}_1\\
    \overline{I}_2\\
    \vdots \\
    \overline{I}_n\\
  \end{bmatrix} = %
  \begin{bmatrix}
    \overline{\epsilon}_1\\
    \overline{\epsilon}_2\\
    \vdots \\
    \overline{\epsilon}_n
  \end{bmatrix}
\end{equation}

Para obtener una de las corrientes de malla, $I_k$, aplicaremos la
regla de Cramer:
\begin{equation}
  \label{eq:corriente-malla-alterna}
  \overline{I}_k = \overline{\epsilon}_1 \frac{\Delta_{1k}}{|Z|} + \overline{\epsilon}_2 \frac{\Delta_{2k}}{|Z|} + \dots + \overline{\epsilon}_n \frac{\Delta_{nk}}{|Z|}
\end{equation}
siendo \(\Delta_{ij}\) el adjunto del elemento \(ij\) de la matriz de
impedancias:
\[
  \Delta_{ij} = (-1)^{i+j} \cdot |M_{ij}|
\]
donde \(M_{ij}\) es la matriz resultante de eliminar la fila \(i\) y
la columna \(j\) de la matriz de impedancias.

La ecuación \ref{eq:corriente-malla-alterna} indica que las respuestas
del circuito (\(I_k\)) dependen de todas las excitaciones que existan
(\(\epsilon_i\)). A partir de esta observación se puede definir la
admitancia generalizada entre dos partes cualesquiera del circuito:

\begin{equation}
  \label{eq:admitancia-generalizada}
  \overline{Y}_{ik} = \frac{\overline{I}_k}{\overline{\epsilon}_i} = \frac{\Delta_{ik}}{|Z|}
\end{equation}

Asimismo, se puede calcular la impedancia de entrada vista por una
fuente que alimenta un circuito pasivo, para lo que es necesario
cancelar todos los términos $\epsilon_i$ salvo el de la fuente
$\epsilon_1$ en la ecuación \ref{eq:corriente-malla-alterna}:

\begin{equation}
  \overline{I}_1 = \overline{\epsilon}_1 \frac{\Delta_{11}}{|Z|} + 0 \cdot \frac{\Delta_{21}}{|Z|} + \dots + 0 \cdot \frac{\Delta_{n1}}{|Z|}
\end{equation}

\begin{equation}
  \label{eq:impedancia-entrada}
  \overline{Z}_{in} = \frac{\overline{\epsilon}_1}{\overline{I}_1}=  \frac{|Z|}{\Delta_{11}}
\end{equation}

También se puede calcular la impedancia de transferencia de un
circuito, es decir, la impedancia entre dos partes del circuito en las
que la primera está alimentada por una fuente, $\epsilon_j$, y la
segunda está cortocircuitada, $I_k$. Para realizar este calculo, todas
las fuentes independientes salvo $\epsilon_j$ deben estar apagadas y,
por tanto, todos los términos $\epsilon_i$ salvo $\epsilon_j$ son
igual a 0:

\begin{equation}
  \overline{I}_k = 0 \cdot \frac{\Delta_{1k}}{|Z|} + 0 \cdot \frac{\Delta_{2k}}{|Z|} + \dots + \epsilon_j \cdot \frac{\Delta_{jk}}{|Z|} + 0 \cdot \frac{\Delta_{nk}}{|Z|}
\end{equation}

\begin{equation}
  \label{eq:impedancia-transferencia}
  \overline{Z}_{Tjk} = \frac{\overline{\epsilon}_j}{\overline{I}_k}=  \frac{|Z|}{\Delta_{jk}}
\end{equation}


\subsection{Impedancia generalizada}
\label{sec:impedancia-generalizada}

El desarrollo del apartado anterior puede aplicarse a un circuito
adecuado para su resolución por el método de nudos (figura
\ref{fig:nudos-alterna}). En este caso, las ecuaciones son:

\begin{equation*}
  \begin{bmatrix}
    {\color{BrickRed}\sum \overline{Y}_A} & - {\color{MidnightBlue}\sum \overline{Y}_{AB}} & - {\color{MidnightBlue}\sum \overline{Y}_{AC}}\\
    -{\color{MidnightBlue}\sum \overline{Y}_{AB}} & {\color{BrickRed}\sum \overline{Y}_B} & -{\color{MidnightBlue}\sum \overline{Y}_{BC}}\\
    -{\color{MidnightBlue}\sum \overline{Y}_{AC}} & -{\color{MidnightBlue}\sum \overline{Y}_{BC}} & {\color{BrickRed}\sum \overline{Y}_C}
  \end{bmatrix} \cdot%
  \begin{bmatrix}
    \overline{V}_A\\
    \overline{V}_B\\
    \overline{V}_C
  \end{bmatrix} = %
  \begin{bmatrix}
    {\color{OliveGreen}\sum \overline{I}_{gA}}\\
    {\color{OliveGreen}\sum \overline{I}_{gB}}\\
    {\color{OliveGreen}\sum \overline{I}_{gC}}
  \end{bmatrix}
\end{equation*}
donde:
\begin{itemize}
\item[{\({\color{BrickRed}\sum \overline{Y}_A}\)}] Suma de las
  admitancias conectadas al nudo \(A\).
\item[{\({\color{MidnightBlue}\sum \overline{Y}_{AB}}\)}] Suma de las
  admitancias conectadas entre los nudos \(A\) y \(B\).
\item[{\({\color{OliveGreen}\sum \overline{I}_{gA}}\)}] Suma de las
  corrientes de los generadores conectados en el nudo A. El signo es
  positivo si el generador inyecta corriente en el nudo.
\end{itemize}

\begin{figure}[H]
  \centering \includegraphics[height=4cm]{../figs/nudosAC.pdf}
  \caption{Circuito de corriente alterna adecuado para su resolución
    por el método de nudos.}
  \label{fig:nudos-alterna}
\end{figure}

La generalización para un circuito de $n$ nudos es:
\begin{equation*}
  \begin{bmatrix}
    \overline{Y}_{11} & \overline{Y}_{12} & \dots & \overline{Y}_{1n} \\
    \overline{Y}_{21} & \overline{Y}_{22} & \dots & \overline{Y}_{2n} \\
    \vdots & \vdots & \ddots & \vdots \\
    \overline{Y}_{n1} & \overline{Y}_{n2} &  \dots & \overline{Y}_{nn}
  \end{bmatrix} \cdot %
  \begin{bmatrix}
    \overline{V}_1\\
    \overline{V}_2\\
    \vdots \\
    \overline{V}_n\\
  \end{bmatrix} = %
  \begin{bmatrix}
    \overline{I}_{g1}\\
    \overline{I}_{g2}\\
    \vdots \\
    \overline{I}_{gn}
  \end{bmatrix}
\end{equation*}

Para obtener el potencial en un nudo cualquiera, $V_k$, aplicamos
nuevamente la regla de Cramer:
\begin{equation}
  \label{eq:potencial-nudos-alterna}
  \overline{V}_k = \overline{I}_{g1} \frac{\Delta_{1k}}{|Y|} + \overline{I}_{g2} \frac{\Delta_{2k}}{|Y|} + \dots + \overline{I}_{gn} \frac{\Delta_{nk}}{|Y|}
\end{equation}
siendo \(\Delta_{ij}\) el adjunto del elemento \(ij\) de la matriz
\(Y\):
\[
  \Delta_{ij} = (-1)^{i+j} \cdot |M_{ij}|
\]
donde \(M_{ij}\) es la matriz resultante de eliminar la fila \(i\) y
la columna \(j\) de la matriz de admitancias.

La ecuación \ref{eq:potencial-nudos-alterna} indica que las respuestas
del circuito (\(V_k\)) dependen de todas las excitaciones existentes
en el circuito (\(I_{gi}\)). A partir de esta expresión se puede
definir la impedancia generalizada entre dos partes del circuito:

\begin{equation}
  \label{eq:impedancia-generalizada}
  \overline{Z}_{ik} = \frac{\overline{V}_k}{\overline{I}_{gi}} = \frac{\Delta_{ik}}{|Y|}
\end{equation}

La ecuación \ref{eq:potencial-nudos-alterna} también permite calcular
la admitancia de entrada vista por una fuente que alimenta un circuito
pasivo (todas las fuentes salvo la de entrada serían nulas en esa
ecuación):

\begin{equation}
  \overline{V}_1 = \overline{I}_{g1} \frac{\Delta_{11}}{|Y|} + 0 \cdot \frac{\Delta_{21}}{|Y|} + \dots + 0 \cdot \frac{\Delta_{n1}}{|Y|}
\end{equation}
\begin{equation}
  \overline{Y}_{in} = \frac{\overline{I}_{g1}}{\overline{V}_1}=  \frac{|Y|}{\Delta_{11}}
\end{equation}

Finalmente, se puede calcular la admitancia de transferencia de un
circuito, es decir, la admitancia existente entre dos partes del
circuito en las que la primera está alimentada por una fuente,
$I_{gj}$, y la segunda está en circuito abierto, $V_k$. Para realizar
este cálculo, todas las fuentes excepto $I_{gj}$ deben estar apagadas,
por lo que todos los términos $I_{gi}$ salvo $I_{gj}$ deben ser nulos.

\begin{equation}
  \overline{V}_k = 0 \cdot \frac{\Delta_{1k}}{|Y|} + 0 \cdot \frac{\Delta_{2k}}{|Y|} + \dots + I_{gj} \cdot \frac{\Delta_{jk}}{|Y|} + 0 \cdot \frac{\Delta_{nk}}{|Y|}
\end{equation}
\begin{equation}
  \label{eq:admitancia-transferencia}
  \overline{Y}_{Tjk} = \frac{\overline{I}_{gj}}{\overline{V}_k}=  \frac{|Y|}{\Delta_{jk}}
\end{equation}


	

%%% Local Variables:
%%% mode: latex
%%% TeX-master: "TC"
%%% ispell-local-dictionary: "castellano"
%%% End:
