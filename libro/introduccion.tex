\chapter{Introducción}
\label{cha:intro}

% Un circuito eléctrico (o red eléctrica) es un conjunto de elementos
% interconectados de forma que puede circular corriente eléctrica. La
% Teoría de Circuitos es la disciplina que agrupa conceptos y
% herramientas que posibilitan el análisis de estas redes
% eléctricas. Sus fundamentos descansan en la Teoría Electromagnética
% sintetizada en las cuatro ecuaciones de Maxwell. Sin embargo, el
% objetivo principal de la Teoría de Circuitos no es el análisis de
% campos electromagnéticos, sino la relación entre las tensiones y
% corrientes existentes en la red. Este análisis se realiza a partir de
% una simplificación de la Teoría Electromagnética, según la cual se
% considera que las dimensiones del circuito son suficientemente
% pequeñas en términos de la longitud de onda de las señales que
% recorren el circuito como para suponer que éstas se propagan de forma
% instantánea. De esta forma, el análisis de redes representa un sistema
% físico mediante un modelo eléctrico equivalente compuesto con
% elementos concentrados, y emplea un cuerpo de teoremas y métodos para
% obtener las relaciones entre tensiones y corrientes existentes.

Este libro está dedicado al \textbf{análisis} de \textbf{circuitos eléctricos} \textbf{lineales} de \textbf{parámetros concentrados}. Todos los términos de esta declaración de intenciones merecen ser descritos con detalle:

\begin{itemize}
  
\item Un \textbf{circuito eléctrico} es un conjunto de componentes eléctricos interconectados mediante conductores que crean un camino cerrado por el que puede circular corriente eléctrica. Como se estudiará en el capítulo \ref{cha:elementos-circuito-lineal}, dedicado a los componentes eléctricos, un circuito eléctrico puede incluir elementos activos (generadores), que entregan potencia al circuito, o elementos pasivos (receptores), que consumen o almacenan la potencia que circula.
  
\item El \textbf{análisis} (o resolución) de un circuito eléctrico existente persigue determinar sus condiciones de funcionamiento. Este objetivo se alcanza definiendo las ecuaciones correspondientes al circuito en cuestión, y obteniendo los valores de determinadas variables importantes a partir de dichas ecuaciones. Debe distinguirse esta labor de la del \textbf{diseño} (o síntesis) de un circuito eléctrico, en la que el objetivo es definir el circuito eléctrico, es decir, determinar los componentes necesarios y su interconexión, para obtener unas condiciones de funcionamiento. 

\item Todos los circuitos eléctricos que se estudian en este libro se comportan como \textbf{sistemas lineales}. Satisfacen, por tanto, las dos propiedades de linealidad:
  \begin{itemize}
  \item $f(x + y) = f(x) + f(y)$: La respuesta $f$ a la suma de dos entradas $x$ e $y$ es igual a la suma de la respuesta individual a cada una de las entradas.
  \item $f(k \cdot x) = k \cdot f(x)$: la respuesta a una entrada que está multiplicada por un factor de escala $k$ es igual a multiplicar por este factor a la respuesta a la entrada.
  \end{itemize}
  Estas propiedades simplifican el tratamiento de los circuitos, y permiten aplicar técnicas de resolución de ecuaciones lineales. Sin embargo, debe tenerse en cuenta que la linealidad es una aproximación de la realidad que no puede aplicarse de manera indiscriminada a cualquier componente y en cualquier condición. En particular, los dispositivos electrónicos como diodos o transistores tienen un comportamiento marcadamente no lineal, de forma que los circuitos que los contienen no pueden analizarse directamente con las técnicas que aquí se exponen sin realizar previamente aproximaciones de su funcionamiento.
\item Los circuitos eléctricos reales ocupan espacio, las máquinas generadoras y los receptores tienen grandes dimensiones, y los cables conductores se extienden a lo largo de longitudes variopintas. Sin embargo, el análisis de circuitos no toma en consideración las propiedades espaciales de los circuitos ni de sus componentes, sino que los confina a elementos puntuales con un modelo de \textbf{parámetros concentrados}. De esta forma, un conductor real de $\SI{100}{\meter}$ se representará habitualmente como un conductor ideal con una resistencia en su punto medio. Este tratamiento es una simplificación de las ecuaciones del electromagnetismo de Maxwell, y es aplicable únicamente cuando las dimensiones del circuito real son inferiores a la longitud de onda de la señal que circula por el circuito. Por ejemplo, a la frecuencia de $\SI{50}{\hertz}$, habitual en sistemas eléctricos industriales, la longitud de onda de la señal es de $\SI{6000}{\kilo\meter}$. Sin embargo, a la frecuencia de $\SI{2.6}{\giga\hertz}$, característica de la telefonía 4G, la longitud de onda se reduce a $\SI{11.5}{\centi\meter}$. 
\end{itemize}

Sin más preámbulos comencemos el viaje.

La primera parte del trayecto, los conceptos básicos, comenzará con la definición de las variables de interés en el análisis de circuitos (capítulo \ref{cha:variables}). A continuación, en el capítulo \ref{cha:elementos-circuito-lineal} visitaremos los elementos que componen un circuito eléctrico lineal. Continuaremos el camino con las leyes de Kirchhoff en el capítulo \ref{cha:kirchhoff}. Sobre estas leyes se asientan los métodos de resolución de circuitos que se exponen en el capítulo \ref{cha:metodos-analisis}.

La segunda parte, dedicada a la principal aplicación industrial de la corriente eléctrica, la corriente alterna sinusoidal, tiene varias etapas. En la primera de ellas, capítulo \ref{cha:calculo-fasorial},  emplearemos los números complejos para aprender el cálculo fasorial, una herramienta que simplifica el análisis de circuitos eléctricos alimentados con corriente alterna. En la siguiente etapa nos centraremos en los circuitos monofásicos (circuitos alimentados por un único generador), y finalmente trabajaremos con circuitos trifásicos (circuitos alimentados por tres generadores desfasados).

%%% Local Variables:
%%% mode: latex
%%% TeX-master: "TC"
%%% ispell-local-dictionary: "castellano"
%%% End:
