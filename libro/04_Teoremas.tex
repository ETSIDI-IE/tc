\chapter{Teoremas Generales}\label{chap:teoremas}
	
\section{Teoremas de linealidad}
	
Un circuito eléctrico es lineal si los elementos pasivos y activos que
incluye son lineales:
\begin{itemize}
\item Un elemento pasivo es lineal si la relación entre la tensión
  entre sus terminales y la corriente que lo recorre es lineal:
  resistencias, condensadores y bobinas.
\item Una fuente dependiente es lineal si su salida (tensión o
  corriente) tiene una relación lineal con la magnitud del circuito de
  la que depende.
\end{itemize}
Un circuito lineal tiene dos propiedades:
\begin{itemize}
\item Homogeneidad o \textbf{proporcionalidad}: Sea $y(t)$ la
  respuesta de un circuito lineal a una excitación $x(t)$. Si la
  excitación es multiplicada por una constante, $K\cdot x(t)$, la
  respuesta del circuito será modificada por la misma constante,
  $K \cdot y(t)$.
\item Aditividad o \textbf{superposición}: La respuesta de un circuito
  lineal a varias fuentes de excitación actuando simultáneamente es
  igual a la suma de las respuestas que se tendrían cuando actuase
  cada una de ellas por separado:
  \begin{equation*}
    y(t)=\sum_i y_i(t)
  \end{equation*}
\end{itemize}

\subsection{Teorema de Proporcionalidad}
\label{sec:proporcionalidad}

Este teorema es consecuencia de la propiedad de homogeneidad. Sea $y(t)$ la respuesta de un circuito lineal a una excitación $x(t)$.  Si la excitación es multiplicada por una constante, $K \cdot x(t)$, la respuesta del circuito será modificada por la misma constante, $K \cdot y(t)$.

\begin{figure}[H]
  \centering
  \includegraphics{../figs/proporcionalidad.pdf}
  \caption{Teorema de Proporcionalidad}
  \label{fig:superposicion_cc}
\end{figure}

\subsection{Teorema de superposición}
\label{sec:superposicion}

Es consecuencia de la propiedad de superposición de los circuitos
lineales: la respuesta total debida a varias fuentes de
excitación actuando simultáneamente es igual a la suma de las
respuestas que se tendrían cuando actuase cada una de ellas por
separado:
\begin{equation*}
  y(t) = \sum_i y_i(t)
\end{equation*}
    
El teorema dice así: \textit{En una red formada por generadores
  (dependientes e independientes) y resistencias, la corriente en una
  rama o la tensión en un nudo, cuando todos los generadores actúan
  simultáneamente, es la suma de las corrientes o las tensiones que
  crearía cada generador \textbf{independiente} si actuase solo
  (individualmente) sobre el circuito}

\begin{figure}[H]
  \centering
  \includegraphics{../figs/superposicion.pdf}
  \caption{Teorema de Superposicion}
  \label{fig:superposicion}
\end{figure}

El procedimiento para analizar un circuito eléctrico mediante
superposición es el siguiente:
\begin{enumerate}
\item Se ``apagan'' todas las fuentes \textbf{independientes} del
  circuito menos una:
  \begin{itemize}
  \item Las fuentes de tensión se sustituyen por un cortocircuito
    ($U = 0$)
  \item Las fuentes de corriente se sustituyen por un circuito abierto
    ($I = 0$)
  \item Las fuentes \textbf{dependientes} \textbf{no} se modifican
  \end{itemize}
\item Se analiza el circuito, obteniendo la respuesta individual a la
  fuente que permanece activa.
\item Se repite este procedimiento para cada una de las fuentes
  \textbf{independientes} del circuito.
\item La respuesta total del circuito es la suma de las respuestas
  individuales.
\end{enumerate}

Para seguir este procedimiento, hay que tener en cuenta una serie de
observaciones:
\begin{itemize}
\item \textbf{Siempre} hay que aplicar este método cuando en un
  circuito conviven fuentes de \textbf{diferente frecuencia} (ya sea
  diferente pulsación, o porque existan fuentes de corriente continua
  y corriente alterna).
\item En el caso de fuentes de corriente alterna \textbf{sinusoidal},
  la respuesta debe expresarse en el \textbf{dominio del
    tiempo}. \textbf{No} se pueden \textbf{sumar} los \textbf{fasores}
  que corresponden a \textbf{frecuencias diferentes}.
\item En el primer paso del procedimiento, se pueden agrupar las
  fuentes que funcionan a la misma frecuencia y calcular la respuesta
  del circuito en esa frecuencia.
\end{itemize}

Hay que resaltar que el principio de superposición se aplica a \textbf{tensiones} y
\textbf{corrientes}, pero \textbf{no} a potencias, como se comprobará en los siguientes apartados.

\begin{example}\label{ex:superposicion_ca}
  \textbf{El circuito de la Figura~\ref{fig:superposicion1} se
    encuentra en régimen permanente. Determinar analíticamente la
    expresión de $i(t)$}

  Datos:
  $e_1(t) = {50 \sin(1000 t)}\,V;\; e_2(t) = {30}\,V;\; R_1 =
  6\,\Omega;\; R_2 = {6}{\Omega};\; L = {8}{mH};\; C = {10}{\mu F}$

\begin{figure}[H]
  \centering
  \includegraphics[width=0.25\linewidth]{../figs/superposicion1.pdf}
  \caption{Ejemplo~\ref{ex:superposicion_ca}}
  \label{fig:superposicion1}
\end{figure}

Se aplica el teorema de superposición.

\underline{Actúa la fuente de corriente alterna}

\begin{figure}[H]
  \centering
  \includegraphics[width=0.25\linewidth]{../figs/superposicion1_AC.pdf}
  \caption{Circuito cuando actúa la fuente de alterna}
  \label{fig:superposicion1_AC}
\end{figure}

La rama $R_2 - C$ está cortocircuitada y, por tanto, se puede
prescindir de ella:

\begin{align*}
  \overline{Z_1} &= R_1 + \mathrm{j}\,X_L = 6 + \mathrm{j}8\Omega\\
  \overline{I'} &= \dfrac{\overline{\epsilon_1}}{\overline{Z_1}} = \dfrac{\frac{50}{\sqrt{2}}\phase{0^\circ}}{6+\mathrm{j}8}=3.54\phase{-53.1301^\circ} A
\end{align*}

En el dominio del tiempo es:

\begin{equation*}
  i(t)' = 5\sin(1000t - 0.9273) A
\end{equation*}

\underline{Actúa la fuente de corriente continua}

\begin{figure}[H]
  \centering
  \includegraphics[width=0.25\linewidth]{../figs/superposicion1_DC.pdf}
  \caption{Circuito cuando actúa la fuente de continua}
  \label{fig:superposicion1_DC}
\end{figure}

En este circuito se sustituye la bobina por un cortocircuito y el
condensador por un circuito abierto. En consecuencia:

\begin{equation*}
  i''(t) = \dfrac{\epsilon_2(t)}{R_1}=\dfrac{30}{6} = {5}A
\end{equation*}

Por tanto:

\begin{equation*}
  i(t) = i'(t) + i''(t) = 5 + 5\sin(1000t - 0.9273) A
\end{equation*}

\end{example}

\subsubsection{Potencia disipada en una resistencia}
Supóngase que $i(t) = i_1(t) + i_2(t)$, donde
$i_1(t)=\sqrt{2}I_{1}\,\sin(\omega_1\,t)$ e
$i_2(t)=\sqrt{2}I_{1}\,\sin(\omega_2\,t)$. La potencia instantánea
disipada en una resistencia $R$ es:
\begin{align*}
  p(t) &= R \cdot i^2(t) = R \cdot (i_1(t) + i_2(t))^2\\xs
  &=R \cdot (i_1^2(t) + i_2^2(t) + 2\cdot i_1(t) \cdot i_2(t))
\end{align*}
es decir, la potencia instantánea disipada en la resistencia no es la
suma de las potencias individuales, $p(t) \neq p_1(t) + p_2(t)$.

No obstante, si calculamos la potencia activa (que, como vimos, es el valor medio de la potencia instantánea), podemos obtener un resultado útil:
\begin{align*}
  P_R &=\dfrac{1}{T}\int_0^{T}R\cdot i^2(t)\,dt=\dfrac{R}{T}\int_0^T\left[ \sqrt{2}\,I_{1}\,\sin(\omega_1\,t)+\sqrt{2}\,I_{2}\,\sin(\omega_2\,t)\right]^2\,dt=\\
  & = \dfrac{R}{T}\left[ \int_0^T (\sqrt{2}\,I_{1})^2\,\sin^2(\omega_1\,t)\,dt + \int_0^T (\sqrt{2}\,I_{1})^2\,\sin^2(\omega_2\,t)\,dt + 
    \int_0^T 2\,\sqrt{2}\,I_{1}\,\sqrt{2}\,I_{2}\,\underbrace{\sin(\omega_1\,t)\,\sin(\omega_2\,t)}_{\frac{1}{2}\,\left[\cos(A-B)-\cos(A+B)\right]}\,dt \right]
\end{align*}
donde $T$ es el periodo de la función $i(t)$\footnote{$i(t)$ es una
  función periódica no senoidal, y su periodo se corresponde con el
  mínimo común múltiplo de los periodos de las funciones que la
  forman, es decir: $T=k_1\cdot T_1=k_2\cdot T_2$, siendo $k_1$ y
  $k_2$ números enteros.}.

Evaluando cada integral de manera independiente, se obtiene que la
potencia disipada en la resistencia es:
\begin{equation*}
  P_R=R\cdot I_1^2+R\cdot I_2^2 = P_{R1} + R_{R2}
\end{equation*}
En general, si la corriente tuviese más componentes, o alguna de ellas
fuera continua, la potencia disipada por una resistencia es:
\begin{equation}\label{eq.P_R_superposicion}
  \boxed{P_R=R\cdot\left(I_{cc}^2+I_1^2+I_2^2+...+I_n^2 \right)}
\end{equation}
\begin{remark}
  Se presenta aquí el desarrollo de las integrales
  anteriores. Evaluando cada integral de manera independiente:
  \begin{align*}
    &\dfrac{R}{T} \int_0^T (\sqrt{2}\,I_{1})^2\,\sin^2(\omega_1\,t)dt=\dfrac{2\,R\,I_{1})^2}{T}\int_0^T\dfrac{1-\cos(2\,\omega_1\,t)}{2}dt=\\
    &=\dfrac{2\,R\,I_{1}^2}{T}\left[\int_0^T\dfrac{1}{2}dt-\cancelto{0}{\int_0^T \dfrac{\cos(2\,\omega_1\,t)}{2}dt}\right]=R\cdot I_1^2
  \end{align*}
  De forma análoga, la segunda integral da por resultado
  $R\cdot I_2^2$. Y la tercera integral:
  \begin{equation*}
    \int_0^T 2\,\sqrt{2}\,I_{1}\, \sqrt{2}\,I_{2}\,\underbrace{\sin(\omega_1\,t)\,\sin(\omega_2\,t)}_{\frac{1}{2}\,\left[\cos(A-B)-\cos(A+B)\right]}\,dt=\dfrac{1}{2}\,4\,I_1\,I_2\,\left[\int_0^T\cos(\omega_1-\omega_2)\,t\,dt-\int_0^T\cos(\omega_1+\omega_2)\,t\,dt\right]
  \end{equation*}
  haciendo
  $\omega_1-\omega_2=\omega'=2\,\pi\left(\frac{1}{T_1}-\frac{1}{T_2}
  \right)=2\,\pi\left(\frac{k_1}{T}-\frac{k_2}{T}\right)=2\,\pi\,k'\,\frac{1}{T}$,
  donde $k'=k_1-k_2$. Del mismo modo,
  $\omega_1+\omega_2=\omega''=2\,\pi\,k''\,\frac{1}{T}$, donde
  $k''=k_1+k_2$. Con estos cambios, la integral anterior queda:
  \begin{align*}
    &2\,I_1\,I_2\,\left[\int_0^T\cos(\omega'\,t)\,dt-\int_0^T\cos(\omega''\,t)\,dt\right]=\\
    =2\,I_1\,I_2\,&\left[\cancelto{0}{\int_0^T \cos\left(2\pi k'\dfrac{t}{T}\right)\,dt}-\cancelto{0}{\int_0^T\cos\left(2\pi k''\dfrac{t}{T}\right)\,dt}\right]=0
  \end{align*}
\end{remark}


\subsubsection{Potencia entregada por una fuente de tensión}
    
Supóngase que $i(t) = i_1(t) + i_2(t)$, donde
$i_1(t)=\sqrt{2}I_{1}\,\sin(\omega_1\,t+\theta_{i1})$ e
$i_2(t)=\sqrt{2}I_{1}\,\sin(\omega_2\,t+\theta_{i2})$. La potencia
instantánea entregada por una fuente de tensión de fem
$\epsilon(t)=\sqrt{2}\,E\,\sin(\omega_1\,t)$ es:
\begin{equation*}
  p(t) = e(t)\cdot i(t) = \sqrt{2}\,E\, \sqrt{2}\,I_{1}\,\sin(\omega_1\,t)\,\sin(\omega_1\,t+\theta_{i1}) +\sqrt{2}\,E\, \sqrt{2}\,I_{2}\,\sin(\omega_1\,t)\,\sin(\omega_2\,t+\theta_{i2})
\end{equation*} 
En este caso, se tiene que $e(t)$ es una función senoidal de periodo
$T_1$, mientras que $i(t)$ es una función periódica no senoidal de
periodo $T=m.c.m.(T_1,T_2)$.

El valor medio de la potencia instantánea se corresponde con la
potencia real (activa) entregada:
\begin{align*}
  P&=\dfrac{1}{T}\int_0^{T}\left( \sqrt{2}\,E\,\sqrt{2}\, I_{1}\,\underbrace{\sin(\omega_1\,t)\,\sin(\omega_1\,t+\theta_{i1}}_{\frac{1}{2}\left[\cos(A-B)-\cos(A+B)\right]}) + \sqrt{2}\,E\,\sqrt{2}\, I_{2}\,\cancelto{\text{0, si $\theta_{i2}=0$}}{\sin(\omega_1\,t)\,\sin(\omega_2\,t+\theta_{i2})}\right)dt=\\
  =&\dfrac{1}{T}\int_0^T \cancel{2}\,E\,I_{1}\dfrac{\cos(-\theta_{i1})-\cos(2\,\omega_1\,t+\theta_{i1})}{\cancel{2}}dt =\dfrac{1}{T}\int_0^T E\,I_{1}\,{\cos(\theta_{i1})}\,dt-\dfrac{1}{T}\int_0^T E\,I_{1}\,\cancelto{0}{{\cos(2\,\omega_1+\theta_{i1})}}\,dt
\end{align*}
por lo que la potencia entregada por la fuente de tensión es:
\begin{equation}\label{eq.P_E_superposicion}
  \boxed{P=E\,I_1\,\cos(\theta_{i1})}
\end{equation}
es decir, que a efectos de potencia, solo actúa la componente de la
intensidad que tiene \textbf{la misma frecuencia} que el generador de
tensión.

\subsubsection{Potencia entregada por un generador de intensidad}

Sea la corriente entregada por la fuente
$i(t)=\sqrt{2}\,I_g\,\sin(\omega_1\,t)$, y la tensión
$u(t)=\sqrt{2}\,U_{1}\,\sin(\omega_1\,t+\theta_{u1})+\sqrt{2}\,U_{2}\,\sin(\omega_2\,t+\theta_{u2})$. La
potencia instantánea es:
\begin{align*}
  p(t) = u(t)\cdot i(t) = \sqrt{2}\,U_{1}\, \sqrt{2}\,I_g\,\sin(\omega_1\,t)\,\sin(\omega_1\,t+\theta_{u1}) + \sqrt{2}\,U_{2}\, \sqrt{2}\,I_g\,\sin(\omega_1\,t)\,\sin(\omega_2\,t+\theta_{u2})
\end{align*} 
Como en los casos anteriores, $T=m.c.m.(T_1,T_2)$, y la potencia
entregada por el generador es:
\begin{equation*}
  P=\dfrac{1}{T}\int_0^T \left[\sqrt{2}\,U_{1}\, \sqrt{2}\,I_g\,\sin(\omega_1\,t)\,\sin(\omega_1\,t+\theta_{u1})+\sqrt{2}\,U_{2}\, \sqrt{2}\,I_g\,\sin(\omega_1\,t)\,\sin(\omega_2\,t+\theta_{u2}) \right] dt
\end{equation*}
integral que, al desarrollarla, incluye los mismos términos nulos que
en los casos anteriores, por lo que el resultado es:
\begin{equation}\label{eq.P_I_superposicion}
  \boxed{P=U_1\,I_g\,\cos(\theta_{u1})}
\end{equation}
    
\begin{remark}
  Las integrales que se han ido anulando son \textbf{señales
    ortogonales en un periodo}. Dos señales son ortogonales si cumplen
  la siguiente ecuación:
  \begin{equation*}
    <f_1, f_2>_T = \int_T f_1(t) \cdot f_2(t) dt = 0
  \end{equation*}
  Son señales ortogonales todas las funciones sinusoidales de
  diferente frecuencia, así como las funciones sinusoidales con
  funciones continuas. Por tanto, en los casos que aparecen en teoría
  de circuitos, las
  expresiones~\eqref{eq.P_R_superposicion}--\eqref{eq.P_I_superposicion}
  serán siempre válidas.
\end{remark}

	
\begin{example}\label{ex:superposicion_ca_potencia}
  \textbf{Determina el balance de potencias del circuito de la Figura~\ref{fig:superposicion1} del ejemplo \ref{ex:superposicion_ca_potencia}.}

\underline{Actúa la fuente de corriente alterna}

El balance de potencias es:

\begin{align*}
  P_{R1}' &= I_1^2\cdot R_1= 3.54^2\cdot 6= {75.19}{W}\\
  P_{R2}' &= {0} W\\
  P_{\epsilon1} &= \epsilon_1 \cdot {I'}\cdot\cos(\theta_{i'}) = \dfrac{50}{\sqrt{2}}\cdot 3.54\cdot\cos(-53.1301) = 75.19 W
\end{align*}

\underline{Actúa la fuente de corriente continua}

El balance de potencias es:

\begin{align*}
  P_{R1}'' &= I_2^2 \cdot R_1=5^2\cdot 6 = {150} W\\
  P_{R2}'' &= {0} W\\
  P_{\epsilon2} &= \epsilon_2 \cdot I''\cdot\cos(\theta_{i''}) = 30\cdot 5\cdot\cos(0)  = {150} W
\end{align*}

Como las señales son ortogonales, se puede hacer el balance de
potencias conjunto con los dos circuitos:

\begin{align*}
  P_{R1} &= P_{R1}' + P_{R1}'' = 75.19+150={225.19} W\\
  P_{R2} &= P_{R2}' + P_{R2}'' = 0+0 = {0} W\\
  P_{\epsilon} &= P_{\epsilon1} + P_{\epsilon2} = 75.19 + 150={225.19} W\\
\end{align*}
\end{example}
	
\section{Teoremas de Thévenin y Norton}
Cuando el interés en el estudio de un circuito se fija en una parte
del mismo (por ejemplo, en una rama), es interesante poder separar
esta rama del resto de la red para no tener que resolver el circuito
completo cada vez que se modifican los parámetros de dicha rama.  Los
teoremas de Thévenin y Norton constituyen dos procedimientos para
sustituir el resto de la red, y hacer más simple el cálculo de
tensiones, corrientes, etc. en la rama que se desea estudiar de un
modo específico. Por tanto, resuelve el problema de sustituir una red
compleja por un circuito equivalente más simple, evitando así cálculos
repetitivos innecesarios.

\subsubsection{Teorema de Thévenin}
Enunciado por León Charles Thévenin en 1883, un ingeniero de
telégrafos francés, este teorema dice lo siguiente: \textit{cualquier
  \textbf{red lineal} compuesta por elementos pasivos y activos
  (dependientes o independientes) se puede sustituir, desde el punto
  de vista de unos terminales externos $A-B$, por una fuente de
  tensión $\overline{\epsilon}_{th}$ (generador de Thévenin) y una
  impedancia en \textbf{serie} $\overline{Z}_{th}$ (impedancia de
  Thévenin)}. La Figura~\ref{fig:thevenin_ca} muestra el circuito
equivalente Thévenin de una red lineal.
\begin{figure}[H]
  \centering \subfloat[Red
  lineal]{\includegraphics[width=0.32\linewidth]{../figs/EquivalenteThevenin.pdf}}\hfil
  \subfloat[Equivalente
  Thévenin]{\includegraphics[width=0.29\linewidth]{../figs/EquivalenteThevenin2.pdf}}
  \caption{Equivalente de Thévenin}
  \label{fig:thevenin_ca}
\end{figure}
     
    
Si ambos circuitos han de ser equivalentes, deberán dar los mismos
valores de tensión y corriente a la impedancia de carga $Z_L$. Entre
todos los valores posibles de $Z_L$, se analizan los dos casos
extremos ($Z_L \to \infty$ y $Z_L=0$):
\begin{itemize}
\item $Z_L \to \infty$: este caso significa físicamente
  \textbf{desconectar} la impedancia de carga del circuito. En esta
  situación, el dipolo de la red lineal dará una tensión en vacío o en
  circuito abierto ($U_0$) siendo $I=0$, que deberá ser idéntica a la
  que debe dar el circuito del equivalente Thévenin, donde la tensión
  entre los terminales $A-B$ es igual a $\epsilon_{th}$, ya que la
  caída de tensión en $\overline{Z}_{th}$ será nula. Por consiguiente, \textbf{el
    valor de $\epsilon_{th}$ de la red equivalente es igual a la
    magnitud $U_0$ de la red lineal que se obtiene entre los
    terminales de salida $A-B$ al desconectar la carga y dejar el
    circuito abierto}.
\item $Z_L=0$: Este caso representa un cortocircuito entre los
  terminales externos $A-B$. Denominando $I_{sc}$ a la corriente que
  circula por este cortocircuito, debe obtenerse la misma $I_{sc}$ en
  el equivalente de Thévenin, resultando, por tanto:
  \begin{equation}\label{eq.Zth}
    \overline{I}_{sc}=\dfrac{\overline{\epsilon}_{th}}{\overline{Z}_{th}}\Rightarrow
    \boxed{\overline{Z}_{th}=\dfrac{\overline{\epsilon}_{th}}{\overline{I}_{sc}}}
  \end{equation}
  es decir, \textbf{el valor de $\overline{Z}_{th}$ se obtiene como el cociente
    entre la tensión que da la red en vacío y la corriente de
    cortocircuito}. Si los generadores del circuito son todos
  independientes, el cálculo de la impedancia de Thévenin es más simple
  que lo expresado en la fórmula~\eqref{eq.Zth}, y representa el
  \textbf{valor de la impedancia que se observa entre los terminales
    $A-B$} de salida cuando se anulan los generadores internos del
  circuito (es decir, se cortocircuitan las fuentes de tensión y se
  abren las de corriente). Téngase en cuenta que, si se anulan los
  generadores, al no existir fuentes de excitación, darán lugar a una
  tensión de Thévenin $\epsilon_{th}=0$ y, si se anula
  $\epsilon_{th}$, la impedancia que se observa entre los terminales
  $A-B$, quitando la carga, coincide con $\overline{Z}_{th}$.
  \begin{remark}
    En ocasiones, calcular $I_{sc}$ puede ser complicado. Otra manera
    de determinar el valor de $\overline{Z}_{th}$ es incluyendo entre $A-B$ una
    fuente de prueba, de fem $\epsilon_0$, y haciendo el cociente
    entre:
    \begin{equation*}
      \overline{Z}_{th}=\dfrac{\overline{\epsilon}_0}{\overline{I}_{0}}
    \end{equation*}
    donde $I_{0}$ es la corriente suministrada por la fuente de
    prueba, que será función de $\epsilon_0$.
  \end{remark}
\end{itemize}
     
     
\subsubsection{Teorema de Norton}
Enunciado por el ingeniero estadounidense Edward Lawry Norton, de los
Laboratorios Bell, que lo publicó en un informe interno en el año
1926, se trata de la versión dual del teorema de Thévenin, diciendo lo
siguiente: En este caso, el teorema se generaliza, de manera que
\textit{cualquier \textbf{red lineal} compuesta por elementos pasivos
  y activos (dependientes o independientes) se puede sustituir, desde
  el punto de vista de unos terminales externos $A-B$, por una fuente
  de corriente $\overline{I_{N}}$ (generador de Norton) y una
  impedancia en \textbf{paralelo} $\overline{Z_{N}}$ (impedancia de
  Norton)}. Al circuito de la Figura~\ref{fig:norton1} se le denomina
equivalente Norton, y si se compara con el equivalente Thévenin, se
observa que no es más que el que resulta de sustituir una fuente de
tensión por una de corriente.
\begin{figure}[H]
  \centering \subfloat[Red
  lineal]{\includegraphics[width=0.35\linewidth]{../figs/EquivalenteThevenin.pdf}}\hfil
  \subfloat[Equivalente
  Norton]{\includegraphics[width=0.3\linewidth]{../figs/EquivalenteNorton.pdf}\label{fig:norton1}}
  \caption{Equivalente de Norton}
\end{figure}

Al circuito de la Figura~\ref{fig:norton1} se le denomina
equivalente Norton, y si se compara con el equivalente Thévenin, se
observa que no es más que el que resulta de sustituir una fuente de
tensión por una de corriente donde se cumple que:
\begin{equation}
  \boxed{\overline{I}_N=\dfrac{\overline{\epsilon}_{th}}{\overline{Z}_{th}}= I_{sc}} \qquad\qquad \boxed{\overline{Z}_N=\overline{Z}_{th}}
\end{equation}
de donde se deduce que el generador de corriente de Norton es igual a
la corriente que se obtiene en la red lineal al juntar sus terminales
($Z_L=0$) y que la impedancia de Norton es el cociente entre la
tensión de vacío ($Z_L \to \infty$) y la corriente de cortocircuito de la
red (al igual que la impedancia de Thévenin).
\begin{remark}
  Gracias a la equivalencia de fuentes
  (expresión~\eqref{eq.equivalencia_fuentes}), una vez obtenido uno de
  los equivalentes se puede obtener el otro mediante una
  transformación.
\end{remark}

\begin{example}\label{ex:Th_cc}
  \textbf{Determinar el equivalente de Thévenin del circuito de la
    Figura~\ref{fig:ej_Th_cc} visto desde los terminales $A-B$, y la
    potencia que se disiparía si se conectase una resistencia de
    5$\Omega$.}
  \begin{figure}[H]
    \centering \includegraphics{../figs/ej_Th_cc1.pdf}
    \caption{Ejemplo~\ref{ex:Th_cc}}
    \label{fig:ej_Th_cc}
  \end{figure}
    
  \underline{Cálculo de $\epsilon_{th}$}
    
  Se corresponde con la caída de tensión en vacío que habría entre
  esos terminales $A-B$. Resolviendo el circuito, se obtiene que las
  corrientes son:
  \begin{align*}
    I_1&=1\,A\\
    I_2&=I_3=-14/23\,A\\
    I_4&=I_5=-9/23\,A
  \end{align*}
  y, por la 2LK:
  \begin{equation*}
    \epsilon_{th}=-I_4\,\dfrac{2}{3}+I_2\,2=\dfrac{9}{23}\cdot\dfrac{2}{3}-\dfrac{14}{23}\cdot 2=-\dfrac{22}{23}\,V
  \end{equation*}
    
  \underline{Cálculo de $R_{th}$}
    
  Al no haber fuentes dependientes, se puede obtener directamente
  calculando la resistencia equivalente vista desde los terminales
  desde los que se calcula el equivalente Thévenin cuando se
  ``anulan'' todas las fuentes. En este caso, dicha resistencia es:
  \begin{equation*}
    R_{th}=\dfrac{(\frac{2}{3}+2)\cdot(1+4)}{(\frac{2}{3}+2)+(1+4)}=\dfrac{40}{23}\Omega
  \end{equation*}
    
  \underline{Potencia de una $R=5\Omega$}
    
  Si se conecta una resistencia de 2 $\Omega$, la resistencia
  equivalente es:
  \begin{equation*}
    R_{eq}=\dfrac{40}{23}+2=\dfrac{155}{23}\Omega
  \end{equation*}
  por lo que la intensidad que circula por el circuito (alternando la
  polaridad de la fuente):
  \begin{equation*}
    I=\dfrac{\epsilon_{th}}{R_{eq}}=\dfrac{\frac{22}{23}}{\frac{155}{23}}=\dfrac{22}{155} A
  \end{equation*}
  siendo la potencia disipada por la resistencia:
  \begin{equation*}
    P=R\, I^2=5\cdot \left( \dfrac{22}{155}\right)^2=0.10\,W
  \end{equation*}
\end{example}

% \begin{example}\label{ex:th_cc_dep}
%   \textbf{Determinar el equivalente de Thévenin del circuito de la
%   Figura~\ref{fig:eq_Th_cc_dep} entre los terminales $A-B$; a partir
%   de él, hallar el valor de $U_0$.}
%   \begin{figure}[H]
%     \centering \includegraphics{../figs/eq_Th_cc_dep.pdf}
%     \caption{Ejemplo~\ref{ex:th_cc_dep}}
%     \label{fig:eq_Th_cc_dep}
%   \end{figure}
    
%   \underline{Cálculo de $\epsilon_{th}$}
    
%   Se corresponde con la caída de tensión en vacío entre los
%   terminales $A-B$, correspondiente al circuito de la
%   Figura~\ref{fig:eq_Th_cc_dep1}. Planteando el método de nudos:
%   \begin{figure}[H]
%     \centering \includegraphics{../figs/eq_Th_cc_dep1.pdf}
%     \caption{Cálculo de $\epsilon_{th}$}
%     \label{fig:eq_Th_cc_dep1}
%   \end{figure}
    
%   \begin{equation*}
%     \dfrac{U_C-2\,I_x}{1}+\dfrac{U_C}{2}+\dfrac{U_C-12}{2}=0
%     \end{equation*}
%     de donde se sabe que $I_x=\frac{U_C-12}{2}$, se obtiene que:
%     \begin{equation*}
%         U_C=-6\,V
%     \end{equation*}
%     por lo que la tensión $U_{AB}=\epsilon_{th}$:
%     \begin{equation*}
%         \epsilon_{th}=U_{AB}=U_C-12=-6-12=-18\,V
%     \end{equation*}
    
%     \underline{Cálculo de $R_{th}$}
    
%     Al existir un generador dependiente, se debe determinar $R_{th}$
%     mediante un generador de prueba de valor $\epsilon_0$ situado en
%     los terminales $A-B$, y ``anulando'' la fuente dependiente, como
%     en la Figura~\ref{fig:eq_Th_cc_dep2}.
%     \begin{figure}[H]
%       \centering \includegraphics{../figs/eq_Th_cc_dep2.pdf}
%       \caption{Cálculo de $R_{th}$}
%       \label{fig:eq_Th_cc_dep2}
%     \end{figure}
    
%     E
    
    
%   \end{example}

\section{Teorema de la máxima transferencia de potencia}
\label{sec:teorema-max_potencia}

En equipos de transmisión--recepción, en sistemas de telecomunicación,
en amplificadores, etc., interesa que la potencia de la señal a la
salida sea máxima, es decir, que se entregue la máxima potencia a la
carga conectada en los terminales de salida.  Este teorema responde a
la siguiente pregunta: ¿\textbf{cuál es el valor de $\overline{Z}_L$ para que,
  al conectarla entre los terminales $A-B$, el circuito entregue la
  máxima potencia disponible}?

Aplicando el teorema de Thévenin (se llegaría a la misma conclusión si
se hiciera con Norton), se convierte el circuito activo en un
generador de fem $\overline{\epsilon}_{th}$ en serie con una
impedancia $\overline{Z}_{th}$ y la impedancia $\overline{Z}_L$
conectada entre $A-B$, como se muestra en la
Figura~\ref{fig:equivalenteThevenin0_ca}.
\begin{figure}[H]
  \centering \includegraphics{../figs/EquivalenteThevenin0.pdf}
  \caption{Ecuaciones del teorema de la máxima transferencia de
    potencia}
  \label{fig:equivalenteThevenin0_ca}
\end{figure}

De manera general, se tiene que:
\begin{align*}
  \overline{Z}_{th} &= R_{th} + \mathrm{j}\,X_{th}\\
  \overline{Z}_L &= R_L + \mathrm{j}\,X_L
\end{align*}
Por tanto, la corriente que circula por el circuito es:
\begin{equation*}
  \overline{I} = \frac{\overline{\epsilon}_{th}}{\overline{Z}_{th} + \overline{Z}_L}
\end{equation*}
cuyo módulo es

\begin{equation*}
  I=\frac{\epsilon_{th}}{\sqrt{(R_L+R_{th})^2+(X_L+X_{th})^2}}
\end{equation*}

Por definición, la potencia consumida por la carga $Z_L$ (la que hay que
maximizar), es:
\begin{equation*}
  P_L= I^2 \cdot R_L\Rightarrow P_L = \dfrac{\epsilon_{th}^2}{{(R_L+R_{th})^2+(X_L+X_{th})^2}} \cdot R_L
\end{equation*}
y, teniendo en cuenta las condiciones para obtener el valor máximo
$\left(\diffp{P_L}{X_L} = 0;\;\; \diffp{P_L}{R_L} = 0\right)$, se
obtiene que:
\begin{itemize}
\item \textbf{Condición de la reactancia:}
  \begin{equation}\label{eq.XL_maxpotencia}
    \diffp{P_L}{X_L} = \epsilon^2_{th} \cdot R_L \cdot \left[\frac{-1}{\left((R_L + R_{th})^2 + (X_L + X_{th})^2\right)^2} \cdot 2 \cdot (X_L + X_{th})\right]=0\Rightarrow \boxed{X_L = - X_{th}}
  \end{equation}
\item \textbf{Condición de la resistencia:} simplificando la expresión
  de la potencia al tener en cuenta la
  ecuación~\eqref{eq.XL_maxpotencia}, y calculando la derivada
  parcial:
  \begin{equation}\label{eq.R_maxpotencia}
    \diffp{P_L}{R_L} = \epsilon^2_{th} \cdot \left[\frac{1}{(R_L + R_{th})^2} - 2 \cdot \frac{R_L}{(R_L + R_{th})^3}\right]= \frac{\epsilon^2_{th} \cdot (R_{th} - R_L)}{(R_L + R_{th})^3}=0\Rightarrow \boxed{R_L = R_{th}}
  \end{equation}
\end{itemize}
Por tanto, la impedancia de carga que hay que conectar entre los
terminales $A-B$ del equivalente de Thévenin del circuito lineal para
obtener la máxima potencia disponible es:
\begin{equation}
  \boxed{\overline{Z}_L = \overline{Z}_{th}^*=R_{th}-\mathrm{j}\,X_{th}}
\end{equation}
siendo la máxima potencia disponible en la carga:
\begin{equation}
  \left.
    \begin{matrix}
      \overline{Z}_L = \overline{Z}_{th}^*\\
      P_L = \dfrac{\epsilon_{th}^2}{{(R_L+R_{th})^2+(X_L+X_{th})^2}} \cdot R_L
    \end{matrix} \right\}\rightarrow
  \boxed{P_L = \frac{\epsilon^2_{th}}{4 R_{th}}}
\end{equation}

\begin{remark}
  Los generadores equivalentes de Thévenin, Norton y los resultados
  del teorema de la máxima transferencia de potencia solo son válidos
  para la frecuencia a la que se obtienen.
\end{remark}

\begin{example}\label{ex:th_ca}
  \textbf{En el circuito de la Figura~\ref{fig:thevenin6}, calcular:
    \begin{itemize}
    \item La fuerza electromotriz del generador equivalente de
      Thévenin respecto de A y B, \(\overline{\epsilon}_{th}\)
    \item La impedancia del generador equivalente de Thévenin respecto
      de A y B, \(\overline{Z}_{th}\)
    \item La impedancia de carga que se debe conectar entre A y B para
      conseguir la máxima potencia disponible
    \item La potencia activa entregada entre A y B cuando se conecta
      cada una de las siguientes impedancias de carga:
      \begin{itemize}
      \item $\overline{Z}_L = \overline{Z}_{th}$
      \item $\overline{Z}_L = R_{th}$ (parte resistiva de
        $\overline{Z}_{th}$)
      \item $\overline{Z}_L = \mathrm{j} X_{th}$ (parte reactiva de
        $\overline{Z}_{th}$)
      \item Impedancia calculada en el apartado anterior
      \end{itemize}
    \end{itemize}}

  Datos:
  $\; \overline{Z}_1 = 3 + \mathrm{j}4\,\Omega;\; \overline{Z}_2 = 2 +
  \mathrm{j}\,\Omega;\; \overline{\epsilon}_g =
  10\phase{30^{\circ}}\,\si{\volt}; \; \overline{I}_g =
  2\phase{15^{\circ}}\,\si{\ampere};\; \beta = 5\,\Omega$

\begin{figure}[H]
  \centering \includegraphics{../figs/thevenin6.pdf}
  \caption{Ejemplo~\ref{ex:th_ca}}
  \label{fig:thevenin6}
\end{figure}

Para calcular la \textit{fem} del generador equivalente de Thévenin,
se calcula la tensión en circuito abierto. Por 1LK, se tiene que:
\begin{equation*}
  \overline{I}_g + \overline{I}_Z + \overline{I}_{Z1} = 0 
\end{equation*}
y, aplicando 2LK:
\begin{align*}
  &- \overline{I}_Z \cdot \overline{Z}_2 = \overline{\epsilon}_g - \overline{I}_{Z1} \cdot \overline{Z}_1\\
  &\overline{U}_{AB} = - \beta \cdot \overline{I}_Z - \overline{I}_Z \cdot \overline{Z}_2
\end{align*}
Combinando estas ecuaciones, se obtiene:
\begin{equation*}
  \overline{U}_{AB}=\overline{\epsilon}_{th} = (\beta + \overline{Z}_2) \, \dfrac{\overline{\epsilon}_g + \overline{I}_g \cdot \overline{Z}_1}{\overline{Z}_1 + \overline{Z}_2} = (5+2+\mathrm{j})\cdot\dfrac{10\phase{30^\circ}+[(2\phase{15^\circ})\cdot (3+\mathrm{j}4)]}{3 + \mathrm{j}4+2 + \mathrm{j}}= \boxed{\vphantom{\frac{a}{a}} 18.90\phase{12.195^\circ}\,\si{\volt}} 
\end{equation*}

Para calcular la impedancia equivalente de Thévenin, se apagan las
fuentes independientes y se conecta un generador de prueba en A-B,
como en la Figura~\ref{fig:thevenin6_zth}:
\begin{figure}[H]
  \centering
  \includegraphics[width=0.5\linewidth]{../figs/thevenin6_fuenteprueba.pdf}
  \caption{Cálculo de $\overline{Z}_{th}$}
  \label{fig:thevenin6_zth}
\end{figure}

Por la 1LK:
\begin{equation*}
  \overline{I}_0 + \overline{I}_Z + \overline{I}_{Z1} = 0
\end{equation*}
y por la 2LK:
\begin{align*}
  &\overline{I}_Z \cdot \overline{Z}_2 = \overline{I}_{Z1} \cdot \overline{Z}_1\\
  &\overline{\epsilon}_0 = - \beta \cdot \overline{I}_Z - \overline{I}_Z \cdot \overline{Z}_2
\end{align*}

Combinando las ecuaciones, se llega a que:
\begin{equation*}
  \overline{Z}_{th} = \dfrac{\overline{\epsilon}_0}{\overline{I}_0} = \dfrac{\overline{Z}_1\,(\beta + \overline{Z}_2)}{\overline{Z}_1+\overline{Z}_2}=\dfrac{(3+\mathrm{j}4)\cdot (5+1+\mathrm{j})}{3+\mathrm{j}4+1+\mathrm{j}} = \boxed{\vphantom{\frac{a}{a}} 5\phase{16.2602^\circ}\,\Omega}
\end{equation*}

Finalmente, para calcular la potencia en la carga A-B, esta depende de
la carga conectada:

\begin{equation*}
  P_{AB} = \dfrac{\epsilon_{th}^2}{{(R_L+R_{th})^2+(X_L+X_{th})^2}} \cdot R_L
\end{equation*}

\begin{itemize}
\item Para $\overline{Z}_L = \overline{Z}_{th}$,
  $\;P_{AB} = \qty{17.15}{\watt}$
\item Para $\overline{Z}_L = R_{th}$, $\;P_{AB} = \qty{18.22}{\watt}$
\item Para $\overline{Z}_L = \mathrm{j}X_{th}$,
  $\;P_{AB} = \qty{0}{\watt}$
\item Para $\overline{Z}_L = \overline{Z}_{th}^*$,
  $\;P_{AB} = \qty{18.61}{\watt}$
\end{itemize}

Se comprueba que el máximo valor se obtiene cuando se conecta la
impedancia de Thévenin conjugada.

\end{example}
	

%%% Local Variables:
%%% mode: latex
%%% TeX-master: "TC"
%%% ispell-local-dictionary: "castellano"
%%% End:
