\vspace*{\fill}

\rule[.5ex]{\linewidth}{1pt} 

Este libro desarrolla parte del temario de las asignaturas de Teoría de Circuitos impartidas en los grados de la \href{http://www.etsidi.upm.es/}{ETSIDI} - \href{http://www.upm.es/}{UPM}. Estas asignaturas impartidas en la ETSIDI cubren el contenido de la Teoría de Circuitos en tres bloques diferenciados:
\begin{itemize}
\item La asignatura \textbf{Teoría de Circuitos} es una introducción amplia incluida en todos los \href{http://www.etsidi.upm.es/Estudiantes/EstudiosTitulaciones/ETTitulosGrado/ETTitulosOficialesGrado}{Grados impartidos en la ETSIDI}. Expone los teoremas generales y métodos de análisis más importantes, estudiando los circuitos de corriente continua, corriente alterna monofásica y trifásica, e incluye una introducción breve al análisis del régimen transitorio.
\item La asignatura \textbf{Teoría de Circuitos II}, incluida sólo en el \href{http://www.etsidi.upm.es/Estudiantes/EstudiosTitulaciones/ETTitulosGrado/ETTitulosOficialesGrado/GradIngElectrica}{Grado de Ingeniería Eléctrica}, intensifica lo expuesto en \textbf{Teoría de Circuitos}, ampliando el estudio de generadores, técnicas de análisis, y sistemas polifásicos, e introduciendo los acoplamientos magnéticos y transformadores. El contenido del libro dedicado a esta asignatura está marcado con la etiqueta \textsuperscript{TC2}.
\item La asignatura \textbf{Teoría de Circuitos III}, incluida sólo en el \href{http://www.etsidi.upm.es/Estudiantes/EstudiosTitulaciones/ETTitulosGrado/ETTitulosOficialesGrado/GradIngElectrica}{Grado de Ingeniería Eléctrica}, expone el estudio del régimen transitorio, el análisis en frecuencia de circuitos, el análisis en variables de estado, cuadripolos, y una introducción a los componentes no lineales. El contenido del libro dedicado a esta asignatura está marcado con la etiqueta \textsuperscript{TC3}.
\end{itemize}

%%% Local Variables:
%%% mode: LaTex
%%% TeX-master: "TC.tex"
%%% End: 
