\usepackage[T1]{fontenc}
\usepackage[utf8]{inputenc}
\usepackage[a4paper]{geometry}
\geometry{verbose,tmargin=2.5cm,bmargin=2.5cm,lmargin=2.5cm,rmargin=2.5cm}
\pagestyle{Ruled}
\usepackage{array}
\usepackage{verbatim}
\usepackage{prettyref}
\usepackage{booktabs}
\usepackage{textcomp}
\usepackage{url}
\usepackage{amsmath}
\usepackage{chemarr}%flechas para reacciones químicas (SFER.tex)
\usepackage{graphicx}
\usepackage{amssymb}
\usepackage{nomencl}
\usepackage[usenames,dvipsnames]{xcolor}
\usepackage{enumitem}
% the following is useful when we have the old nomencl.sty package
% \providecommand{\printnomenclature}{\printglossary}
% \providecommand{\makenomenclature}{\makeglossary}
\makenomenclature

\usepackage{float}
\usepackage{tikz}

\usepackage{subfig}
%Configuración de los caption
\PassOptionsToPackage{caption=false}{subfig}%Evita que el paquete subfig lo descabale todo
\captiontitlefont{\itshape}
\captionnamefont{\scshape}
\captionstyle{\centering}
\hangcaption
\usepackage{cancel}
\usepackage{steinmetz}
\usepackage{diffcoeff}
\usepackage{mathtools}

\usepackage[spanish]{babel}
\addto\shorthandsspanish{\spanishdeactivate{~<>}}

\usepackage{hyperref}


\hypersetup{
    bookmarks=true,         % show bookmarks bar?
    unicode=true,          % non-Latin characters in Acrobat’s bookmarks
    bookmarksnumbered=false,
    bookmarksopen=false,
    breaklinks=true,
    backref=true,
    pdftoolbar=true,        % show Acrobat’s toolbar?
    pdfmenubar=true,        % show Acrobat’s menu?
    pdffitwindow=false,     % window fit to page when opened
    pdfstartview={FitH},    % fits the width of the page to the window
    pdftitle={Teoria de Circuitos},    % title
    pdfsubject={ETSIDI},   % subject of the document
    pdfcreator={Overleaf},   % creator of the document
    pdfproducer={LaTeX}, % producer of the document
    pdfnewwindow=true,      % links in new window
    pdfborder={0 0 0},
    colorlinks=true,       % false: boxed links; true: colored links
    linkcolor=Brown,          % color of internal links
    citecolor=BrickRed,        % color of links to bibliography
    filecolor=black,      % color of file links
    urlcolor=Blue           % color of external links 
}

\usepackage[output-complex-root = j]{siunitx}
\DeclareSIUnit\kWh{kWh}
\DeclareSIUnit{\watthour}{Wh}
\DeclareSIUnit\Wh{Wh}
\DeclareSIUnit{\voltampere}{VA}
\DeclareSIUnit{\var}{var}

\sisetup{per-mode=symbol-or-fraction}
%\usepackage{lscape}
\usepackage{mathpazo}%Letra palatino con fuentes para matemáticas
\usepackage{flafter}%obliga a que los flotantes aparezcan después de su referencia
\usepackage{memhfixc}

\raggedbottom
\sloppybottom
\clubpenalty=10000
\widowpenalty=10000

%\raggedbottomsection
\feetbelowfloat


\addto\captionsspanish{%
\def\tablename{Tabla}%
\def\listtablename{\'Indice de tablas}}


%\spanishdecimal{.} %Para que no lo sustituya automáticamente por comas
%\def\nompreamble{\addcontentsline{toc}{chapter}{\nomname}\markboth{\nomname}{\nomname}}

%Configuración de MEMOIR
%%Pone la fecha en SMALL CAPS y hacia la derecha
%%pagina 60 de memman.pdf
\pretitle{
  \vfill
  \centering \bfseries \scshape \HUGE \color{BrickRed}
}

\posttitle{\par}

\preauthor{
  \vfill
  \centering
  \large \scshape
}
\postauthor{\par }

\predate{\vfill \begin{flushright}\large\scshape}
\postdate{\par\end{flushright}\vfill}


\setsecnumdepth{subsection}


% \definecolor{ared}{rgb}{.647,.129,.149}
% \renewcommand{\colorchapnum}{\color{ared}}
% \renewcommand{\colorchaptitle}{\color{ared}}
% \chapterstyle{pedersen}
\chapterstyle{ger}

\setlength{\afterchapskip}{35pt}
\maxtocdepth{subsection}

%\setcounter{topnumber}{3}
%\setcounter{bottomnumber}{2}
%\setcounter{totalnumber}{4}
\renewcommand{\topfraction}{0.85}
\renewcommand{\bottomfraction}{0.5}
\renewcommand{\textfraction}{0.15}
\renewcommand{\floatpagefraction}{0.7}


\renewcommand{\textfloatsep}{10pt}%Espacio entre el flotante y el texto



\usepackage[framemethod=default]{mdframed} % Required for creating the theorem, definition, exercise and corollary boxes

% Exercise box	  
\newmdenv[skipabove=7pt,
skipbelow=7pt,
rightline=false,
leftline=false,
topline=false,
bottomline=false,
backgroundcolor=MidnightBlue!5,
linecolor=MidnightBlue,
innerleftmargin=5pt,
innerrightmargin=5pt,
innertopmargin=5pt,
innerbottommargin=5pt,
leftmargin=0cm,
rightmargin=0cm,
linewidth=4pt]{eBox}

\newtheorem{exerciseT}{Ejemplo}[chapter]

\newenvironment{example}{\begin{eBox}\begin{exerciseT}}{\hfill{\color{MidnightBlue}\tiny\ensuremath{\blacksquare}}\end{exerciseT}\end{eBox}}	

\usepackage{xparse} % For "overbrace/underbrace but with an arrow instead", from https://tex.stackexchange.com/questions/8720/overbrace-underbrace-but-with-an-arrow-instead

% Para poner flechas sobre los signos de igual, de aquí: https://tex.stackexchange.com/questions/8720/overbrace-underbrace-but-with-an-arrow-instead
\NewDocumentCommand{\overarrow}{O{=} O{\uparrow} m}{%
  \overset{\makebox[0pt]{\begin{tabular}{@{}c@{}}#3\\[0pt]\ensuremath{#2}\end{tabular}}}{#1}
}
\NewDocumentCommand{\underarrow}{O{=} O{\downarrow} m}{%
  \underset{\makebox[0pt]{\begin{tabular}{@{}c@{}}\ensuremath{#2}\\[0pt]#3\end{tabular}}}{#1}
}

\usepackage{rotating,stackengine,scalerel}
\newcommand\wye{\scalerel*{\stackengine{-1pt}{%
  \rotatebox[origin=c]{30}{\rule{10pt}{.9pt}}\kern-1pt%
  \rotatebox[origin=c]{-30}{\rule{10pt}{1.3pt}}}{%
  \rule{.9pt}{10pt}}{O}{c}{F}{F}{S}}{\Delta}} % https://tex.stackexchange.com/questions/481532/star-wye-electrical-connection-math-symbol
  
\newenvironment{remark}{\par\vspace{10pt}\small % Vertical white space above the remark and smaller font size
\begin{list}{}{
\leftmargin=35pt % Indentation on the left
\rightmargin=25pt}\item\ignorespaces % Indentation on the right
\makebox[-2.5pt]{\begin{tikzpicture}[overlay]
\node[draw=MidnightBlue!60,line width=1pt,circle,fill=MidnightBlue!25,font=\sffamily\bfseries,inner sep=2pt,outer sep=0pt] at (-15pt,0pt){\textcolor{MidnightBlue}{N}};\end{tikzpicture}} % Orange R in a circle
\advance\baselineskip -1pt}{\end{list}\vskip5pt} % Tighter line spacing and white space after remark


