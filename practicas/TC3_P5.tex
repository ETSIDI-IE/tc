\documentclass[a4paper,10pt]{article} % Uses article class in A4 format

%%%%%%%%%%%%%%%%%%%%%%%%%%%%%%%%%%%%%%%%%
% Homework Assignment Article
% LaTeX Template
% Version 1.3.5r (2018-02-16)
%
% This template has been downloaded from:
% /cl.uni-heidelberg.de/~zimmermann/
%
% Original author:
% Victor Zimmermann (zimmermann@cl.uni-heidelberg.de)
%
% License:
% CC BY-SA 4.0 (https://creativecommons.org/licenses/by-sa/4.0/)
%
%%%%%%%%%%%%%%%%%%%%%%%%%%%%%%%%%%%%%%%%%

%----------------------------------------------------------------------------------------
%----------------------------------------------------------------------------------------
%	FORMATTING
%----------------------------------------------------------------------------------------

\setlength{\parskip}{0pt}
\setlength{\parindent}{0pt}
\setlength{\voffset}{-15pt}

%----------------------------------------------------------------------------------------
%	PACKAGES AND OTHER DOCUMENT CONFIGURATIONS
%----------------------------------------------------------------------------------------

\usepackage[a4paper, margin=2.5cm]{geometry} % Sets margin to 2.5cm for A4 Paper
\usepackage[onehalfspacing]{setspace} % Sets Spacing to 1.5
\usepackage{subfigure}

\usepackage{diffcoeff}
\usepackage[T1]{fontenc} % Use European encoding
\usepackage[utf8]{inputenc} % Use UTF-8 encoding
\usepackage{charter} % Use the Charter font
\usepackage{microtype} % Slightly tweak font spacing for aesthetics

\usepackage[spanish]{babel} % Language hyphenation and typographical rules

\usepackage{amsthm, amsmath, amssymb} % Mathematical typesetting
\usepackage{marvosym, wasysym} % More symbols
\usepackage{float} % Improved interface for floating objects
\usepackage[final, colorlinks = true, 
            linkcolor = black, 
            citecolor = black,
            urlcolor = black]{hyperref} % For hyperlinks in the PDF
\usepackage{graphicx, multicol} % Enhanced support for graphics
\usepackage{xcolor} % Driver-independent color extensions
\usepackage{rotating} % Rotation tools
\usepackage{pseudocode} % Environment for specifying algorithms in a natural way
\usepackage{style/avm} % Environment for f-structures, !uses style file!
\usepackage{booktabs} % Enhances quality of tables

\usepackage{tikz-qtree} % Easy tree drawing tool
\tikzset{every tree node/.style={align=center,anchor=north},
         level distance=2cm} % Configuration for q-trees
\usepackage{style/btree} % Configuration for b-trees and b+-trees, !uses style file!

% \usepackage{titlesec} % Allows customization of titles
% \renewcommand\thesection{\arabic{section}.} % Arabic numerals for the sections
% \titleformat{\section}{\large}{\thesection}{1em}{}
% \renewcommand\thesubsection{\arabic{subsection})} % Arabicnumerals for subsections
% \titleformat{\subsection}{\large}{\thesubsection}{1em}{}
% \renewcommand\thesubsubsection{\roman{subsubsection}.} % Roman numbering for subsubsections
% \titleformat{\subsubsection}{\large}{\thesubsubsection}{1em}{}

\usepackage[all]{nowidow} % Removes widows

\usepackage[backend=biber,style=numeric,
            sorting=nyt, natbib=true]{biblatex} % Complete reimplementation of bibliographic facilities
\addbibresource{main.bib}
\usepackage{csquotes} % Context sensitive quotation facilities

\usepackage[yyyymmdd]{datetime} % Uses YEAR-MONTH-DAY format for dates
\renewcommand{\dateseparator}{-} % Sets dateseparator to '-'

\usepackage{fancyhdr} % Headers and footers
\pagestyle{fancy} % All pages have headers and footers
\fancyhead{}\renewcommand{\headrulewidth}{0pt} % Blank out the default header
%\fancyfoot[L]{\textsc{Práctica 6}} % Custom footer text
\fancyfoot[C]{} % Custom footer text
\fancyfoot[R]{\thepage} % Custom footer text

\newcommand{\note}[1]{\marginpar{\scriptsize \textcolor{red}{#1}}} % Enables comments in red on margin

\usepackage{hyperref}

\usepackage{siunitx}
\usepackage{mathpazo}
%----------------------------------------------------------------------------------------
\newcommand{\qucs}{\texttt{Qucs}}

\newcommand{\printtitle}[3]{
  \title{template_assignment} % Article title
  \fancyhead[C]{}
  \begin{minipage}{0.33\textwidth} % Left side of title section
    \raggedright
    \qucs\\ % Your lecture or course
    \footnotesize % Authors text size
    % \hfill\\ % Uncomment if right minipage has more lines
    Teoría de circuitos III % Your name, your matriculation number
    \medskip\hrule
  \end{minipage}
  \begin{minipage}{0.33\textwidth} % Center of title section
    \centering 
    \large % Title text size
    #1\\ % Assignment title and number
    \normalsize % Subtitle text size
    #2\\ % Assignment subtitle
  \end{minipage}
  \begin{minipage}{0.33\textwidth} % Right side of title section
    \raggedleft
    % 11/2022\\ % Date
    \footnotesize % Email text size
    % \hfill\\ % Uncomment if left minipage has more lines
    #3% Your email
    \medskip\hrule
  \end{minipage}
}

\begin{document}

\printtitle{Práctica 5}{Parámetros de cuadripolos}{2023/2024}

\vspace{1cm}

En esta práctica se van a extraer las diferentes familias de parámetros del cuadripolo resistivo de la figura:

\begin{minipage}{0.7\linewidth}
  \begin{center}
    \includegraphics{../figs/circuito_P5}
  \end{center}
\end{minipage}
\begin{minipage}{0.3\linewidth}
  Datos:

  $R_1 = \qty{10}{\ohm}$

  $R_2 = \qty{5}{\ohm}$

  $R_3 = \qty{40}{\ohm}$

  % $R_4 = \qty{20}{\ohm}$
\end{minipage}

\begin{enumerate}
\item En primer lugar, calcula las siguientes familias de parámetros:
  \begin{itemize}
  \item Parámetros de Impedancia
  \item Parámetros de Admitancia
  \item Parámetros Híbridos
  \item Parámetros Híbridos Inversos
  \end{itemize}
\item A continuación, obtén las mismas familias simulando en Qucs las correspondientes medidas al cuadripolo. Se recomienda encapsular el circuito dentro de un subcircuito (veáse anexo)
\item A partir de los parámetros del cuadripolo, calcula la resistencia de carga que hay que conectar a la salida para obtener la máxima transferencia de potencia, si el cuadripolo está alimentado en la entrada por un generador de tensión de corriente continua cuya resistencia interna es $R_g = \qty{5}{\ohm}$.
\item Mediante un barrido en Qucs comprueba que la resistencia calculada es la que obtiene la máxima potencia.
\end{enumerate}

\clearpage

\section*{Anexo: subcircuitos en Qucs}

Cuando un circuito se va a utilizar de forma repetida, se puede definir como un componente de tipo subcircuito e insertarlo dentro de otro circuito. El procedimiento es el siguiente:

\begin{enumerate}

\item Dibuja el circuito a encapsular asignando una variable a cada uno de los componentes. 

  \begin{center}
    \includegraphics[width=.7\linewidth]{../figs/CircuitoResistivo_Qucs.png}
  \end{center}

  Hay que añadir una conexión en cada uno de los cuatro terminales. Estas conexiones están numeradas, y serán los puertos del cuadripolo.

  \begin{center}
    \includegraphics[width=\linewidth]{../figs/InsertarConexion_qucs.png}
  \end{center}

\item Graba este circuito en un fichero sch, y no lo cierres.
\item En este mismo circuito, en el menú ``Archivo'' elige ``Edit Circuit Symbol''. Aparecerá un cuadripolo en lugar del circuito. Debajo de este cuadripolo hay una etiqueta con el texto ``SUB''. Haciendo doble click en esta etiqueta aparece un cuadro de diálogo en el que se pueden definir las variables del circuito y sus valores por defecto. En este cuadro añade las resistencias y sus valores (con unidades).

  \begin{center}
    \includegraphics[width=.5\linewidth]{../figs/variablesCuadripolo.png}
  \end{center}
  Para volver al circuito, nuevamente en el menú ``Archivo'' elige ``Edit Schematic''.
  
\item En un fichero nuevo, en el menú de ``Componentes'', selecciona ``File Components'' y ahí ``Subcircuit''. Inserta este componente y edita sus propiedades. En el cuadro de diálogo se puede seleccionar el fichero creado en los pasos anteriores.

  \begin{center}
    \includegraphics[width=.5\linewidth]{../figs/InsertarSubcircuito.png}
  \end{center}

Una vez insertado y enlazado al fichero, aparecen las resistencias del circuito con sus valores por defecto asignados.

  \begin{center}
    \includegraphics[width=.2\linewidth]{../figs/SubcircuitoValores.png}
  \end{center}
  
\item A este subcircuito se le pueden conectar otros componentes y construir un circuito completo. Es importante conectar sendos terminales de puesta a tierra en la entrada y en la salida.

  \begin{center}
    \includegraphics[width=\linewidth]{../figs/SubcircuitoTierra.png}
  \end{center}
  

\end{enumerate}

\end{document}
