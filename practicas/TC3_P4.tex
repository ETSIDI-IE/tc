\documentclass[a4paper,10pt]{article} % Uses article class in A4 format

%%%%%%%%%%%%%%%%%%%%%%%%%%%%%%%%%%%%%%%%%
% Homework Assignment Article
% LaTeX Template
% Version 1.3.5r (2018-02-16)
%
% This template has been downloaded from:
% /cl.uni-heidelberg.de/~zimmermann/
%
% Original author:
% Victor Zimmermann (zimmermann@cl.uni-heidelberg.de)
%
% License:
% CC BY-SA 4.0 (https://creativecommons.org/licenses/by-sa/4.0/)
%
%%%%%%%%%%%%%%%%%%%%%%%%%%%%%%%%%%%%%%%%%

%----------------------------------------------------------------------------------------
%----------------------------------------------------------------------------------------
%	FORMATTING
%----------------------------------------------------------------------------------------

\setlength{\parskip}{0pt}
\setlength{\parindent}{0pt}
\setlength{\voffset}{-15pt}

%----------------------------------------------------------------------------------------
%	PACKAGES AND OTHER DOCUMENT CONFIGURATIONS
%----------------------------------------------------------------------------------------

\usepackage[a4paper, margin=2.5cm]{geometry} % Sets margin to 2.5cm for A4 Paper
\usepackage[onehalfspacing]{setspace} % Sets Spacing to 1.5
\usepackage{subfigure}

\usepackage{diffcoeff}
\usepackage[T1]{fontenc} % Use European encoding
\usepackage[utf8]{inputenc} % Use UTF-8 encoding
\usepackage{charter} % Use the Charter font
\usepackage{microtype} % Slightly tweak font spacing for aesthetics

\usepackage[spanish]{babel} % Language hyphenation and typographical rules

\usepackage{amsthm, amsmath, amssymb} % Mathematical typesetting
\usepackage{marvosym, wasysym} % More symbols
\usepackage{float} % Improved interface for floating objects
\usepackage[final, colorlinks = true, 
            linkcolor = black, 
            citecolor = black,
            urlcolor = black]{hyperref} % For hyperlinks in the PDF
\usepackage{graphicx, multicol} % Enhanced support for graphics
\usepackage{xcolor} % Driver-independent color extensions
\usepackage{rotating} % Rotation tools
\usepackage{pseudocode} % Environment for specifying algorithms in a natural way
\usepackage{style/avm} % Environment for f-structures, !uses style file!
\usepackage{booktabs} % Enhances quality of tables

\usepackage{tikz-qtree} % Easy tree drawing tool
\tikzset{every tree node/.style={align=center,anchor=north},
         level distance=2cm} % Configuration for q-trees
\usepackage{style/btree} % Configuration for b-trees and b+-trees, !uses style file!

% \usepackage{titlesec} % Allows customization of titles
% \renewcommand\thesection{\arabic{section}.} % Arabic numerals for the sections
% \titleformat{\section}{\large}{\thesection}{1em}{}
% \renewcommand\thesubsection{\arabic{subsection})} % Arabicnumerals for subsections
% \titleformat{\subsection}{\large}{\thesubsection}{1em}{}
% \renewcommand\thesubsubsection{\roman{subsubsection}.} % Roman numbering for subsubsections
% \titleformat{\subsubsection}{\large}{\thesubsubsection}{1em}{}

\usepackage[all]{nowidow} % Removes widows

\usepackage[backend=biber,style=numeric,
            sorting=nyt, natbib=true]{biblatex} % Complete reimplementation of bibliographic facilities
\addbibresource{main.bib}
\usepackage{csquotes} % Context sensitive quotation facilities

\usepackage[yyyymmdd]{datetime} % Uses YEAR-MONTH-DAY format for dates
\renewcommand{\dateseparator}{-} % Sets dateseparator to '-'

\usepackage{fancyhdr} % Headers and footers
\pagestyle{fancy} % All pages have headers and footers
\fancyhead{}\renewcommand{\headrulewidth}{0pt} % Blank out the default header
%\fancyfoot[L]{\textsc{Práctica 6}} % Custom footer text
\fancyfoot[C]{} % Custom footer text
\fancyfoot[R]{\thepage} % Custom footer text

\newcommand{\note}[1]{\marginpar{\scriptsize \textcolor{red}{#1}}} % Enables comments in red on margin

\usepackage{hyperref}

\usepackage{siunitx}
\usepackage{mathpazo}
%----------------------------------------------------------------------------------------
\newcommand{\qucs}{\texttt{Qucs}}

\newcommand{\printtitle}[3]{
  \title{template_assignment} % Article title
  \fancyhead[C]{}
  \begin{minipage}{0.33\textwidth} % Left side of title section
    \raggedright
    \qucs\\ % Your lecture or course
    \footnotesize % Authors text size
    % \hfill\\ % Uncomment if right minipage has more lines
    Teoría de circuitos III % Your name, your matriculation number
    \medskip\hrule
  \end{minipage}
  \begin{minipage}{0.33\textwidth} % Center of title section
    \centering 
    \large % Title text size
    #1\\ % Assignment title and number
    \normalsize % Subtitle text size
    #2\\ % Assignment subtitle
  \end{minipage}
  \begin{minipage}{0.33\textwidth} % Right side of title section
    \raggedleft
    % 11/2022\\ % Date
    \footnotesize % Email text size
    % \hfill\\ % Uncomment if left minipage has more lines
    #3% Your email
    \medskip\hrule
  \end{minipage}
}

\begin{document}

\printtitle{Práctica 4}{Resonancia}{2024/2024}

\vspace{1cm} Un circuito RLC serie está alimentado por una fuente de
tensión alterna $U = \qty{1}{\volt}$ y frecuencia variable.

En esta práctica se analiza este circuito en las cercanías de la
frecuencia de resonancia. Para este análisis se debe emplear el modo
de simulación AC.
	
\begin{enumerate}
\item En primer lugar, analizaremos el circuito con unos elementos
  ideales cuyos valores son $L =\qty{21.4}{\milli\henry}$,
  $C=\qty{1.41}{\micro\farad}$ y $R=\qty{18}{\ohm}$ (en este orden de
  conexión), tomando la tensión de salida del circuito en bornes de la
  resistencia. En estas condiciones:

  \begin{center}
    \includegraphics[height=0.2\textheight]{../figs/circuito_P4}
  \end{center}

  \begin{enumerate}
  \item Determina el valor de los siguientes parámetros: pulsación y
    frecuencia de resonancia, factor de calidad en la pulsación de
    resonancia, y ancho de banda.
  \item Utilizando el modo de simulación \textsc{AC} simula el
    comportamiento del circuito en las cercanías de la frecuencia de
    resonancia empleando un rango de frecuencias que incluya el ancho
    de banda del circuito.
  \end{enumerate}
	
\item En segundo lugar analizaremos el funcionamiento de un circuito
  RLC con elementos reales. Este circuito está compuesto por una
  bobina con $L =\qty{21.4}{\milli\henry}$ y factor de calidad 10, un
  condensador $C=\qty{1.41}{\micro\farad}$ y factor de calidad 10, y
  una resistencia $R=\qty{18}{\ohm}$. El orden de conexión es
  nuevamente $L$, $C$ y $R$, y la tensión de salida se toma en los
  bornes de la resistencia. Con este circuito vuelve a realizar los
  cálculos de los parámetros y la simulación del circuito, comparando
  los resultados con los obtenidos en el circuito de elementos
  ideales.
  
  \begin{center}
    \includegraphics[height=0.2\textheight]{../figs/circuito_P4_2}
  \end{center}
	
\item En tercer lugar, emplea el modo de simulación \textsc{Sweep}
  para analizar el efecto del factor de calidad de los componentes. En
  primer lugar, realiza un barrido del factor de calidad de la bobina
  desde 10 hasta 100, y a continuación otro barrido del factor de
  calidad del condensador desde 10 hasta 100. Representa los
  resultados de cada barrido tanto en forma gráfica como en una tabla.
	
  % \item Finalmente, analizaremos el efecto de realizar medidas con
  %   aparatos de medida reales. En este caso emplearemos los mismos
  %   elementos del primer circuito pero conectados en orden $R$, $L$ y
  %   $C$, tomando la salida de tensión en el condensador. Con este
  %   circuito vuelve a realizar los cálculos de los parámetros y la
  %   simulación del circuito, considerando que el voltímetro es
  %   ideal.

  %   \begin{center}
  %     \includegraphics[height=0.2\textheight]{../figs/circuito_P4_3}
  %   \end{center}

  %   A continuación, modelaremos un voltímetro real mediante la
  %   inclusión de una resistencia en paralelo con el
  %   voltímetro. Considerando un valor de $R_v = \qty{50}{\kilo\ohm}$,
  %   obtén los valores de los parámetros y compara los resultados con
  %   el caso anterior. Realiza una simulación del circuito con un
  %   barrido de valores para $R_v$ desde $\qty{50}{\kilo\ohm}$ hasta
  %   $\qty{500}{\kilo\ohm}$, comparando con los resultados obtenidos
  %   con un voltímetro ideal.

  %   \begin{center}
  %     \includegraphics[height=0.2\textheight]{../figs/circuito_P4_4}
  %   \end{center}
	
  \end{enumerate}
	

\end{document}
