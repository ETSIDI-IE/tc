\documentclass[a4paper,10pt]{article} % Uses article class in A4 format

%%%%%%%%%%%%%%%%%%%%%%%%%%%%%%%%%%%%%%%%%
% Homework Assignment Article
% LaTeX Template
% Version 1.3.5r (2018-02-16)
%
% This template has been downloaded from:
% /cl.uni-heidelberg.de/~zimmermann/
%
% Original author:
% Victor Zimmermann (zimmermann@cl.uni-heidelberg.de)
%
% License:
% CC BY-SA 4.0 (https://creativecommons.org/licenses/by-sa/4.0/)
%
%%%%%%%%%%%%%%%%%%%%%%%%%%%%%%%%%%%%%%%%%

%----------------------------------------------------------------------------------------
%----------------------------------------------------------------------------------------
%	FORMATTING
%----------------------------------------------------------------------------------------

\setlength{\parskip}{0pt}
\setlength{\parindent}{0pt}
\setlength{\voffset}{-15pt}

%----------------------------------------------------------------------------------------
%	PACKAGES AND OTHER DOCUMENT CONFIGURATIONS
%----------------------------------------------------------------------------------------

\usepackage[a4paper, margin=2.5cm]{geometry} % Sets margin to 2.5cm for A4 Paper
\usepackage[onehalfspacing]{setspace} % Sets Spacing to 1.5
\usepackage{subfigure}

\usepackage{diffcoeff}
\usepackage[T1]{fontenc} % Use European encoding
\usepackage[utf8]{inputenc} % Use UTF-8 encoding
\usepackage{charter} % Use the Charter font
\usepackage{microtype} % Slightly tweak font spacing for aesthetics

\usepackage[spanish]{babel} % Language hyphenation and typographical rules

\usepackage{amsthm, amsmath, amssymb} % Mathematical typesetting
\usepackage{marvosym, wasysym} % More symbols
\usepackage{float} % Improved interface for floating objects
\usepackage[final, colorlinks = true, 
            linkcolor = black, 
            citecolor = black,
            urlcolor = black]{hyperref} % For hyperlinks in the PDF
\usepackage{graphicx, multicol} % Enhanced support for graphics
\usepackage{xcolor} % Driver-independent color extensions
\usepackage{rotating} % Rotation tools
\usepackage{pseudocode} % Environment for specifying algorithms in a natural way
\usepackage{style/avm} % Environment for f-structures, !uses style file!
\usepackage{booktabs} % Enhances quality of tables

\usepackage{tikz-qtree} % Easy tree drawing tool
\tikzset{every tree node/.style={align=center,anchor=north},
         level distance=2cm} % Configuration for q-trees
\usepackage{style/btree} % Configuration for b-trees and b+-trees, !uses style file!

% \usepackage{titlesec} % Allows customization of titles
% \renewcommand\thesection{\arabic{section}.} % Arabic numerals for the sections
% \titleformat{\section}{\large}{\thesection}{1em}{}
% \renewcommand\thesubsection{\arabic{subsection})} % Arabicnumerals for subsections
% \titleformat{\subsection}{\large}{\thesubsection}{1em}{}
% \renewcommand\thesubsubsection{\roman{subsubsection}.} % Roman numbering for subsubsections
% \titleformat{\subsubsection}{\large}{\thesubsubsection}{1em}{}

\usepackage[all]{nowidow} % Removes widows

\usepackage[backend=biber,style=numeric,
            sorting=nyt, natbib=true]{biblatex} % Complete reimplementation of bibliographic facilities
\addbibresource{main.bib}
\usepackage{csquotes} % Context sensitive quotation facilities

\usepackage[yyyymmdd]{datetime} % Uses YEAR-MONTH-DAY format for dates
\renewcommand{\dateseparator}{-} % Sets dateseparator to '-'

\usepackage{fancyhdr} % Headers and footers
\pagestyle{fancy} % All pages have headers and footers
\fancyhead{}\renewcommand{\headrulewidth}{0pt} % Blank out the default header
%\fancyfoot[L]{\textsc{Práctica 6}} % Custom footer text
\fancyfoot[C]{} % Custom footer text
\fancyfoot[R]{\thepage} % Custom footer text

\newcommand{\note}[1]{\marginpar{\scriptsize \textcolor{red}{#1}}} % Enables comments in red on margin

\usepackage{hyperref}

\usepackage{siunitx}
\usepackage{mathpazo}
%----------------------------------------------------------------------------------------
\newcommand{\qucs}{\texttt{Qucs}}

\newcommand{\printtitle}[3]{
  \title{template_assignment} % Article title
  \fancyhead[C]{}
  \begin{minipage}{0.33\textwidth} % Left side of title section
    \raggedright
    \qucs\\ % Your lecture or course
    \footnotesize % Authors text size
    % \hfill\\ % Uncomment if right minipage has more lines
    Teoría de circuitos III % Your name, your matriculation number
    \medskip\hrule
  \end{minipage}
  \begin{minipage}{0.33\textwidth} % Center of title section
    \centering 
    \large % Title text size
    #1\\ % Assignment title and number
    \normalsize % Subtitle text size
    #2\\ % Assignment subtitle
  \end{minipage}
  \begin{minipage}{0.33\textwidth} % Right side of title section
    \raggedleft
    % 11/2022\\ % Date
    \footnotesize % Email text size
    % \hfill\\ % Uncomment if left minipage has more lines
    #3% Your email
    \medskip\hrule
  \end{minipage}
}

\begin{document}

\printtitle{Práctica 6}{Conexión de cuadripolos}{2023/2024}

\vspace{0.3cm}

En la figura siguiente se representan un circuito RL y un circuito RC. En esta práctica se analiza el comportamiento en frecuencia de estos dos circuitos en diferentes conexiones.

\begin{center}
  \includegraphics[height=0.09\textheight]{../figs/circuito_P6}
\end{center}

\begin{enumerate}
\item En primer lugar, calcula los parámetros de transmisión del circuito RL y del circuito RC, particularizando para $R_L = R_C = \qty{1}{\ohm}$, $L = \qty{1}{\milli\henry}$ y $C = \qty{1}{\milli\farad}$.
  
\item A continuación, calcula los parámetros de transmisión y la función de transferencia de los siguientes circuitos (RL-RC, RC-RL, RL-RL, y RC-RC), consistentes en interconexiones de estos circuitos simples. Indica el tipo de filtro resultante en cada caso. Se recomienda dibujar un diagrama de Bode.

  \begin{minipage}{0.5\linewidth}
    \begin{center}
      \includegraphics[height=0.09\textheight]{../figs/circuito_P6_RLRC}
    \end{center}
  \end{minipage}
  \begin{minipage}{0.5\linewidth}
    \begin{center}
      \includegraphics[height=0.09\textheight]{../figs/circuito_P6_RCRL}
    \end{center}
  \end{minipage}

  \begin{minipage}{0.5\linewidth}
    \begin{center}
      \includegraphics[height=0.09\textheight]{../figs/circuito_P6_RLRL}
    \end{center}
  \end{minipage}
  \begin{minipage}{0.5\linewidth}
    \begin{center}
      \includegraphics[height=0.09\textheight]{../figs/circuito_P6_RCRC}
    \end{center}
  \end{minipage}

\item Mediante Qucs, realiza la simulación de cada uno de los cuatro circuitos y representa el módulo y fase de la función de transferencia\footnote{En Qucs están disponibles las funciones \texttt{dB} y \texttt{phase} que calculan el módulo en decibelios y la fase de una función de transferencia, respectivamente.}. La simulación se debe realizar empleando el modo AC y un \textbf{barrido de tipo logarítmico para la frecuencia}, y los resultados se representan en un diagrama con el eje X logarítmico. 

\item En Qucs es posible modelar un cuadripolo del que se conoce una familia de parámetros. Para realizarlo, en el menú \texttt{Componentes} > \texttt{Lumped Components} elegimos el componente ``\texttt{Ecuación de componente 2-port RF}'' (al final de la lista de opciones). Editamos las propiedades y elegimos ``A'' como tipo de parámetros. Rellenamos los resultados obtenidos en el primer apartado para el circuito RL, y repetimos para el circuito RC\footnote{En Qucs se debe usar la S mayúscula. Por ejemplo, \texttt{R1 + S * L1}.}. Estas cajas se pueden interconectar entre sí para conformar los circuitos del apartado 2 y simular su funcionamiento.

  \begin{center}
    \includegraphics[height=0.15\textheight]{../figs/2-port.png}
  \end{center}

\item Compara los resultados obtenidos en los puntos 2, 3 y 4.

\end{enumerate}

\section*{Recordatorio}
\begin{itemize}
\item Los parámetros transmisión de un cuadripolo en L invertida son:

  
  \begin{minipage}{0.5\linewidth}
    \begin{center}
      \includegraphics[height=0.1\textheight]{../figs/circuito_P6_impedancia}
    \end{center}
  \end{minipage}
  \begin{minipage}{0.5\linewidth}
    \[
      [ABCD] = \left[
        \
        \begin{array}{cc}
          1 + \mathbf{Z}_1/\mathbf{Z}_2 & \mathbf{Z}_1\\
          1/\mathbf{Z}_2 & 1
        \end{array}
      \right]
    \]
  \end{minipage}

\item La función de transferencia es el inverso del parámetro A:

  \[
    H(\mathbf{s}) = 1/A(\mathbf{s})
  \]

\item Los parámetros de transmisión de una asociación de cuadripolos en cascada se calculan como producto de las matrices:

  \[
    [ABCD] = [ABCD]_1 \cdot [ABCD]_2
  \]

\item   La función de transferencia de una asociación de cuadripolos \textbf{no} se puede calcular como producto de las funciones de transferencia, porque la salida del circuito 1 no está en abierto.

  \[
    H(\mathbf{s}) \neq H_1(\mathbf{s}) \cdot H_2(\mathbf{s})
  \]
  Se puede obtener a partir de los parámetros de transmisión de la asociación:
  \[
    H(\mathbf{s}) = 1/A(\mathbf{s})
  \]

\end{itemize}

\end{document}
