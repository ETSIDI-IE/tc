\documentclass[a4paper,10pt]{article} % Uses article class in A4 format

%%%%%%%%%%%%%%%%%%%%%%%%%%%%%%%%%%%%%%%%%
% Homework Assignment Article
% LaTeX Template
% Version 1.3.5r (2018-02-16)
%
% This template has been downloaded from:
% /cl.uni-heidelberg.de/~zimmermann/
%
% Original author:
% Victor Zimmermann (zimmermann@cl.uni-heidelberg.de)
%
% License:
% CC BY-SA 4.0 (https://creativecommons.org/licenses/by-sa/4.0/)
%
%%%%%%%%%%%%%%%%%%%%%%%%%%%%%%%%%%%%%%%%%

%----------------------------------------------------------------------------------------
%----------------------------------------------------------------------------------------
%	FORMATTING
%----------------------------------------------------------------------------------------

\setlength{\parskip}{0pt}
\setlength{\parindent}{0pt}
\setlength{\voffset}{-15pt}

%----------------------------------------------------------------------------------------
%	PACKAGES AND OTHER DOCUMENT CONFIGURATIONS
%----------------------------------------------------------------------------------------

\usepackage[a4paper, margin=2.5cm]{geometry} % Sets margin to 2.5cm for A4 Paper
\usepackage[onehalfspacing]{setspace} % Sets Spacing to 1.5
\usepackage{subfigure}

\usepackage{diffcoeff}
\usepackage[T1]{fontenc} % Use European encoding
\usepackage[utf8]{inputenc} % Use UTF-8 encoding
\usepackage{charter} % Use the Charter font
\usepackage{microtype} % Slightly tweak font spacing for aesthetics

\usepackage[spanish]{babel} % Language hyphenation and typographical rules

\usepackage{amsthm, amsmath, amssymb} % Mathematical typesetting
\usepackage{marvosym, wasysym} % More symbols
\usepackage{float} % Improved interface for floating objects
\usepackage[final, colorlinks = true, 
            linkcolor = black, 
            citecolor = black,
            urlcolor = black]{hyperref} % For hyperlinks in the PDF
\usepackage{graphicx, multicol} % Enhanced support for graphics
\usepackage{xcolor} % Driver-independent color extensions
\usepackage{rotating} % Rotation tools
\usepackage{pseudocode} % Environment for specifying algorithms in a natural way
\usepackage{style/avm} % Environment for f-structures, !uses style file!
\usepackage{booktabs} % Enhances quality of tables

\usepackage{tikz-qtree} % Easy tree drawing tool
\tikzset{every tree node/.style={align=center,anchor=north},
         level distance=2cm} % Configuration for q-trees
\usepackage{style/btree} % Configuration for b-trees and b+-trees, !uses style file!

% \usepackage{titlesec} % Allows customization of titles
% \renewcommand\thesection{\arabic{section}.} % Arabic numerals for the sections
% \titleformat{\section}{\large}{\thesection}{1em}{}
% \renewcommand\thesubsection{\arabic{subsection})} % Arabicnumerals for subsections
% \titleformat{\subsection}{\large}{\thesubsection}{1em}{}
% \renewcommand\thesubsubsection{\roman{subsubsection}.} % Roman numbering for subsubsections
% \titleformat{\subsubsection}{\large}{\thesubsubsection}{1em}{}

\usepackage[all]{nowidow} % Removes widows

\usepackage[backend=biber,style=numeric,
            sorting=nyt, natbib=true]{biblatex} % Complete reimplementation of bibliographic facilities
\addbibresource{main.bib}
\usepackage{csquotes} % Context sensitive quotation facilities

\usepackage[yyyymmdd]{datetime} % Uses YEAR-MONTH-DAY format for dates
\renewcommand{\dateseparator}{-} % Sets dateseparator to '-'

\usepackage{fancyhdr} % Headers and footers
\pagestyle{fancy} % All pages have headers and footers
\fancyhead{}\renewcommand{\headrulewidth}{0pt} % Blank out the default header
%\fancyfoot[L]{\textsc{Práctica 6}} % Custom footer text
\fancyfoot[C]{} % Custom footer text
\fancyfoot[R]{\thepage} % Custom footer text

\newcommand{\note}[1]{\marginpar{\scriptsize \textcolor{red}{#1}}} % Enables comments in red on margin

\usepackage{hyperref}

\usepackage{siunitx}
\usepackage{mathpazo}
%----------------------------------------------------------------------------------------
\newcommand{\qucs}{\texttt{Qucs}}

\newcommand{\printtitle}[3]{
  \title{template_assignment} % Article title
  \fancyhead[C]{}
  \begin{minipage}{0.33\textwidth} % Left side of title section
    \raggedright
    \qucs\\ % Your lecture or course
    \footnotesize % Authors text size
    % \hfill\\ % Uncomment if right minipage has more lines
    Teoría de circuitos III % Your name, your matriculation number
    \medskip\hrule
  \end{minipage}
  \begin{minipage}{0.33\textwidth} % Center of title section
    \centering 
    \large % Title text size
    #1\\ % Assignment title and number
    \normalsize % Subtitle text size
    #2\\ % Assignment subtitle
  \end{minipage}
  \begin{minipage}{0.33\textwidth} % Right side of title section
    \raggedleft
    % 11/2022\\ % Date
    \footnotesize % Email text size
    % \hfill\\ % Uncomment if left minipage has more lines
    #3% Your email
    \medskip\hrule
  \end{minipage}
}

\begin{document}
% ------------------------------------------------------------------------------------
% TITLE SECTION
% ----------------------------------------------------------------------------------------
\printtitle{Práctica 2}{Transitorios de primer orden}{2023/2024}

% ----------------------------------------------------------------------------------------
% ARTICLE CONTENTS
% ----------------------------------------------------------------------------------------
\vspace{1cm}


El circuito de la figura se encuentra en régimen permanente. En el
instante $t=0$ se abre el interruptor. Las variables de interés son
$u_1$ y $u_2$.
\begin{center}
  \includegraphics{../figs/ej2_BT4.pdf}
\end{center}

El procedimiento a seguir es:

\begin{enumerate}
\item Calcula:
  \begin{itemize}
  \item $u_1(0^-)$ y $u_2(0^-)$
  \item $u_1(\infty)$ y $u_2(\infty)$
  \item $u_1(t)$ y $u_2(t)$ para $t > 0$.
  \end{itemize}
\item Simula el circuito en {\qucs} obteniendo el comportamiento de
  las variables $u_1$ y $u_2$. Para poder comprobar el funcionamiento
  antes de la apertura del interruptor, configura este elemento para
  que se abra a partir de $t = \qty{1}{\milli\second}$. Incluye en la
  simulación las ecuaciones correspondientes para mostrar los
  resultados de los cálculos del punto 1.
\item Compara los resultados obtenidos mediante la resolución
  analítica y mediante la simulación empleando gráficas y tablas.
\item Sustituye la fuente de tensión por un generador de corriente
  alterna de \qty{15}{\volt} de valor pico, y \qty{500}{\hertz} de
  frecuencia. Simula el circuito alimentado por este generador y
  compara el comportamiento con el circuito anterior.
\item Empleando nuevamente el circuito original
  (con generador de corriente continua),
  analiza el comportamiento ante variaciones de las resistencias $R_1$
  y $R_2$. Para este análisis debes emplear el modo de simulación
  \texttt{Sweep} para hacer un barrido de valores. En primer lugar
  realiza un barrido de $R_1$ con valores comprendidos entre
  \qty{10}{\ohm} y \qty{2}{\kilo\ohm} manteniendo fijo el valor de
  $R_2 = \qty{100}{\ohm}$. A continuación, manteniendo fijo el valor
  de $R_1 = \qty{200}{\ohm}$ realiza un barrido de $R_2$ con valores
  comprendidos entre \qty{10}{\ohm} y \qty{1}{\kilo\ohm}. Compara los
  resultados de ambos barridos con los obtenidos en la resolución
  analítica y la simulación del circuito original.
\end{enumerate}

\end{document}
